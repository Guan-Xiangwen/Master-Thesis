%! Tex root: ../master.tex
\newpage
A fact is that: Dp-branes are BPS states.
That is, preserves half of the supersymmetries (16 out of 32).

\begin{correct}
I should highlight the major questions that we have in mind:
\begin{itemize}
	\item Consider the dilaton profile $\Phi(r)$ of the Dp-brane geometry.
		Notice that there is a scaling as $r\to 0$.
		How this scaling relates to the property of the Yang-Mills coupling of the field theory on the Dp-branes.
	\item Consider the case of D(-1)-brane.
		In this case, the field theory is the IKKT matrix model.
		Dose our understanding about the IKKT matches with the dilaton profile?
\end{itemize}
\end{correct}

The discussion about the energy scale of the field theory in
\pdfref{IMSY98}
based on a parameter (it may be interesting to think about the physical interpretation of $\alpha'$ giving the correct dimension of energy.)
\[
U = \frac{r}{\alpha'}
.\] 
According to the explanation given in the paper,
$r$ could be understood as 
the expectation value of the collective coordinate of the Dp-branes.
There is a point that needs to be clarified.
For a D0-brane, the transverse coordinate is $X_1,\cdots,X_9$.
Then $r$ should be understood as $r = \left<X_i \right>,i=1,\cdots,9$
or $r^2 = \left<\sum_i X_i^2 \right>$.
The second case seems to make more sense.
But what do we know about $r$ from the field theory perspective?

It's important to clarify the validity regime of the Yang-Mills and the supergravity solution.
Both are the $\alpha'\to 0$ limit of string theory, which enables a field theory description of string states.
The validity regime of Yang-Mills is defined by the small effective coupling.
Or more precisely, this is where one can trust the perturbative calculation in Yang-Mills.
This regime is specified by the energy scale $U$
\[
	U \gg (g_{YM}^2 N)^{1 / (3-p)},\quad p<3
.\] 
There is an point that is not clear.
If $U$ is an expectation value in field theory,
it should be in principle determined from the theory itself.
That is, it's not a value that is independent of $g_{YM}$ and $N$.
Also, $U$ actually is related the transverse fluctuation of the Dp-branes.
A large value of $U$ seems imply that the separation between the Dp-branes can be large.
However, as noted in \pdfref{IMSY98}, $\alpha'\to 0$ actually implies that
any finite $U$ will correspond to a distance at substringy scale.
This evokes the question: the following discussion of supergravity also works in such a substringy scale?

The regime where the supergravity solution being valid is featured by two conditions: small curvature $\alpha' R \ll 1$ and small string coupling $e^\phi \ll 1$.
The first condition is just opposite to the perturbative Yang-Mills regime;
while the second gives a lower bound for $U$
\[
	(g_{YM}^2)^{1 / (3-p)} N^{1 / (7-p)} \ll U \ll (g_{YM}^2 N)^{1 / (3-p)}
.\] 
What we are interesting to discuss is how $U$ scaling with $N$?
