%! Tex root: ../master.tex

\begin{todo}
	Think about the scaling of the Yang-Mills coupling $g_{\text{YM}}$.	
	We have a good understanding of its classical dimension in field theory.
	Now think about it in the matrix model.
	\begin{itemize}
		\item Consider first the bosonic part of IKKT.
			Can you specifiy the scaling of $g_{\text{YM}}$ with respect to $N$?
		\item The idea behind this simplification is 
			to first ignore the terms that are required by susy. 
			Susy will mild the scaling beyond leading order.
			If we are only interested in the leading order behavior,
			then it may be reasonable to ignore those terms?
	\end{itemize}
\end{todo}

The IKKT model is essentially a matrix integral.
One could write it down, stare it without doing any other thing.
If one only have pen and paper, then the next thing that one could think about is: ``What's the symmetries of this model?"
This is a good question.
For people who learn something about QFT, ``symmetry'' seems a magic word.
Without trying to ask any question like ``what's the function''
one could talk a lot about the model already by just discussing the ``symmetry'', which usually just an appearance of the model.

A proper question to ask is always important.
Imagine all possible matrices.
This is a straightforward starting point, however not many structures can be revealed at this point.
Diagonal matrices are special, obviously.
Or put it generally, the commutative set of matrices is special.
Do you want to diagonalize them?
Before doing this, it seems natural to have an understanding of the unitary orbit in the whole matrix space.
Of course, I can't imagine that orbit.
But there must be something that I can say about it in general.
For example, the fixed points of the action must be interesting.
This will lead to some zeros or poles? after diagonalization.

In one matrix model, we have a good control over the ``volume of unitary orbit''.
If there are multiple matrices, the difficulty is that
you don't have an easy parameterization of all of the matrices up to a unitary orbit.
It's always not easy to talk about two matrices (or more) that are not simultaneously diagonalizable.
In the sense that how you parameterize them.
Or in other words, it's hard to assign a coordinate for such a quotient space.
Is there a measure of how hard two matrices could be simultaneously diagonalized?
Or there are some features about large N Lie algebra that we can use?
No, the crucial point seems not here.

What's the idea behind changing the scope of the space of all matrices to a subspace or a neighbourhood around a point that should be easier to deal with.
Other points may be not so relevant for the question?
To deal with the scaling, we tend to look at a straightline orbit.
If two matrices commute, then on such a line orbit, they still commute.
If they do not commute, their ``distance'' will change along the orbit.
``Highly non-commutative matrix'' may not contribute too much to the partition function?
I'm not sure about this point, because it feels like non-commutative matrix are the majority of the whole matrix space.
These are all need to be considered.
