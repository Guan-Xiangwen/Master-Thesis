%! Tex root: ../master.tex

We want to write
\[
	\mathrm{Tr} \left( \overline{\varphi} \Gamma^\mu [A_\mu,\varphi] \right) 
\] 
in the Majorana basis,
and express the integration result (over $\varphi$)
as the Puffafin of $\mathrm{Ad}(A_\mu)$.

Because we are going to use the basis $\{\xi^{(k)}\}_{k=1,\cdots,16}$,
the first thing to figure out is how different products
\[
	\overline{\xi^{(k)}} \Gamma^{a\pm} \mathfrak{B}_a^{\pm} \xi^{(l)}
\] 
appearing in the action $\mathrm{Tr} \left( \overline{\varphi} \Gamma^\mu [A_\mu,\varphi] \right) $.

Let's first check the product without $\Gamma^{a\pm}$.
Consider for $\zeta^{\mathbf{s}}$
\[
	\overline{\zeta^{\mathbf{s}'}} \zeta^{\mathbf{s}}
		= (\zeta^{\mathbf{s}'})^{\text{T}} C \zeta^{\mathbf{s}}
.\] 
By using the following property of the $C$ matrix:
		\[
				C \Sigma^{\mu\nu} C^{-1} = - (\Sigma^{\mu\nu})^{\text{T}}
			,\] 
One can derive that
\[
					s_a (\zeta^{(\mathbf{s}')})^{\text{T}} C \zeta^{(\mathbf{s})}
						= - s'_a (\zeta^{(\mathbf{s}')})^{\text{T}} C \zeta^{(\mathbf{s})}
						,\quad a=0,1,2,3,4
					.\] 
					Therefore, to get non-vanishing result, 
					$\mathbf{s}$ must be opposite to $\mathbf{s}'$.
Does the similar thing holds for the $B^{-1} \zeta^{\mathbf{s}*}$?
We need to consider the following possibilities
\[
	\overline{\zeta^{(\mathbf{s}')}} (B^{-1} \zeta^{(\mathbf{s})*}),\quad
	\overline{B^{-1} \zeta^{(\mathbf{s}')*}} \zeta^{(\mathbf{s})},\quad
	\overline{B^{-1} \zeta^{(\mathbf{s}')*}} 
	(B^{-1} \zeta^{(\mathbf{s})*})
.\] 
We know that $B^{-1} \zeta^*$ has the same Lorentz transformation property as $\zeta$,
so the same conclusion holds.

The next thing to consider is how $\Gamma^{a\pm}$ acting on $\zeta^{(\mathbf{s})}$ and $B^{-1} \zeta^{(\mathbf{s})*}$.
$\Gamma^{a\pm}$ simply raise or lower the $s_a$ lin $\zeta^{(\mathbf{s})}$.
However, when acting on $B^{-1} \zeta^{(\mathbf{s})*}$,
it will have an opposite effect for $a=1,2,3,4$
because the complex conjugate will interchange $\pm$;
for $a=0$ it's the same.

\begin{correct}
	\begin{equation}
		\overline{\xi^{(1)}} (\Gamma^{1+} \mathfrak{B}_1^+
		+ \Gamma^{1-} \mathfrak{B}_1^- )\xi^{(12)}
	\end{equation}
	The prolem is what does this mean for the Grassmann integration?
	Is it possible to write $(\cdots)\xi^{(12)}$ as $\xi^{(1)}$
	Try to find a basis of Gamma-matrices such that it has a simple action on the Majorana basis. (simply take one state to another state.)
\end{correct}

When writing down above formula,
we always have in mind that the spinors are written as a ``column vector''.
The entries are taken to be Grassmann number.
The transpose is understood as acting on the ``column vector''.
But when considering the integration,
it's not the case that we integrate over each entry.
Instead, we define the Grassmann integration to along the direction
specified by the choice of basis
\[
	[\mathrm{d}\xi^{(k)}][\mathrm{d}\chi^{(k)}]
.\] 
Note that $(\xi^{(1)})^{\text{T}} [\overline{B}_0,\xi^{(1)}]$
will not contribute to the integration due to the over-saturation.
(This is wrong because $\xi^{(1)}_a,a=1,\cdots,N^2$)

\begin{wrong}
	The discussion below need to be corrected (using a wrong basis)!!
\end{wrong}

$\Gamma^{a\pm}$ changes the $a$-spin $\mp\to\pm$.
For example, $\Gamma^{1+}$ will take $(+-++-)$ to $(++++-)$,
then the only non-vanshing product is with $(----+)$.
One then should write $\Gamma^{1+}$ coefficent $[B_1,\cdot]$
at the $(----+)$-row $(+-++-)$-column.
We have a term that is proportional (normalization?) to
\[
	\mathrm{Tr} c_{(----+)} [B_1,c_{(+-++-)}] = 
	c_{(----+)}^a \mathrm{Ad}(B_1)_{ab} c_{(+-++-)}^b 
.\] 
where the $\mathrm{Ad}(B_1)$ is a matrix defined as
\[
	\mathrm{Tr} T_a [B_1,T_b] \equiv \mathrm{Ad}(B_1)_{ab}
.\] 
Note that this is not an $N\times N$ matrix,
but $(N^2-1)\times (N^2-1)$.

The Grassmann number that we are going to integrate is $c_{(s)}^a$.
The quadratic term has the form 
$ c_{(s)}^a \mathrm{Ad}(?_{(s,s')})_{ab} c_{(s')}^b $. 
The integration would be easier 
if one could diagonalize the $(s,s')$ dependence, 
then the integration could be done for a fixed $s$.

\begin{table}[ht]
\centering
\begin{adjustbox}{width=1\textwidth}
	\small
\begin{tabular}{|c|c|c|c|c|c|c|c|c|c|c|c|c|c|c|c|c|}
\hline
      &       &       &       &       &       &       &       &       &       &       & 0-    & 1-    & 2-    & 3-    & 4-    & +++++ \\ \hline
	  &       &       &       &       &       &       &       & 0-    & 1-    & 2-    &       &       &       & 4+    & 3+    & +++-{}- \\ \hline
      &       &       &       &       &       & 0-    & 1-    &       &       & 3-    &       &       & 4+    &       & 2+    & ++-+- \\ \hline
      &       &       &       &       & 0-    &       & 2-    &       & 3-    &       &       & 4+    &       &       & 1+    & +-++- \\ \hline
      &       &       &       &       & 1-    & 2-    &       & 3-    &       &       & 4+    &       &       &       & 0+    & -+++- \\ \hline
	  &       &       & 0-    & 1-    &       &       &       &       &       & 4-    &       &       & 3+    & 2+    &       & ++-{}-+ \\ \hline
      &       & 0-    &       & 2-    &       &       &       &       & 4-    &       &       & 3+    &       & 1+    &       & +-+-+ \\ \hline
      &       & 1-    & 2-    &       &       &       &       & 4-    &       &       & 3+    &       &       & 0+    &       & -++-+ \\ \hline
	  & 0-    &       &       & 3-    &       &       & 4-    &       &       &       &       & 2+    & 1+    &       &       & +-{}-++ \\ \hline
      & 1-    &       & 3-    &       &       & 4-    &       &       &       &       & 2+    &       & 0+    &       &       & -+-++ \\ \hline
	  & 2-    & 3-    &       &       & 4-    &       &       &       &       &       & 1+    & 0+    &       &       &       & -{}-+++ \\ \hline
	0-    &       &       &       & 4+    &       &       & 3+    &       & 2+    & 1+    &       &       &       &       &       & +-{}-{}-{}- \\ \hline
	1-    &       &       & 4+    &       &       & 3+    &       & 2+    &       & 0+    &       &       &       &       &       & -+-{}-{}- \\ \hline
	2-    &       & 4+    &       &       & 3+    &       &       & 1+    & 0+    &       &       &       &       &       &       & -{}-+-{}- \\ \hline
3-    & 4+    &       &       &       & 2+    & 1+    & 0+    &       &       &       &       &       &       &       &       & -{}-{}-+- \\ \hline
4-    & 3+    & 2+    & 1+    & 0+    &       &       &       &       &       &       &       &       &       &       &       & -{}-{}-{}-+ \\ \hline
+++++ & +++-{}- & ++-+- & +-++- & -+++- & ++-{}-+ & +-+-+ & -++-+ & +-{}-++ & -+-++ & -{}-+++ & +-{}-{}-{}- & -+-{}-{}- & -{}-+-{}- & -{}-{}-+- & -{}-{}-{}-+ &       \\ \hline
\end{tabular}
\end{adjustbox}
\end{table}
This table shows how the spinor components mix in
$\overline{\varphi}\Gamma^\mu[A_\mu,\varphi]$;
the number $a\pm,a=0,\cdots,4$ denotes the which bosonic matrix $B_a,\overline{B}_a$ comes into play.

Label by number: $(+++++) - 1$ etc.
The slots for the adjoint matrix $\mathrm{Ad}(B)$:
\begin{align*}
	0-:&\quad (1,12),(2,9),(3,7),(4,6)\\
	0+:&\quad (5,16),(8,15),(10,14),(11,13)\\
	1-:&\quad (1,13),(2,10),(3,8),(5,6)\\
	1+:&\quad (4,16),(7,15),(9,14),(11,12)\\
	2-:&\quad (1,14),(2,11),(4,8),(5,7)\\
	2+:&\quad (3,16),(6,15),(9,13),(10,12)\\
	3-:&\quad (1,15),(3,11),(4,10),(5,9)\\
	3+:&\quad (2,16),(6,14),(7,13),(8,12)\\
	4-:&\quad (1,16),(6,11),(7,10),(8,9)\\
	4+:&\quad (2,15),(3,14),(4,13),(5,12)
\end{align*}
The integration over Grassmann number will give non-zero result
only if all the components are just saturated.
(each number appears once)

Consider first only using $(0-,0+)$.
Then consider only using $(0-,0+,1-,1+)$.
Is it possible to use $0-$ without using $0+$?
(seems not possible due to the reality)
The next case is to use $(0,1,2,3)$.

Three example cases after integrating out the Grassmann directions:
\begin{gather*}
	[\mathrm{Ad}(B_1)]^4	[\mathrm{Ad}(\overline{B}_1)]^4 \\
	[\mathrm{Ad}(B_1)]^2 [\mathrm{Ad}(\overline{B}_1)]^2
	[\mathrm{Ad}(B_2)]^2 [\mathrm{Ad}(\overline{B}_2)]^2\\
	\mathrm{Ad}(B_1)\mathrm{Ad}(\overline{B}_1)
	\mathrm{Ad}(B_2) \mathrm{Ad}(\overline{B}_2)
	\mathrm{Ad}(B_3) \mathrm{Ad}(\overline{B}_3)
	\mathrm{Ad}(B_4)\mathrm{Ad}(\overline{B}_4)
\end{gather*}
Something to work out: numeric coefficients; the order of product.
