%! Tex root: ../master.tex

To match the near horizon D$p$-brane metric with the standard AdS metric,
one could use the following coordinate transformation \pdfref{KST08}	
\[
	u^2 = \mathcal{R}^{-2} (D_p g_d^2 N)^{-1} U^{5-p}
.\] 
Then $u$ is identified with the energy scale of the boundary theory $u\sim E$.

Check the metric in this coordinate.
Start from the standard solution
\[
	\mathrm{d}s^2 = H_p(r)^{-\frac{1}{2}} \left( 
	-\mathrm{d}t^2 + \sum_{i=1}^p \mathrm{d}x_i^2\right)
	+ H_p(r)^{\frac{1}{2}} \left( \mathrm{d}r^2 + r^2 \mathrm{d}\Omega_{8-p}^2 \right) 
.\] 
with
\[
	H_p(r) = 1 + \frac{Q_p}{r^{7-p}},\quad Q_p = d_p N g_{\text{s}} \alpha'^{\frac{7-p}{2}}
.\] 
This is the solution in string frame.

\begin{question}
	Check the form in Einstein frame?
\end{question}

The first step is to change from $r$ to $U\equiv r / \alpha'$.
\[
	\mathrm{d}s^2 = H_p(U)^{- \frac{1}{2}} \left( -\mathrm{d}t^2 + \sum_{i=1}^p \mathrm{d}x_i^2\right) + \alpha'^2 H_p(U)^{\frac{1}{2}} \left( \mathrm{d}U^2 + U^2 \mathrm{d}\Omega_{8-p}^2 \right) 
.\] 
As given in \pdfref{KST08}, the field theory limit (near horizon limit? decoupling limit?) is
\[
	g_s \to 0,\quad \alpha' \to 0,\quad U = \text{fixed},\quad g_d^2 N = \text{fixed}
.\] 
The $d$-dimensional Yang-Mills coupling is related to the string coupling $g_{\text{s}}$ as
\[
	g_d^2 = g_s (2\pi)^{p-2} \alpha'^{(p-3) / 2}
.\] 
Note that it's not clear what's the limit of $g_{d}$ when $p<3$ (to $0$ or to $\infty$?).
Rem. Field theory works for any $N$, but one may expect that the geometry emerges when $N\to\infty$.
Then we are also taking the 't Hooft limit.
In this limit, it's proper to write $Q_p$ as
\[
	Q_p = (2\pi)^{2-p} d_p g_d^2 N \alpha'^{5-p}
.\] 
with $H_p(U) = 1 + Q_p / (\alpha' U)^{7-p}$, $H_p(U)$ has a $p$-independent $\alpha'$ scaling in this limit
\[
	H_p(U) \sim \alpha'^{-2} \frac{(2\pi)^{2-p} d_p g_d^2 N}{U^{7-p}}
.\] 
Then $\mathrm{d}s^2$ has a uniform $\alpha'$ scaling
\[
	\mathrm{d}s^2 \sim \alpha' \left[ \frac{U^{\frac{7-p}{2}}}{[(2\pi)^{2-p}d_p g_d^2 N]^{1 / 2}} \left( -\mathrm{d} t^2 + \sum_{i=1}^p \mathrm{d}x_i^2 \right) + \frac{[(2\pi)^{2-p}d_p g_d^2 N]^{1 / 2}}{U^{\frac{7-p}{2}}} \left( \mathrm{d}U^2 + U^2 \mathrm{d}\Omega_{8-p}^2 \right)   \right]
.\] 

\[
	\mathrm{d}s^2 \sim \alpha' \frac{[(2\pi)^{2-p}d_p g_d^2 N]^{1 / 2}}{U^{\frac{3-p}{2}}}
	\left[ \frac{U^{5-p}}{(2\pi)^{2-p} d_p g_d^2 N} 
		\left( - \mathrm{d}t^2 + \sum_{i=1}^p \mathrm{d}x_i^2 \right) 
	+ \frac{1}{U^2}\left( \mathrm{d}U^2 + U^2 \mathrm{d}\Omega_{8-p} \right)\right] 
.\] 

Except for $p=3$, this is not a AdS-type geometry.
But in \pdfref{KST08}, it's said that the geometry is conformal to AdS.
This is done by first Weyl rescaling the metric such that the $\mathrm{d}U^2$ term takes the form (the name ``dual frame'' in \pdfref{KST08})
\[
	\frac{\mathrm{d}U^2}{U^2}
.\] 
This Weyl rescaling is related to the non-constant dilaton background.
Then the AdS metric
\[
	u^2 \left( - \mathrm{d}t^2 + \sum_{i=1}^p \mathrm{d}x^2 \right)  
	+	\frac{\mathrm{d}u^2}{u^2}
.\] 
can be obtained by simply changing of coordinates $U\to u$.
\[
	u^2 = \frac{U^{5-p}}{(2\pi)^{2-p} d_p g_d^2 N}
.\] 
Note that $u$ has the same dimension as $U$, so can be regarded as energy as well.

The dual frame is \pdfref{KST08}:
it's related with the string frame by a Weyl transformation
\[
	\mathrm{d}s_{\text{dual}}^2 = (N \mathrm{e}^{\phi})^c \mathrm{d}s_{\text{st}}^2,\quad c = - \frac{2}{7-p}
.\] 
such that the supergravity action has the form
\[
	S = \frac{N^2}{(2\pi)^7 \alpha'^4}
	\int \mathrm{d}^{10} x \sqrt{-g}
	(N \mathrm{e}^\phi)^{2(p-3) / (7-p)}
	\left( R + 4 \frac{(p-1)(p-4)}{(7-p)^2} (\partial\phi)^2 - 
	\frac{1}{2(8-p)! N^2} F_{8-p}^2\right) 
.\] 
where $F_{8-p}$ is the Hodge dual of the R-R field strength sourced by the D$p$-brane.

Consider the metric and dilaton solution in the dual frame.
\[
	N \mathrm{e}^\phi \sim \left(\frac{D_p g_d^2 N}{U^{3-p}}\right)^{\frac{7-p}{4}}
\] 
\[
	N e^\phi \sim \left( \frac{D_p g_d^2 N}{u^{3-p}} \right)^{\frac{7-p}{2(5-p)}}
.\] 
\[
	\mathrm{d}s_{\text{dual}}^2 \sim \alpha' \left[\frac{U^{5-p}}{(2\pi)^{2-p} d_p g_d^2 N} 
		\left( - \mathrm{d}t^2 + \sum_{i=1}^p \mathrm{d}x_i^2 \right) 
	+ \frac{1}{U^2}\left( \mathrm{d}U^2 + U^2 \mathrm{d}\Omega_{8-p} \right)\right]
.\] 
\[
	\mathrm{d}s_{\text{dual}}^2 \sim \alpha' \left[ u^2
		\left( - \mathrm{d}t^2 + \sum_{i=1}^p \mathrm{d}x_i^2 \right) 
	+ \frac{1}{u^2}\left( \mathrm{d}u^2 + u^2 \mathrm{d}\Omega_{8-p} \right)\right]
.\] 

\begin{question}
	It's natural(?) to use $U$ in the string frame and $u$ in the dual frame?
\end{question}

The open string description of the D$p$-brane \pdfref{KST08}
\begin{equation}
	S_{\text{st}} = - \frac{1}{(2\pi)^{p-2}\alpha'^{(p-3) / 2}}
	\int \mathrm{d}^{p+1} x \sqrt{-g_{\text{st}}} \mathrm{e}^{-\phi}
	\frac{1}{4} \mathrm{Tr}(F_{ij} F_{kl}) g_{\text{st}}^{ik}
	g_{\text{st}}^{jl}
\end{equation}
The subscript is to emphasize it's written in the string frame.
Use
\[
	\sqrt{-g_{\text{dual}}} = (N \mathrm{e}^\phi)^{- \frac{p+1}{7-p}}
	\sqrt{-g_{\text{st}}}
.\] 
\[
	(g_{\text{dual}})^{ij} = (N \mathrm{e}^\phi)^{\frac{2}{7-p}} (g_{\text{st}})^{ij}
.\] 
The dual frame action reads
\[
	S_{\text{dual}} = - \frac{1}{(2\pi)^{p-2}\alpha'^{(p-3) / 2}}
	\int \mathrm{d}^{p+1}x (N \mathrm{e}^\phi)^{\frac{2(p-5)}{7-p}}
	(N u^{p-3}) \frac{1}{4} \mathrm{Tr} F^2
.\] 
All the metric dependences are written out.
If plugging in the $u$-dependence of $N \mathrm{e}^\phi$,
we find that the $u$-dependence cancels out,
left with
\[
	(N \mathrm{e}^\phi)^{\frac{2(p-5)}{7-p}} (N u^{p-3}) = \frac{1}{D_p g_d^2}
.\] 
This is a good feature of the dual frame?

\begin{question}
	Is the supergravity action Weyl invariant? (We are not expecting this...
	What's the idea behind the dual frame?)
\end{question}

The string frame supergravity action (\pdfref{KST08})
\begin{equation}
	S_{\text{st}} = \frac{1}{(2\pi)^7 \alpha'^4}
	\int \mathrm{d}^{10} x \sqrt{-g}
	\left[ \mathrm{e}^{-2\phi} (R + 4 (\partial\phi)^2 - \frac{1}{12}H_3^2)
	- \frac{1}{2(p+2)!}F_{p+2}^2\right] 
\end{equation}
The action written in the dual frame is \pdfref{KST08}
\begin{equation}
	S_{\text{dual}} = \frac{N^2}{(2\pi)^7 \alpha'^4}
	\int \mathrm{d}^10 x \sqrt{-g}
	(N \mathrm{e}^\phi)^\gamma
	\left( R + 4 \frac{(p-1)(p-4)}{(7-p)^2} (\partial\phi)^2
	- \frac{1}{2(8-p)! N^2}F_{8-p}^2\right) 
\end{equation}
with $\gamma = 2(p-3)/(7-p)$ and $F_{8-p}$ be the dual field strength of $F_{p+2}$, $F_{p+2}= *F_{8-p}$.

It's common to Weyl rescale the metric, for example, transformation between the Einstein's frame and the string frame.
Here the dual frame is something similar. \pdfref{DGT94}
Why people use the dual field strength?

An interesting action is discussed in \pdfref{KST08}.
It's similar to the bosonic part of SYM,
but there is an extra curvature term to make the action invariant under the Weyl transformation.
The action is written in the Euclidean metric
\begin{equation}
	S = -\int \mathrm{d}^d x \sqrt{g}
	\left( -\Phi \frac{1}{4} \mathrm{Tr}F_{ij}F^{ij}
	+ \frac{1}{2}\mathrm{Tr}\left( X (D^2 - \frac{(d-2)}{4(d-1)}R)X \right)  
+ \frac{1}{4\Phi}\mathrm{Tr}[X,X]^2\right) 
\end{equation}
$g$ to be understood as a background metric;
$\Phi$ to be understood as a background scalar field.
$\Phi$ is a generalization of the Yang-Mills coupling $\frac{1}{g^2}$.
$D_\mu = \nabla_\mu - i [A_\mu,\cdot]$ is the covariant derivative in the adjoint representation.
The combination $D^2 - \# R$ is defined such that it transforms under the Wely transformation homogeneously.
The Weyl transformation is
\[
	g \to \mathrm{e}^{2\sigma} g,\quad X\to \mathrm{e}^{(1-\frac{d}{2})\sigma}X,\quad \Phi \to \mathrm{e}^{-(d-4)\sigma}\Phi,\quad A_i\to A_i
.\] 
The conformal laplacian $P_1 \equiv D^2 - \# R$ transforms as
\[
	P_1 \to \mathrm{e}^{-(d / 2 + 1)} P_1 \mathrm{e}^{(d / 2 - 1)\sigma}
.\] 
Remember here the Weyl transformation of $R$ and $\nabla_\mu$.

It's interesting to have a term like $\mathrm{Tr}XRX$.
Note $A_i$ does not change under the Weyl transformation.
It's also interesting to note that the normalization $g^2 \mathrm{Tr}[X,X]^2$.
It's different from what people usually write $\frac{1}{g^2} \mathrm{Tr}[X,X]^2$.
As said in \pdfref{KST08}, the latter corresponds to the worldvolume D-brane theory in string frame (overall dilaton dependence).
However, because $\mathrm{Tr}F^2$ and $\mathrm{Tr}[X,X]^2$ behaves differently under the change of the worldvolume metric $g$,
it's not natural to construct a Weyl invariant action if they have the same $\Phi$ normalization.

The significance of this Weyl invariance?
