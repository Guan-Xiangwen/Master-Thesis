%! Tex root: ../master.tex

To get a feeling about what kind of terms could be generated through this RG procedure, let's calculate the decomposition of $\frac{1}{6}\mathrm{Tr} M^6$:
\begin{align*}
v^\dagger M^4 v + (v^\dagger v)(v^\dagger M^2 v) + \frac{1}{2} (v^\dagger M v) (v^\dagger M v) + \frac{1}{3}(v^\dagger v)^3 \\
+ \alpha^2 v^\dagger M^2 v + \frac{3}{2} \alpha^2 (v^\dagger v)^2 + \alpha^4 v^\dagger v.
\end{align*}
Looking at the first line (terms without $\alpha$), one could think about the possible contractions.
The first term is quadratic in $v$, while other terms should be treated by perturbation theory.
It's important to note that, for the second term, if contracting $(v^\dagger v)$ one gets
\[
	\frac{1}{N+1} \mathrm{Tr} \left( \frac{1}{\mathds{1}+\cdots} \right) 
.\] 
where $\cdots$ coming from the quadratic terms like $ g M^2$.
The leading order in $g$ would be
\[
	\frac{1}{N+1} \mathrm{Tr} \mathds{1} = \frac{N}{N+1}
.\] 
One would say that this term has the $N$-dimension $0$.
This is inconsistent with the dimension of $(v^\dagger v)$, which is the same as $\mathrm{Tr}M^2$.
The reason is that $\mathrm{Tr} \mathds{1}$ has dimension $1$.
The dimension of $\mathrm{Tr}$ should also be taken into account correctly.

To the zeroth order of $g$, contraction of the first line gives
\[
\frac{N}{(N+1)^2} \mathrm{Tr} M^2 + \frac{3}{2(N+1)^2} \mathrm{Tr} M^2 + \frac{1}{2(N+1)^2} \left(\mathrm{Tr}M\right)^2 + \frac{1}{3} \left( \frac{N}{N+1} \right)^3 
.\] 
The first term has the ``wrong'' dimension because of the $v^\dagger v$ contraction.
When $N$ is large, however, it gives the leading contribution to $\mathrm{Tr}M^2$.
The $N$-dimension essentially tells us how the terms behave when changing $N$.
This is an assumption we made for our theory.
