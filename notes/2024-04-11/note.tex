%! Tex root: ../master.tex

Trace formula:
define
\[
	(\Sigma^{(\mathcal{A})}_{ab})^c_d \equiv
	i (\delta^c_b \delta_{ad} - \delta^c_a \delta_{bd})
.\] 
This is the Lorentz generators acting on the vector representation of
$SO(9,1)$.
\[
	\mathrm{tr} (
	\underbrace{
		\Sigma^{(\mathcal{A})}\cdots \Sigma^{(\mathcal{A})})
	}_{\text{odd}}
	=0
.\] 
\[
	\mathrm{tr}(\Sigma^{(\mathcal{A})}_{ab}\Sigma^{(\mathcal{A})}_{cd})
	= - 2 (\delta_{bc} \delta_{ad} - \delta_{bd} \delta_{ac})
.\] 
\[
	\mathrm{tr}(\Sigma^{(\mathcal{A})}_{ab}
	\Sigma^{(\mathcal{A})}_{cd}
	\Sigma^{(\mathcal{A})}_{ef}
	\Sigma^{(\mathcal{A})}_{gh})
	=  \delta_{bc}\delta_{de}\delta_{fg}\delta_{ha}
	+ \text{pert. (ab),(cd),(ef),(gh)}
.\] 
define
\[
	\Sigma_{ab}^{(\psi)} \equiv \frac{i}{4} [\Gamma_a,\Gamma_b]
.\] 
This is the Lorentz generators acting on the spinor representation of
$SO(9,1)$.
\[
	\mathrm{tr} (
	\underbrace{
		\Sigma^{(\psi)}\cdots \Sigma^{(\psi)})
	}_{\text{odd}}
	=0
.\] 
\[
\mathrm{tr} ( 
\Sigma_{ab}^{(\psi)} \Sigma_{cd}^{(\psi)}) 
= -\frac{1}{16} \mathrm{tr}
([\Gamma_a,\Gamma_b][\Gamma_c,\Gamma_d])
= - \frac{1}{16}(\mathrm{tr}\mathds{1})
4(\delta_{bc} \delta_{ad} - \delta_{ac} \delta_{bd})
.\] 
\[
\mathrm{tr} ( 
\Sigma_{ab}^{(\psi)} \Sigma_{cd}^{(\psi)}) 
= - 4
(\delta_{bc} \delta_{ad} - \delta_{ac} \delta_{bd})
.\] 
\[
	(\Sigma_{ab}^{(\psi)} \Sigma_{cd}^{(\psi)}) 
= 2\mathrm{tr}(\Sigma^{(\mathcal{A})}_{ab}\Sigma^{(\mathcal{A})}_{cd})
.\] 
\[
	\mathrm{tr}(\Sigma^{(\psi)}_{ab}
	\Sigma^{(\psi)}_{cd}
	\Sigma^{(\psi)}_{ef}
	\Sigma^{(\psi)}_{gh})
	= 
	\delta_{bc}\delta_{de}\delta_{fg}\delta_{ha}
	+ (\text{pair constractions } \cdots)
\] 
The ``flux'' term in the one-loop (background) effective action
\[
	(\mathfrak{A}_\rho \mathfrak{A}^\rho - i \epsilon)^{-1}
	\Sigma_{\mu\nu}^{(\mathfrak{A},\psi)} \mathfrak{F}^{\mu\nu}
.\] 

\begin{correct}
	Consequence of Jacobi identity
	(adjoint representation)
	\begin{equation}
		\mathfrak{F}_{\mu\nu} = [\mathfrak{A}_{\mu},
		\mathfrak{A}_{\nu}]
	\end{equation}
	A map
	\begin{equation}
		\text{fund. repr. } \propto \mathds{1}
		\Longrightarrow \text{adj. repr. }=0
	\end{equation}
	\begin{equation}
		\text{fun. } = \text{ diag. }
		\Longrightarrow \text{adj. } = \text{ symplectic(?) }
	\end{equation}

	\paragraph{observations}
	Adjoint matrix is antisymmetric, therefore traceless.
	Also Hermitian (?)
	Diagonalizable with eigenvalues forming $\pm$ pairs.

	When
	\[
		[A_\rho,[A_\mu,A_\nu]] \propto \mathds{1} \neq 0
	.\] 
	which implies that
	\[
		[\mathfrak{A}_\rho,\mathfrak{F}_{\mu\nu}] = 0
	.\] 
	Example? 
	No example in finite dimension $[\cdot,\cdot]\propto \mathds{1}$.
	Necessary condition $[A_\mu,A_\nu]\neq i\lambda\mathds{1},\lambda\in \mathbb{R}$.

	\begin{idea}
		The condition is easier to impose on the adjoint matrix?
	\end{idea}
\end{correct}

$\Gamma$-matrices
\[
	(\Gamma^\mu)^{\alpha}_{~~\beta}:\quad
	\zeta^\alpha \to {\zeta'}^{\alpha}
	=(\Gamma^\mu)^\alpha_{~~\beta}\zeta^\beta,\quad
	\alpha,\beta=1,\cdots,32
.\] 
Complex conjugate
\[
	(\Gamma^{\mu*})^{\dot{\alpha}}_{~~\dot{\beta}}
	:\quad
	\zeta^{*\dot{\alpha}} \to {\zeta'}^{*\dot{\alpha}}
	= (\Gamma^{\mu*})^{\dot{\alpha}}_{~~\dot{\beta}}
	{\zeta*}^{\dot{\beta}},\quad
	\dot{\alpha},\dot{\beta}=1,\cdots,32
.\] 
$\alpha,\dot{\alpha}$ transform under different Lorentz generators.
The transpose
\[
		 (- \Gamma^{\mu\text{T}})_{\alpha}^{~~\beta}
	:\quad
	(\zeta^{\text{T}})_{\alpha} \to ({\zeta'}^{\text{T}})_{\alpha}
	=  (- \Gamma^{\mu\text{T}})_{\alpha}^{~~\beta}(\zeta^{\text{T}})_\beta
.\] 
The lower index $\alpha$ transform under the transpose $\Gamma$-matrices
$-\Gamma^{\text{T}}$.

There exists a $B$-matrix that relating $\Gamma$ and $\Gamma^*$:
$ B \Gamma B^{-1} = \Gamma^*$
\[
	B^{\dot{\alpha}}_{~~\alpha} (\Gamma^\mu)^\alpha_{~~\beta}
	({B^{-1}})^{\beta}_{~~\dot{\beta}} 
	= (\Gamma^{\mu*})^{\dot{\alpha}}_{~~\dot{\beta}}
.\] 
with the property
\[
	(B^*)^{\alpha}_{~~\dot{\alpha}}
	= (B^{-1})^{\alpha}_{~~\dot{\alpha}}
.\] 

There exits a $C$-matrix that relating $\Gamma$ and $-\Gamma^{\text{T}}$:
$ C \Gamma C^{-1} = - \Gamma^{\text{T}}$
\[
	C_{\alpha \gamma} (\Gamma^\mu)^{\gamma}_{~~\delta} (C^{-1})^{\delta\beta}
	= (-\Gamma^{\text{T}})_\alpha^{~~\beta}
.\] 

\begin{correct}
Consider basis transformation
\[
\zeta^\alpha \to \xi^\alpha = \mathrm{U}^\alpha_\beta \zeta^\beta
.\] 
$\Gamma$-matrices transform correspondingly
\[
	(\Gamma^\mu)^\alpha_{~~\beta} \to (\tilde{\Gamma}^\mu)^\alpha_{~~\beta}=
	\mathrm{U}^\alpha_\gamma (\Gamma^\mu)^\gamma_{~~\delta}
	(\mathrm{U}^{-1})^\delta_\beta
.\] 
$B$-matrix transforms correspondingly
\[
	B^{\dot{\alpha}}_{~~\alpha}
	\to \tilde{B}^{\dot{\alpha}}_{~~\alpha}
	= (\mathrm{U}^*)^{\dot{\alpha}}_{\dot{\beta}}B^{\dot{\beta}}_{~~\beta}
	(\mathrm{U}^{-1})^\beta_\alpha
.\] 
$C$-matrix transforms correspondingly
\[
	C_{\alpha \beta}\to\tilde{C}_{\alpha\beta}
	= (\mathrm{U}^{-1 \text{T}})_\alpha^\gamma
	C_{\gamma \delta} (\mathrm{U}^{-1})^\delta_\beta
.\] 

Some relations of $B$ and $C$ can be seen in a certain basis.
Then checking whether it still holds under a certain basis transformation
(unitary).
For example
\[
	B^{\text{T}} = B
\] 
holds under unitary basis transformation.

Properties of $B$ and $C$
\begin{equation}
	B^{\text{T}} = B ,\quad B^* = B^{-1},\quad C^{\text{T}} = -C
\end{equation}

``Wrong properties''
\[
	B^* = B,\quad C^* = C,\quad C^{\text{T}} = C^{-1}
\] 
can be true in a certain basis,
but not invariant under unitary basis transformations.
\end{correct}

\paragraph{conjugation}
It's about the dual space.
Define $\overline{\psi}:\text{spinor} \to \text{number}$.
Require Lorentz invariance and basis independence.
Consider Majorana spinor space.
The isomorphic between dual spaces
\[
	\psi \mapsto \overline{\psi} = \psi^\dagger \Gamma^0 = \psi^{\text{T}} C
.\] 
\[
	\psi^\alpha \mapsto \overline{\psi}_\alpha = 
	(\psi^*)^{\dot{\beta}} (\Gamma^0)_{\dot{\beta}\alpha}
	= \psi^\beta C_{\beta \alpha}
.\] 
I think the transpose for representation and the transpose for dual space
are different.
\begin{problem}
	$\overline{\psi}_\alpha$ and $\psi^\alpha$ 
	as independent Grassmann directions?
	(NO!)
\end{problem}
