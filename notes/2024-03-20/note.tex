%! Tex root: ../master.tex

It's of interesting to study how $\frac{1}{g^2}\mathrm{Tr}[A,A]^2$
contributes to the scaling of $g$ itself.
To get a quartic term in $A$, we need to look at the following combination
\[
	(\overline{\alpha}A^2\alpha)(\overline{\alpha}A^2\alpha),\quad
	(\overline{\alpha}A^2\alpha)(\overline{\alpha}a A\alpha)^2,\quad
	(\overline{\alpha}aA\alpha)^4
.\] 
Any number of
\[
	(\overline{\alpha}a^2\alpha),\quad (\overline{\alpha}\alpha \overline{\alpha}\alpha),\quad (\overline{\alpha}\alpha),(aa)
.\] 
can be attached.

The term
\[
	(\overline{\alpha}A^2\alpha)(\overline{\alpha}A^2\alpha)
	(\overline{\alpha}\alpha \overline{\alpha}\alpha)
.\] 
has two sorts of contraction.
One will give schematically $\mathrm{Tr}A^2 \mathrm{Tr}A^2$;
another is what we focus on $\mathrm{Tr}A^4$.
There are also two sorts of contraction that gives $\mathrm{Tr}A^4$.
The first case is that there is one contraction between the brackets
$(\overline{\alpha}A^2\alpha)$.
(There can not be two because then we get a disconnected contraction.)
The result of such a contraction is
\[
32 \left( \frac{v}{N_0} \right)^4(N+1)D
\left[ (3+D)\mathrm{Tr}(A^\mu A_\mu A^\nu A_\nu) 
- 4 \mathrm{Tr}(A^\mu A^\nu A_\mu A_\nu) \right] 
.\] 
Another possibility is that
all $\alpha,\overline{\alpha}$ in $(\overline{\alpha}A^2\alpha)$
are contracted with $(\overline{\alpha}\alpha \overline{\alpha}\alpha)$.

Before continuing the calculation
let's remark on the dependence of $v$.
Note first that $v$ always appear with $N_0$ in the combination $\frac{v}{N_0}$.
Because the true result should be independent of $v$,
this calculation could be a good approximation only after a careful choice of $v$.
In the literature, for example
\urlref{https://journals.aps.org/prd/abstract/10.1103/PhysRevD.35.1835}{Anna87}
people chooses the free parameter such that
the result is insensitive to the small change of the parameter.
Let's try to apply this strategy here.
What if there is a multiple insensitive points?
