%! Tex root: ../master.tex

Questions and Ideas in \pdfref{IMSY98}:

\begin{question}
	It's counter-intuitive that a large number of branes 
	corresponds to a small curvature solution of supergravity.
	Is there any physical interpretation?
\end{question}

The radial coordinate $r$ transversal to the Dp-branes
relates to the energy scale of the field theory on the Dp-brane.
From a field theory perspective, the transverse coordinates
should be interpreted as the vaccum expectation value of the scalar fields
that describing the transverse oscillation of the Dp-branes.

\begin{question}
About the scalar fields:
\begin{itemize}
	\item What's the form of the action for those scalar fields?
	\item What's the dimension of those scalar fields?
	\item What's the interpretation of the scalar fields?
		(Higgs? Goldstone?)
	\item The diagonal and off-diagonal components of the scalar fields
		(matrix valued in N-coincident Dp-branes)
		plays different roles?
\end{itemize}
\end{question}

The argument given in \pdfref{IMSY98} to relate $r$ with the energy scale:

\begin{info}
	Consider a D-brane sitting at the position $r$.
	This will lead to a gauge symmetry broken $U(N)\to U(N-1)\times U(1)$.
	Large $r$ will correspond to a large vacuum expectation value of the Higgs field (What's the Higgs field here?).
	The expectation value has the dimension of energy (clarify this)
	(recall that the energy dimension of a scalar field is $[\varphi]=(d-2)/2$)
	Therefore, large $r$ corresponds to a large energy scale.
\end{info}

To have a picture about the couplings that are involved in the discussion,
let's review the result of the tension of Dp-branes from string theory.
Let's start by looking at the Dp-brane action (Polchinski (8.7.2))
\begin{equation}
	S_p = - T_p \int d^{p+1} \xi e^{-\Phi}	
	\left[ - \mathrm{det} (G_{ab} + B_{ab} + 
	2\pi \alpha' F_{ab}) \right]^{\frac{1}{2}} 
\end{equation}
When there is a constant background dilaton field $\Phi = \Phi_0 + \tilde{\Phi}$,
the tension $T_p$ will be modified to $\tau_p = T_p e^{-\Phi_0}$.
One could calculate the interaction amplitude (one loop) in string theory,
the result is (Polchinski (13.3.1))
\begin{equation}
	\mathcal{A}_{\text{NS-NS}} \approx
	i V_{p+1} 2\pi (4\pi^2\alpha')^{3-p} G_{9-p}(y).
\end{equation}
The $G(y)$ is the scalar Green's function, $y$ is the separation between two Dp-branes.
The R-R amplitude cancels with the NS-NS amplitude $\mathcal{A}_{\text{R-R}}
=- \mathcal{A}_{\text{NS-NS}}$.
One the other hand, the same thing can be calculated by using the field theory.
(I don't know the detail here)
The result relates the slope $\alpha'$ with the Dp-brane tension $\tau_p$
and the supergravity coupling $\kappa$ (Polchinski (13.3.4)).
\begin{equation}
	\tau_p^2 = \frac{\pi}{\kappa^2} (4\pi^2 \alpha')^{3-p}	.
\end{equation}
The supergravity coupling $\kappa$ is a low energy parameter to describe the closed string interactions.
The closed string coupling $g_s$ is defined as a normalization of the closed string vertex operator (Polchinski (3.6.1)).
It relates to the gravitational coupling $\kappa$ in the following way
(Polchinski (12.3.47))
\[
	g_{s} = \frac{\kappa}{2\pi}
.\] 
In this way, the Dp-brane tension $\tau_p$ relates directly to the closed string coupling $g_s$.
Let's define the low energy (the lowest order of $\alpha'$) field theory
on the Dp-brane from the action $S_p$.
It's a Yang-Mills theory, the coupling is (Polchinski (13.3.25))
\[
	g_{D_p}^2 = \frac{1}{(2\pi\alpha')^2\tau_p}
.\] 
All of these allow us to relate the Yang-Mills coupling
$g_{D_p}^2$ (or $g_{YM}^2$) to the closed string coupling $g_s$.
The result is the (1) in \pdfref{IMSY98}
\begin{equation}
	g_{YM}^2 = (2\pi)^{p-2} g_s {\alpha'}^{(p-3) / 2}	
\end{equation}
The open string theory on the Dp-brane can be taken as a field theory
when $\alpha'\to 0$.
One could take the limit in such a way that the $g_{YM}$ keeps fixed.
This is called the ``field theory limit'' in \pdfref{IMSY98}.
\begin{question}
It seems like that we don't have a fundamental understanding of the closed string coupling $g_s$ in terms of the slope $\alpha'$.	
The lack of knowledge about $g_s$ is related to the unkown vacuum value of the dilaton field $\phi_0$?
\end{question}

\begin{question}
	What's the open string theory on the Dp-branes?
	Type I or Heterotic?
\end{question}

Let's take the example of a collection of D2-branes in \pdfref{IMSY98}.
The energy scale of the field theory is set by the following value
(This is general, not depending on the dimension of the branes)
\[
U = \frac{r}{\alpha'}
.\] 
In terms of the field theory, this is vacuum expectation value of the Higgs.

\begin{question}
Why this particular combination $U=r / \alpha'$ appears as
the vacuum expectation value of the Higgs?
Why this combination does not depend on the dimension of the field theory?
\end{question}

\begin{wrong}
One needs to make difference between the dimension of the string theory
and the dimension of the field theory.
For example, in string theory,
$\alpha'$ has the unit of $[\text{length}]^{2}$,
here the length should be understood as the space-time length.
Then it's obvious that $U$ has nothing to do with the energy of the space-time.
Also, let's look at the Yang-Mills coupling $g_{YM}^2$.
We know from the field theory that 
it's dimension is $[\text{energy}]^{3-p}$.
This has nothing to do with the space-time dimension reading from
$ g_{YM}^2 = (2\pi)^{p-2} g_s {\alpha'}^{(p-3) / 2}$.
\end{wrong}

\begin{todo}
It's desirable to have a clear discussion of various dimensions: from a space-time perspective and from the worldvolume field theory perspective.
\end{todo}

\begin{info}
It is said that the effective dimensionless coupling of the super-Yang-Mills
theory at the energy scale $U$ is
\[
	g_{eff}^2 \approx g^2_{YM} N U^{p-3}
.\] 
This is the coupling constant according to which we do the perturbative calculation in field theory.
The $N$ appears in the effective coupling because we are interested in such a large $N$ scaling: the coupling $g_{YM}$ changing with $N$ such that $g_{YM}^2 N$ keeps fixed.
\end{info}

It's necessary to clarify the different limits of string theory
that appearing in the AdS/CFT duality.
First, on both sides of the duality we have field theories,
which in general are the low energy limit of string theories
$\alpha'\to 0$.
However, remember that $[\alpha'] = [L]^2$,
one should always compare $\alpha'$ with some other scales.

The AdS side of the duality is essentially a theory
in which the string moving in a geometry background.
Such a theory is in general described by a non-linear sigma model
which generalizes the Polyakov action (Pol (3.76))
\begin{equation}
	S_\sigma = \frac{1}{4 \pi \alpha'}
	\int_M d^2\sigma g^{1 / 2}
	\left[ \left( g^{ab} G_{\mu\nu}(X) + i \epsilon^{ab} B_{\mu\nu}(X) \right)
	\partial_a X^\mu \partial_b X^\nu
	+ \alpha' R \Phi(X)\right] 
\end{equation}
This model could have a field theory description in the following limit:
Consider the target space has a characteristic radius of curvature
$R_c$.
Then the effective dimensionless coupling in this theory is $\alpha'^{\frac{1}{2}} R_c^{-1}$.
If it is small,
one can ignore the extended structure of string
instead using a field theory description of the string states.
This leads to the use of the ``low energy effective action''.
For example (Pol (3.7.20) for the bosonic string)
\begin{equation}
	\mathbf{S} = \frac{1}{2\kappa_0^2} \int d^D x (-G)^{1 / 2}	
	e^{-2\Phi} \left[ - \frac{2(D-26)}{3\alpha'} + R - \frac{1}{12} H_{(3)}^2 + 4 (\partial\Phi)^2 + O(\alpha')\right] 
\end{equation}

%AdS/CFT is interesting.
%On one hand we have a solution of a theory (the supergravity);
%on the other hand we are talking about a theory itself (the CFT).
%Well, it's not necessarily a CFT.
%It's just a quantum field theory whose action is given by the DBI.
%Now it's the case that we have two field theories.
%They are both in the low energy regime of string theory $\alpha'\to 0$.
%To be specific, the Yang-Mills theory is a good approximation of the Dp-brane dynamics in $\alpha'\to 0$ limit.
%In the AdS side,
%we should think about it as a theory of strings moving in a curved background.
%The action is described by a non-linear sigma model.
%(Polchinski 3.7)
%Why we don't use a dimensionless coupling in the ``field theory limit''?
