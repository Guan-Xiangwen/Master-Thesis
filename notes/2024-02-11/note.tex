The saddle point equation of resolvent method.
\begin{equation}
	\omega^2(z) + \frac{1}{g} V'(z) \omega(z)
	+ \frac{1}{4g^2} R(z) = 0
.\end{equation}
where $R(z)$ is an unknow polynomial of degree $l-2$ when $V$ is of degree $l$.
\begin{equation*}
	R(z) = 4g \int \mathrm{d}\lambda \rho(\lambda) \frac{V'(z)-V'(\lambda)}{z-\lambda}
.\end{equation*}
This equation is important because it tells us the singular part of $\omega_{\text{sing}}(z)$ has the form
\[
	2g\omega_{\text{sing}}(z) = \sqrt{(V'(z))^2 - R(z)}
.\]
Therefore, there are branch cuts connecting the zeros of the polynomial $(V'(z))^2-R(z)$.
However, it's impossible to determine the zeros directly because $R(z)$ is unknown.

\subsection{review the resolvent method}

In the resolvent method, instead of $\rho(\lambda)$, the central object under study is the resolvent $\omega(z),z\in \mathbb{C}\backslash\mathbb{R}$
\[
	\omega(z) = \int \frac{\rho(\lambda){\mathrm{d}\lambda}}{\lambda-z}
.\]
Then the basic equations of $\omega(z)$ follow from the saddle point equation.
One of them simply says that
\begin{equation}
	\omega(\lambda+i0) + \omega(\lambda-i0) = -\frac{1}{g} V'(\lambda) \quad \lambda\in\mathbb{R}
.\end{equation}
This combining with the analyticity of $\omega(\lambda)$ will give us an expression of $\omega(z),z\in\mathbb{C}$.
The simplest assumption is $\omega(z)$ has one branch cut at $[a_1,a_2]$,
which corresponding to $\rho(\lambda)$ supported on $[a_1,a_2]$.
It turns out that the form
\[
	\omega(z) = \frac{\sqrt{(z-a_1)(z-a_2)}}{2\pi g} \int_{a_1}^{a_2} \frac{\mathrm{d}\lambda}{\lambda - z}\frac{V'(\lambda)}{\sqrt{(a_2-\lambda)(\lambda-a_1)}}
.\]
satisfies all the requirement for the one cut solution.

$a_1,a_2$ can be determined by the asymptotic $\omega(z)\sim - 1/z$.
The integral form simplies for $|z|\to\infty$, the leading piece $O(1)$ should vanish
\[
	\frac{1}{2\pi g} \int_{a_1}^{a_2} \frac{V'(\lambda)\mathrm{d}\lambda}{\sqrt{(a_2-\lambda)(\lambda-a_1)}} = 0
.\]
The $O(1/z)$ piece should give the correct coefficient
\[
	\int_{a_1}^{a_2} \mathrm{d}\lambda \frac{\lambda V'(\lambda)}{\sqrt{(a_2-\lambda)(\lambda-a_1)}} = 2\pi g
.\]
The integral can be simplified by changing the variables
\[
	\lambda = z + \frac{1}{2} (a_1 + a_2) + \frac{(a_1-a_2)^2}{16 z}
.\]
Then the integral path can be written as a circle around origin for $z\in\mathbb{C}$
\begin{align}
	\oint \frac{\mathrm{d}z}{2\pi i z} V'(\lambda(z)) = 0 \\
	\oint \frac{\mathrm{d}z}{2\pi i} V'(\lambda(z)) = g
.\end{align}
There are some properties can be read directly from above equations.
If $V(\lambda)$ is even, $V'(\lambda)$ is odd.
Then the first equation is always satisfied by $a_1+a_2=0$.
This means that one can always find a solution whose branch cut is symmetric $[-a,a]$.
If $a_1 = a_2$,  $\lambda = z + \frac{1}{2}(a_1 + a_2)$,
the second equation implies that $g=0$.
Therefore it seems like not possible to obtain $g_*\neq 0$ for one cut solution.

\section{motivation letter}

I've been lucky with my educations, 
learning not just the knowledge and ideas in physics, 
but also how to enjoy learning from people who have genuine passion in physics.
Sometimes, I'd like to dream about being in Plato's Academy, 
thinking about our knowledge of nature and sharing those ideas with others.
I find that people today study physics in a similar way. 
That's my motivation for continuing my physics career.

What are my interests in general? 
Quite generally, I'm interested in using the tools and ideas from quantum field theory (QFT) to understand physical models. 
One of the intriguing aspects may be that how QFT offering people insights on the dynamics of a quantum system with numerous degrees of freedom. 
The spectrum of applications of QFT is also wide: particle physics, string theory, and even mathematics. 
Many deep physical ideas and mathematical structures can be formulated in the language of QFT. 
It's definitely one of the most interesting subjects to study.
 
Prof. Skenderis' research interests me, especially those about CFT.
The initial success of QFT based on its perturbative structure,
however many nowadays researches focus on the elusive non-perturbative structure.
One highlight is the AdS/CFT.
It not only provide a feasible way to understand quantum gravity,
but also allowing people to rethink the problems in QFT from the geometry point of view.
I want to learn more about QFTs with special symmetries,
they show interesting structures while allowing people to apply various methods to study them.
They may appear to be unrealistic,
but there may be some universalities that pertaining to the real world.
Skenderis's researches involve discussions on those toy model QFTs, 
and look at them from various sides.
The central role of CFT is obvious, which surprisingly related to a bunch of physics interests.
CFT manifests itself in terms of the energy-momentum tensor, 
the study of the energy-momentum tensor of a lattice field theory is interesting.
It's interesting to see how the stress tensor is formulated in the lattice theory, and how the lattice regularization modifies the property of the stress tensor.
Is it possible to have a lattice regularization of CFT?
I love to think about these questions.

I will also give a brief statement about my bachelor and master thesis topics. 
The bachelor thesis is about the linear perturbation theory of black hole solution in pure general relativity. 
I was fascinated by the interrelation between three properties of the theory: 
separability, D-type curvature and Killing spinor. 
The physics solution surprised me by its rich structure, 
which somewhat is hidden behind the complex metric.

In master thesis, I'm exploring possible scale invariant matrix models using the idea of renormalization group.
The physics motivation is a D-instanton system in which the partition function being expected to be conformal invariant. 
Without an intrinsic notion of spacetime, the scaling of couplings can only arise from the scaling of matrix rank: 
a typical example is the double scaling limit. 
With a similar spirit, we wonder whether the couplings in the D-instanton system are scale invariant. 
This point of view is not well explored, especially for a supersymmetric system. Rum trying to gain some insights on this problem.

\section*{REFERENCES}
\textbf{Prof. Toine Van Proeyen}\\
\textit{Emeritus Professor,  Institute for Theoretical Physics,  KU Leuven,  Celestijnenlaan 200D, B-3001 Leuven,  Belgium}\\
E-mail: \href{antoine.vanproeyen@kuleuven.be}{antoine.vanproeyen@kuleuven.be}\\
\textbf{Prof. Thomas Van Riet}\\
\textit{Associate Professor,  Institute for Theoretical Physics,  KU Leuven,  Celestijnenlaan 200D, B-3001 Leuven,  Belgium}\\
E-mail: \href{thomas.vanriet@kuleuven.be}{thomas.vanriet@kuleuven.be}

