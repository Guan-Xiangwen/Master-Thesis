\section{matrix integral from N to N-1}

The model under consideration is the one-matrix $\phi^4$ model
\begin{equation}
    \zeta_N(g) = \int \mathrm{d}\phi_N \mathrm{exp}(-S_N[\phi_N,g]),
\end{equation}
with an action
\begin{equation}
    S_N[\phi_N,g] = \mathrm{Tr} (\frac{1}{2}\phi_N^2 + \frac{g}{4}\phi_N^4).
\end{equation}

Start with the rank $N+1$ model, and decomposing the matrix $\phi_{N+1}$ as
\begin{equation}
    \phi_{N+1} = \begin{pmatrix}
\phi_N & v\\ 
 v^\dagger & \alpha
\end{pmatrix}.
\end{equation}
Then the action can be expanded as
\begin{align}
	S_{N+1}[\phi_{N+1},g] = &\mathrm{Tr}\left( \frac{1}{2}\phi_N^2+\frac{g}{4}\phi_N^4\right) + v^\dagger v + \frac{1}{2}\alpha^2 \notag \\
	&+ g\left( v^\dagger \phi_N^2 v + \alpha v^\dagger \phi_N v + \alpha^2v^\dagger v + \frac{1}{2}(v^\dagger v)^2 + \frac{1}{4}\alpha^4\right ).
\end{align}
Now the trace is over $N\times N$ matrix.

Recall that the matrix model is $U(N)$-invariant;
it may be useful to first gauge away some variables in $v,a$.
Let's consider how the gauge transformation acting on $\phi_N, v$ and $a$.
The infinitesimal transformation reads
\begin{align}
    \delta_t\phi_N &= i (vt^\dagger - tv^\dagger) \notag\\
	\delta_t v &= i(\phi_N - \alpha\mathds{1})t\notag\\
\delta_t v^\dagger &= it^\dagger (\alpha\mathds{1}-\phi_N) \notag\\
\delta_t \alpha &= i(v^\dagger t - t^\dagger v)
\end{align}
where $t$ ($N\times 1$ vector) is the components of the following generator
\[
 T = \begin{pmatrix}
	 0 & t \\
	 t^\dagger & 0
 \end{pmatrix}
.\] 
It's possible to choose $t$ to gauge away $v$ if $\phi_N - \alpha \mathds{1}$ is not degenerate.
So let's impose the gauge fixing condition as $v=0,v^\dagger=0$.

How to implement this condition in the integral?
Essentially what we should do is to change the integration variable from $v$ to $t$.
$t$ is understood as the parameter for the coset $U(N+1)/U(N)$ through exponent.
Around infinitesimal neighbour of $v=0$, $v(t)$ is a linear function given by the infinitesimal gauge transformation
\[
	v(t) = - i (\phi_N -\alpha \mathds{1})t
.\] 
Then the ``coordinate transformation'' will give a Jacobian that is proportional to $\mathrm{det} (\phi_N - \alpha \mathds{1})$.
The same factor will be obtained from $v^\dagger$.
Therefore we get
\[\left[\mathrm{det}(\phi_N - \alpha \mathds{1})\right]^2\]
in the integrand, which can be re-exponentiate to give a term in the action
\begin{equation}
    S_{N+1}[\phi_{N+1},g] = \mathrm{Tr}\left( \frac{1}{2}\phi_N^2+\frac{g}{4}\phi_N^4\right) + \frac{1}{2}\alpha^2 +  \frac{g}{4}\alpha^4 - 2 \mathrm{Tr} \mathrm{ln}(\phi_N - \alpha \mathds{1}).
\end{equation}
Note that this formula is equivalent to the eigenvalue representation if we diagonalize $\phi_N$.

However, the above calculation based on the linearized version of gauge transformation.
The full gauge transformation is more complicate.
We can proceed as following.
Consider a $(N+1)\times (N+1)$-matrix $\phi_{N+1}(\phi_N,0,a)$ with $v=0$.
Then do a gauge transformation, parameterized by $t$, on this matrix $\phi_{N+1}\to \tilde{\phi}_{N+1} = U(t) \phi_{N+1} U^\dagger(t)$.
The idea is that the integration over $\tilde{\phi}_{N+1}$ can be replaced with an integration over $\phi_{N+1}$ and $t$,
with a proper Jacobian that taking into account the functional dependence $\tilde{\phi}_{N+1}(\phi_{N+1},t)$.
In the exponent, because of the gauge invariance, one can replace $\tilde{\phi}$ directly with $\phi$.
To write down the Jacobian, one needs to solve for the $\tilde{v}(t)$, which is fixed by the condition $v=0$.
It turns out that the $t$-dependence of the Jacobian can be factorized out, so the $t$-integral can be performed independently.
This factorization property also implies that the linearized result giving the exact Jacobian for the $\phi$ factor,
although we don't know the $t$-factor.

In summary, we have
\begin{equation}
	\int \mathrm{d}\phi_N \mathrm{d}\alpha  [\mathrm{det}(\phi_N-\alpha\mathds{1})]^2 \mathrm{e}^{-\frac{1}{2}\alpha^2 - \frac{g}{4}\alpha^4-S_N[\phi_N,g]}
\end{equation}
or
\begin{equation}
	\int \mathrm{d} \phi_N \mathrm{d}\alpha \mathrm{e}^{-\frac{1}{2} \alpha^2 - \frac{g}{4}\alpha^4 - S_N[\phi_N,g] + 2 \mathrm{Tr}\ln (\phi_N - \alpha \mathds{1})} 
\end{equation}
The $\alpha$ integral is difficult to understand.
But I try to study it as follows.

The determinant is a characteristic polynomial $\mathrm{det}( \alpha\mathds{1} - \phi_N)\equiv p_N (\alpha)$ of $\phi_N$ in $\alpha$ with coefficients given by the following formula
\begin{equation}
    p_N (\alpha) = \sum_{k=0}^N \alpha^{N-k}(-1)^k \mathrm{Tr} \left(\wedge^k \phi_N \right),
\end{equation}
with
\[\mathrm{Tr} \left(\wedge^k \phi \right) = \frac{1}{k!}\mathrm{det}\begin{vmatrix}
\mathrm{Tr} \phi & k-1 & 0 & \cdots & 0\\ 
\mathrm{Tr} \phi^2 & \mathrm{Tr} \phi & k-2 & \cdots & 0 \\ 
\vdots & \vdots & \ddots  &  &\vdots \\ 
\mathrm{Tr} \phi^{k-1} & \mathrm{Tr} \phi^{k-2} &  & \ddots  & 1\\ 
\mathrm{Tr} \phi^k & \mathrm{Tr} \phi^{k-1} & \mathrm{Tr} \phi^{k-2} & \cdots & \mathrm{Tr} \phi
\end{vmatrix}\]
The leading term of $p_N(\alpha)$ is $\alpha^N$, and the last term is $(-1)^N \mathrm{det}\phi_N$. Also, let's write down first few terms to get a feeling
\[p_N(\alpha) = \alpha^N - \alpha^{N-1}\mathrm{Tr}\phi + \frac{1}{2}\alpha^{N-2}\left[(\mathrm{Tr} \phi)^2 - \mathrm{Tr} \phi^2\right] - \frac{1}{6}\alpha^{N-3}\left[(\mathrm{Tr}\phi)^3 - 3\mathrm{Tr}\phi \mathrm{Tr}\phi^2 + \mathrm{Tr}\phi^3\right] + \cdots\]
If we square it as in the integrand, we get
\begin{align}
	p_N^2(\alpha) =& \alpha^{2N} - 2 \alpha^{2N-1} \mathrm{Tr}\phi + \alpha^{2N-2}\left[2(\mathrm{Tr}\phi)^2 - \mathrm{Tr} \phi^2 \right] \notag\\
				   & -\alpha^{2N-3} \left[\frac{4}{3}(\mathrm{Tr}\phi)^3 -2 (\mathrm{Tr}\phi)(\mathrm{Tr}\phi^2) + \frac{1}{3}\mathrm{Tr}\phi^3\right] \cdots
\end{align}
One can only consider even terms if the potential is even.
For each term, $\alpha$ is decoupled from $\phi$, therefore can be integrated out.

To simpilfy the result, the following equation maybe useful
\begin{align}
	2\left\langle \mathrm{Tr}\phi^2 \right\rangle + \left\langle (\mathrm{Tr}\phi)^2 \right\rangle &= \left\langle \mathrm{Tr}(\phi^4 + g \phi^6) \right\rangle \notag\\
	2\left\langle \mathrm{Tr}\phi^3 \right\rangle + 2\left\langle (\mathrm{Tr}\phi)(\mathrm{Tr}\phi^2) \right\rangle &= \left\langle \mathrm{Tr}(\phi^5 + g \phi^7) \right\rangle
\end{align}
The $\langle \rangle$ indicates the matrix integral.
These equations follow from changing the integral variable $\phi \to \phi + \epsilon \phi^{3,4}$, with infinitesimal $\epsilon$.
The left hand side comes from the change of measure; while the right hand side follows from the change of the action.
For example, the $\alpha^{2N-2}$ term becomes
\[
	\alpha^{2N-2}\left[\mathrm{Tr}(-5\phi^2 + 2\phi^4 + 2g\phi^6)\right]
.\] 

As for the integration over $\alpha$
\[
	\int (\mathrm{d}\alpha) \alpha^{2N - k} \mathrm{e}^{- \frac{1}{2} \alpha^2 - \frac{g}{4} \alpha^4}
.\] 
The same trick allows us to derive that
\[
	(2N-k) \langle \alpha^{2N-k-1}\rangle = \langle \alpha^{2N-k + 1} \rangle + g \langle \alpha^{2N-k+3} \rangle
.\] 
For example $k=1$
\[
	\langle \alpha^{2N-2} \rangle = \frac{1}{2N-1} (\langle \alpha^{2N} \rangle + g \langle \alpha^{2N+2}\rangle)
.\] 
$k=3$
\[
	\langle \alpha^{2N-4} \rangle = \frac{1}{2N-3}(\langle \alpha^{2N-2} \rangle + g \langle \alpha^{2N} \rangle)
.\] 
For small $g$, it's reasonable that we only keep the $\alpha^{2N},\alpha^{2N-2}$ terms in large $N$ limit.
For a general $g$, it's not clear that whether or not $\alpha^{2N-2},\alpha^{2N-4}$ are in the same order or not.

Let's try to study the small $g$ limit
\begin{equation}
    r_k (g) \equiv \int \alpha^{2(N-k)}\mathrm{e}^{-\frac{1}{2}\alpha^2 - \frac{g}{4}\alpha^4}\mathrm{d}\alpha
\end{equation}
to the order of $g$ we have
\[r_k(g) \sim \sqrt{2\pi}(2N-2k-1)!! - \frac{g}{4}\sqrt{2\pi}(2N-2k+3)!! + O(g^2)\]
In the large $N$ limit
\[(2N-2k-1)!!\sim \sqrt{2}\left(\frac{2N-2k-1}{e}\right)^{N-k}\]
so we have
\[r_k(g) \sim 2\sqrt{\pi}\left(\frac{2N-2k-1}{e}\right)^{N-k} - \frac{g}{2}\sqrt{\pi}\left(\frac{2N-2k+3}{e}\right)^{N-k+1}\]
Let's factor out a $k$ independent factor $2\sqrt{\pi}[(2N+3)/e]^{N+1}$ such that
\[\tilde{r}_0(g) \sim \frac{e^3}{2N} - \frac{g}{4}\]
\[ \tilde{r}_1(g) \sim -\frac{e^2 g}{8}\frac{1}{N} + O(\frac{1}{N^2})\]
\[\tilde{r}_2 (g) \sim O(\frac{1}{N^2})\]
Then the asymptotic behavior of the integrand is
\[-\frac{g}{4} + \frac{e^3}{2N} -\frac{e^2g}{8N}\left[2(\mathrm{Tr}\phi)^2 - \mathrm{Tr}\phi^2\right] + O(\frac{1}{N^2})\]


