%! Tex root: ../master.tex

\begin{correct}
	Calculate the spinor product in the Majorana basis.
\end{correct}

The following are non-vanishing products for $\overline{\xi^{(1)}}$
\begin{gather*}
	\overline{\xi^{(1)}} \Gamma^{1-} \xi^{(8)},\quad
	\overline{\xi^{(1)}} \Gamma^{1-} \chi^{(8)},\quad
	\overline{\xi^{(1)}} \Gamma^{1+} \xi^{(8)},\quad
	\overline{\xi^{(1)}} \Gamma^{1+} \chi^{(8)} \\
	\overline{\xi^{(1)}} \Gamma^{2-} \xi^{(7)},\quad
	\overline{\xi^{(1)}} \Gamma^{2-} \chi^{(7)},\quad
	\overline{\xi^{(1)}} \Gamma^{2+} \xi^{(7)},\quad
	\overline{\xi^{(1)}} \Gamma^{2+} \chi^{(7)}\\
	\overline{\xi^{(1)}} \Gamma^{3-} \xi^{(6)},\quad
	\overline{\xi^{(1)}} \Gamma^{3-} \chi^{(6)},\quad
	\overline{\xi^{(1)}} \Gamma^{3+} \xi^{(6)},\quad
	\overline{\xi^{(1)}} \Gamma^{3+} \chi^{(6)}\\
	\overline{\xi^{(1)}} \Gamma^{4-} \xi^{(5)},\quad
	\overline{\xi^{(1)}} \Gamma^{4-} \chi^{(5)},\quad
	\overline{\xi^{(1)}} \Gamma^{4+} \xi^{(5)},\quad
	\overline{\xi^{(1)}} \Gamma^{4+} \chi^{(5)}
\end{gather*}
There is only one term involving $\Gamma^{0\pm}$,
$\overline{\xi^{(1)}} \Gamma^{0-} \xi^{(1)}$.
Note that above $\xi^{(k)},\chi^{(k)},k=5,6,7,8$ appear on an equal footing.
Also $\Gamma^{a\pm},a=1,2,3,4$ appears on an equal footing.

One could make the following table to clarify the product structure
in the Majorana basis.
\begin{table}[ht]
\begin{tabular}{|c|c|c|c|c|c|c|c|c|}
\hline
$\xi^{(1)},\chi^{(1)}$ & $\Gamma^{0-}$        &                        &                        &                        & $\Gamma^{4\pm}$        & $\Gamma^{3\pm}$        & $\Gamma^{2\pm}$        & $\Gamma^{1\pm}$        \\ \hline
$\xi^{(2)},\chi^{(2)}$ &                        & $\Gamma^{0-}$        &                        &                        & $\Gamma^{3\pm}$        & $\Gamma^{4\pm}$        & $\Gamma^{1\pm}$        & $\Gamma^{2\pm}$        \\ \hline
$\xi^{(3)},\chi^{(3)}$ &                        &                        & $\Gamma^{0-}$        &                        & $\Gamma^{2\pm}$        & $\Gamma^{1\pm}$        & $\Gamma^{4\pm}$        & $\Gamma^{3\pm}$        \\ \hline
$\xi^{(4)},\chi^{(4)}$ &                        &                        &                        & $\Gamma^{0-}$        & $\Gamma^{1\pm}$        & $\Gamma^{2\pm}$        & $\Gamma^{3\pm}$        & $\Gamma^{4\pm}$        \\ \hline
$\xi^{(5)},\chi^{(5)}$ & $\Gamma^{4\pm}$        & $\Gamma^{3\pm}$        & $\Gamma^{2\pm}$        & $\Gamma^{1\pm}$        & $\Gamma^{0+}$        &                        &                        &                        \\ \hline
$\xi^{(6)},\chi^{(6)}$ & $\Gamma^{3\pm}$        & $\Gamma^{4\pm}$        & $\Gamma^{1\pm}$        & $\Gamma^{2\pm}$        &                        & $\Gamma^{0+}$        &                        &                        \\ \hline
$\xi^{(7)},\chi^{(7)}$ & $\Gamma^{2\pm}$        & $\Gamma^{1\pm}$        & $\Gamma^{4\pm}$        & $\Gamma^{3\pm}$        &                        &                        & $\Gamma^{0+}$        &                        \\ \hline
$\xi^{(8)},\chi^{(8)}$ & $\Gamma^{1\pm}$        & $\Gamma^{2\pm}$        & $\Gamma^{3\pm}$        & $\Gamma^{4\pm}$        &                        &                        &                        & $\Gamma^{0+}$        \\ \hline
                       & $\xi^{(1)},\chi^{(1)}$ & $\xi^{(2)},\chi^{(2)}$ & $\xi^{(3)},\chi^{(3)}$ & $\xi^{(4)},\chi^{(4)}$ & $\xi^{(5)},\chi^{(5)}$ & $\xi^{(6)},\chi^{(6)}$ & $\xi^{(7)},\chi^{(7)}$ & $\xi^{(8)},\chi^{(8)}$ \\ \hline
\end{tabular}
\end{table}

Let's write 
\[
	\varphi = \sum_{k=1}^8 \xi^{(k)} +  \chi^{(k)}
.\] 
Then
\[
	\overline{\varphi} \Gamma^\mu [A_\mu,\varphi]
	=  \sum_{k=1}^8 \left( \overline{\xi^{(k)}} + \overline{\chi^{(k)}}  \right) 
	\sum_{a=0}^4 [\left( \Gamma^{a+} B_a + \Gamma^{a-} \overline{B}_a \right) ,
	\sum_{l=1}^8 \left(\xi^{(l)} +  \chi^{(l)}  \right) ]
.\] 

To avoid clustering of notation, let's denote
$ [B,\cdot] \equiv \mathfrak{B}$, which is understood as the adjoint matrix.

For those non-vanishing terms
\[
	\overline{(\xi,\chi)^{(k)}} \Gamma^{a \pm} \mathfrak{B}^\pm_a 
	(\xi,\chi)^{(l)}
	=
	{(\xi,\chi)^{(k)}}^{\text{T}} C \Gamma^{a \pm} \mathfrak{B}^\pm_a 
	(\xi,\chi)^{(l)}
,\] 
we want to check the action
\[
	C \Gamma^{a\pm} (\xi,\chi)^{(l)} = B \Gamma^0 \Gamma^{a\pm} (\xi,\chi)^{(l)} = B (\Gamma^{0+} - \Gamma^{0-})\Gamma^{a\pm} (\xi,\chi)^{(l)}
.\] 

Take one example (bad notation, $B_a$ use for bosonic matrices and $B$ use for Majorana conjugation)
\[
	C \Gamma^{4+} \mathfrak{B}_4 \xi^{(1)} 
	+ C \Gamma^{4-} \overline{\mathfrak{B}}_4 \xi^{(1)}
	=
	-B \mathfrak{B}_4 \Gamma^{0-}\Gamma^{4+}\xi^{(1)} 
	- B \Gamma^{4-} \overline{\mathfrak{B}}_4\Gamma^{0-}\Gamma^{4-} \xi^{(1)}
.\] 
\[
	-\Gamma^{0-}\Gamma^{4+}\xi^{(1)} = -\frac{1}{\sqrt{2}}
	\zeta^{(----+)} = \frac{1}{2} (\xi^{(5)} + i \chi^{(5)})
.\] 
\[
	-\Gamma^{0-}\Gamma^{4-}\xi^{(1)} = -\frac{1}{\sqrt{2}}
	\zeta^{(----+)} = -\frac{1}{2} (\xi^{(5)} - i \chi^{(5)})
.\] 
The Majorana basis $B (\xi,\chi) = \xi,\chi$, so we have
\[
	C \Gamma^{4+} \mathfrak{B}_4 \xi^{(1)} 
	+ C \Gamma^{4-} \overline{\mathfrak{B}}_4 \xi^{(1)}
	= \frac{1}{2} \left( \mathfrak{B}_4 - \overline{\mathfrak{B}}_4 \right)\xi^{(5)} 
	+ \frac{i}{2} (\mathfrak{B}_4 + \overline{\mathfrak{B}}_4)\chi^{(5)}
.\] 

Then we would like to understand the integration
\[
	\int \prod_{k=1}^8 [\mathrm{d}\xi^{(k)}] [\mathrm{d}\chi^{(k)}]
.\] 

To simplify the consideration a little bit,
let's focus on a particular combination
\[
\overline{\xi} \Gamma^+ \xi
.\] 
Then other terms follow just by replacing one or several $\xi,\Gamma^+$
by $\chi,\Gamma^-$.
However, be careful to the $\Gamma^0$ terms:
many combinations vanish, like $\overline{\xi}\Gamma^0 \chi$.


There are three different ways to saturate the measure
$\prod_{k=1}^8 \mathrm{d}\xi^{(k)}$:
the first kind is like
\begin{align*}
	(\overline{\xi^{(1)}} \mathfrak{B}_1 \xi^{(8)})
	(\overline{\xi^{(2)}} \mathfrak{B}_1 \xi^{(7)})
	(\overline{\xi^{(3)}} \mathfrak{B}_1 \xi^{(6)})
	(\overline{\xi^{(4)}} \mathfrak{B}_1 \xi^{(5)})
	+ (\text{all permutations like $(1) \leftrightarrow (8)$})
\end{align*}
These terms will cancel in pairs due to the interchange of Grassmann variables.
(the same cancellation happens for all terms not involving $\Gamma^0$?)
