%! Tex root: ../master.tex
\begin{todo}
Review the geometry of the Dp-branes:
focus on the relation between the supergravity solution
and the physical properties of the Dp-branes.

Keep in mind the following questions:
\begin{itemize}
	\item How the dilaton field profile depends on the dimension?
	\item The geometry of the solution: the curvature and the area of horizon.
	\item Comment on the validity regime of the solution.
\end{itemize}
\end{todo}

A particular type IIB supergravity solution is proved to be sourced by Dp-branes, which carry the R-R charge and preserve half of the supersymmetry.
Let's try to understand this, starting by looking at the supergravity theory.

The type IIB supergravity provides a field theory description of the massless states of the corresponding superstring theory.
The massless states are summarized in the following table: (TODO)

\begin{question}
	About the Type IIB supergravity:
	\begin{itemize}
		\item How to construct the action?
		\item How the couplings are related to the string couplings
			and the string length scale?
		\item How could it reproduce the string interaction?
		\item What's the supersymmetry algebra?
		\item How the fields transform under the supersymmetry?
		\item Clarify that ``the supergravity theory
			captures the low energy dynamics of the superstrings
			in the strong coupling regime?''
	\end{itemize}
\end{question}

\begin{correct}
The bosonic part of the type IIB supergravity \urlref{}{Joh03}
\begin{align}
	S_{\text{IIB}} = \frac{1}{2\kappa_0^2}
	\int d^{10} x (-G)^{1 / 2} \Big\{
		e^{-2\Phi} \left[ R + 4 \partial_\mu\Phi \partial^\mu\Phi
		-\frac{1}{12}(H^{(3)})^2\right] \notag \\
		- \frac{1}{12} (G^{(3)} + C^{(0)} H^{(3)})^2
		- \frac{1}{2} (d C^{(0)})^2
	-\frac{1}{480} (G^{(5)})^2 \Big\} \notag\\
	+ \frac{1}{4\kappa_0^2} \int
	\left( C^{(4)} + \frac{1}{2} B^{(2)} C^{(2)} \right) G^{(3)} H^{(3)}. 
\end{align}
$H^{(3)}=dB^{(2)}$ and $\Phi,G$ are in the NS-NS sector;
$G^{(3)} = dC^{(2)}$, $G^{(5)}=dC^{(4)}+H^{(3)}C^{(2)}$ and $C^{(0)}$
are in the R-R sector.
The normalization of the kinetic terms of the forms:
there is a prefactor of the inverse of $-2\times p!$ for a $p$-form field strength.
There is a self-dual condition that need to be imposed by hand
\[
	F^{(5)} = dC^{(4)},\quad F^{(5)}=*F^{(5)}
.\] 
to keep the correct number of degrees of freedom.
\end{correct}

One can split this action into three parts
\[
	S_{\text{IIB}} = S_{\text{NS}} + S_{\text{R}} + S_{\text{CS}}
.\] 
The last part is the Chern-Simon action.
(Check the gauge invariance.)
The $S_{\text{NS}}$ is written in the string frame;
while there is no dilaton coupling in $S_{\text{R}}$.
It's interesting to understand more about the modified strength
$G^{(5)} = dC^{(4)} + H^{(3)}C^{(2)}$.

\begin{todo}
Write down the equation of motion of this action.	
(Maybe it's easier to do it in the Einstein's frame?)
\end{todo}
