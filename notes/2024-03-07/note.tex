%! Tex root: ../master.tex

% understand the Dp-brane and its backreaction on the geometry:  <07-03-24, Xiangwen> %

The Dp-brane is dynamical: oscillating by itself 
(described by open strings ending on the Dp-brane)
and scattering with the closed strings.
Let's first review some physics of the Dirichlet brane
in the context of the worldsheet string theory.

\begin{problem}
Understand the D-brane from the T-duality. Refer to the Polchinski.
\end{problem}

A freely moving open string could be restricted to a hyperplane in the T-dual picture of the toroidal compactification.
The oscillation of the open string is then interpreted as the oscillation of the hyperplane in the T-dual picture.

\begin{problem}
An example of the T-duality in the toroidal compactification.	
\end{problem}

The target-space duality (T-duality) is an equivalence between
two string theories with different target spaces.
The simplest example is in the context of the toroidal compactification:
a certain direction in the target space is periodic
\[
X \simeq X + 2\pi R
.\] 

The closed string and the open string have different dynamics along the periodic direction.
An extreme case the $R\to 0$ limit:
the closed string behaves exactly the same as $R\to\infty$;
the open string has no motion along that direction.
One can see this by checking the mass spectrum
in terms of the momentum (and other numbers).

The bosonic closed strings with such a target space has the following mass spectrum
\begin{equation}
	m^2 = \frac{n^2}{R^2} + \frac{w^2 R^2}{\alpha'^2}
	+ \frac{2}{\alpha'} (N + \tilde{N} - 2)
\end{equation}
Here $n \in \mathbb{Z}$ labels the quantized momentum along the periodic direction; 
$w \in \mathbb{Z}$ is the closed string winding number.
$N,\tilde{N}$ are the levels the left-moving and right-moving string oscillators respectively.
The interesting point is that the spectrum is invariant under
\[
R \to R' = \frac{\alpha'}{R},\quad
n \to w
.\] 
In particular, the $R\to 0$ and $R\to\infty$ limits are physically identical.

The bosonic open strings do not have the winding number,
but its spectrum can be modified by a flat background for the gauge field
(the Wilson line).
The Wilson line will have a very interesting explanation in the T-dual picture.

Consider the open strings with $U(n)$ Chan-Paton factors.
The background gauge field along the periodic direction is assumed to be
\[
A = - \frac{1}{2 \pi R} \mathrm{diag} (\theta_1,\cdots,\theta_n)
.\] 
This is diagonal.
(It may be interesting to consider non-diagonal case.)
Then the open string spectrum has the form
\begin{equation}
m^2 = \frac{(2 \pi l - \theta_j + \theta_i)^2}{4 \pi^2 R^2}
+ \frac{1}{\alpha'} (N-1).
\end{equation}
Remember the general states of open strings with the Chan-Paton factor
transforming under the adjoint representation of the gauge group.
The gauge field couples to the Chan-Paton factor as $[A,\lambda]$.
This is the reason for the $(\theta_j-\theta_i)$ part of the spectrum.

Now one can see that
the open string spectrum has a different $R\to 0$ behavior.
There is no counterpart of the winding number $wR$
that leading to a continuum spectrum.
So the mass gap between $l=0$ and $l=1$ will tend to infinity
as $R\to 0$.
Roughly speaking, in the $R\to 0$ limit,
the open string moves in the $25$ spacetime dimension ($l=0$).
This requires that, in the dual picture $R'\to\infty$,
to keep the open string spectrum the same,
the end points must be fixed along this direction.


\begin{problem}[T-duality]
How the T-duality is realized in the worldsheet theory?	
\end{problem}

First separate the scalar field $X(z,\overline{z}) = X_L(z) + X_R(\overline{z})$.
Then define a new field $X'(z,\overline{z})$
\[
X'(z,\overline{z}) = X_L(z) - X_R(\overline{z})
.\] 
One claims that the worldsheet CFT using $X'$ and $X$
are T-dual to each other.
This further means that
the Neumann condition and the Dirichlet condition
are exchanged under the T-duality
\[
\partial_n X = - i \partial_t X'
.\] 
where $n$ is the normal and $t$ is the tangent at the boundary.

\begin{question}
This is not obvious: the T-duality is formulated for the periodicity of the target space;
How this can be formulated in an equivalent way in terms of the worldsheet field?
\end{question}

The string dilaton $\Phi$ will change under the T-duality.
This bases on the following argument:
the gravitational coupling $\kappa$ is related to the dilaton field
\[
	\kappa \propto \mathrm{e}^{\Phi}
.\] 
In the toroidal compactification,
the gravitational couplings in different dimensiona are related by
\[
\frac{1}{\kappa_d^2} = \frac{2 \pi \rho}{\kappa^2}
.\] 
with $2\pi\rho$ being the volume of the compactified dimension.
The point is that
the lower dimensional gravitational coupling $\kappa_d$
should be invariant under the T-duality.
It captures the physics that is un-related to the periodic dimension.
This implies that $\kappa$, and therefore $\Phi$, must change under the T-duality $\rho\to\rho'$.
\begin{equation}
	\rho' = \frac{\alpha'}{\rho},\quad
	\kappa' = \frac{\alpha'^{1 / 2}}{\rho}\kappa.
\end{equation}

\begin{todo}[D-brane in type II superstring]
Understand the physics of D-brane
and its implication in the superstring theory.  
\end{todo}

\begin{todo}[the superstring spectrum]
There are many different kinds of superstrings: try to summarize them.	
\end{todo}

The worldsheet fields: $X,\psi,\tilde{\psi}$;
The worldsheet theory of the superstring
starts from a $(N,\tilde{N})=(1,1)$ SCFT.
It may be useful to keep in mind the space-time interpretation.
The center-of-mass modes of the worldsheet current $(\partial X^\mu,\overline{\partial}X^\mu)$ is the space-time momenta $p^\mu$.
The center-of-mass modes of the worldsheet fermions $\psi_0^\mu,\tilde{\psi}_0^\mu$ is the gamma matrices $\Gamma^\mu$.
The supercurrents $T_F,T_B$ then has a similar form with
the Dirac operator and the Klein-Gordon operator.
(Polchinski 10.1)

\begin{question}
Why there are two copies $(\partial X,\overline{\partial}X)$
and $\psi,\tilde{\psi}$: what's the space-time interpretation?
\end{question}

The superconformal ghosts: $b,c,\beta,\gamma$.

The spectrum of the $X\psi$ SCFT: the R and the NS sector.
The vertex operators for the spectrum.

The superconformal ghosts: the spectrum and the vertex operators.

The physical states and the consistent superstring theories.
The superconformal constranits and the BRST formalism.
Type IIA, Type IIB and Type I $SO(32)$.
