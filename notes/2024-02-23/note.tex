%! Tex root: ../master.tex

\paragraph{A comparison between the ``dimensional model'' and the ``dimensionless model}

Let's start with a ``dimensionful'' model, whose partition function is given by
\begin{equation}\label{dim-quar-mod}
	Z_N (g) = \int \mathrm{exp}\left[- N \left(\frac{1}{2}\Tr M^2 + \frac{g}{4} \Tr M^4 \right)\right]\mathrm{d} M.
\end{equation}
The scaling of the matrix $M$, and the coupling $g$ under $N \to \lambda N$ is
\begin{align*}
	M \to \lambda^{-1 / 2}M \\
	g \to \lambda g
\end{align*}
One could also write the same theory in a ``dimensionless'' way,
by introducing $\tilde{M} = \sqrt{N} M$ and $\tilde{g} = N^{-1}g$.
Then in terms of $\tilde{M}$ and $\tilde{g}$ the partition function reads
\begin{equation}\label{quar-mod}
	Z_N (g) = N^{N^2 / 2} \int \mathrm{exp}\left[- \left(\frac{1}{2} \Tr \tilde{M}^2 
	+ \frac{\tilde{g}}{4} \Tr \tilde{M}^4\right)\right] \mathrm{d}\tilde{M}.
\end{equation}
Starting from the ``dimensionless model'' \eqref{quar-mod},
it's possible to recover the ``dimensionful model'' \eqref{dim-quar-mod} by specifying the quadratic term $\frac{N}{2}\Tr M^2$ first.

\paragraph{the notions of scaling}

First is the scaling from the RG flow.
Let's start by thinking about the RG calculation to the lowest order.
This will just reproduce the classical scaling: $g \to \lambda g$ and $\tilde{g}$ is invariant.
However, they are essentially different.
In the language of QFT, one can think of $N$ as a cut-off.
Then the RG method allows us to relate theories with different cut-off such that they will produce the same results (the correlation functions).
Specifically, to the lowest order, the coupling $g\to \lambda g$ with the change of cut-off $N\to \lambda N$.
To the lowest order, these are exactly the same as the classical scaling.
However, the higher order corrections are essential for the RG flow.

Conceptually, the RG flow keeps the theory invariant, 
while the classical scaling not:
one knows that the matrix model has a non-trivial $N$ dependence 
although they share the same form of action.
The classical scaling is just a natural way to define 
how the matrix and coupling depending on the underlying scale 
such that the form of the action keeping the same.
The classical scaling works like ``zooming in'' or ``zooming out''.
However, the RG flow works like ``coarse graining''.
There is no reason to believe that the ``coarse graining'' will give a similar result comparing to the ``zooming'' in general.
The Gaussian model is a special example that they giving exactly the same result.

Now what about the notion of ``scaling invariance''?
In this case, it's more interesting to consider yet another scaling.
Let's call it ``dynamical scaling''.
The meaning is that only ``dynamical variables'' should be rescaled.
The matrix is dynamical but the coupling constant is not.
Therefore, this scaling should not be understood as a change of dimension;
It's a symmetry of the action.
The interesting thing about the RG flow is that
the dynamical scaling invariance could emerge at certain critical point $g_*$.
The existence of such a point $g_* \neq 0$ seems impossible by just looking at the action, 
because it is written in a form that only the classical scaling is obvious.
It's impossible to obtain a dynamical scaling invariance along the classical scaling.
While the RG flow could deviate from the classical scaling significantly at some points, along which the scaling of $g$ could be frozen.

%\paragraph{To understand RG flow without the distraction from classical scaling, one should work with the dimensionless form.}
%If $N$ is taken to be a scale, it's not reasonable to apply the saddle point approximation of the action \eqref{dim-quar-mod}.
%Anyway, let's try to analyze the RG flow of \eqref{quar-mod} further.

Start with the $N+1$-model, and decomposing the matrix $M$ as
\begin{equation*}
\begin{pmatrix}
M & v\\ 
v^\dagger & \alpha
\end{pmatrix}.
\end{equation*}
The action can be expanded as
\begin{align}
S_{N+1}[M,v,v^\dagger,\alpha;g] 
= &(N + 1)\Tr\left( \frac{1}{2}M^2+\frac{g}{4}M^4\right) 
+ (N + 1) \left(v^\dagger v +\frac{1}{2}\alpha^2\right) \notag \\
&+ g(N+1)\left( v^\dagger M^2 v + \alpha v^\dagger M v 
+ \alpha^2 v^\dagger v + \frac{1}{2}(v^\dagger v)^2 
+\frac{1}{4}\alpha^4\right ).
\end{align}
%Making use of the unitary symmetry,
%it's possible to shift away the $v,v^\dagger$
%with a proper Jacobian corresponding to the choose of gauge. 
%The partition function then could be written as
%\begin{equation}
%Z_{N+1}(g) = \int  [\mathrm{Det}(M-\alpha\mathds{1})]^2 
%\mathrm{exp}\left[-\left(\frac{1}{2}\alpha^2 + \frac{g}{4}\alpha^4
%+ S_N[M;g]\right)\right]
%\mathrm{d}\alpha \mathrm{d}M. 
%\end{equation}
%The same form holds for a general action 
%\[
%S[M] = \frac{1}{2}\Tr M^2 + \sum_{k=3}^{\infty} \frac{g_k}{k} \Tr M^k
%.\] 
%
%The characteristic determinant $\mathrm{Det}(M - \alpha \mathds{1})$
%is responsible for the non-trivial behavior of the RG flow.
