%! Tex root: ../master.tex

\begin{todo}
Learn something about the D1/D5 CFT.	
\end{todo}

Such a D5/D1 system is described in
\urlref{https://arxiv.org/abs/hep-th/9711200}{Mal97}
as the following:
Consider IIB string theory compactified on $M^4$ 
(where $M^4 = T^4 \text{or} K3$)
to six spacetime dimensions.
Introduce a D5-brane
with four dimensions wrapping on $M^4$
giving a string in six dimensions.
Consider a system with $Q_5$ D5-branes and $Q_1$ D1-branes,
where the D1-branes are parallel to the string in six dimensions
arising from the D5-branes.
This system is described at low energies by
a 1+1 dimensional (4,4) superconformal field theory.

\begin{problem}
Why such a D-brane system has the (4,4) supersymmetry?
Why the low energy field theory is conformal?
\end{problem}

\begin{todo}
	Read \urlref{https://arxiv.org/abs/hep-th/9510135}{Wit95}
	for the CFT description of p-branes.
\end{todo}

The symmetry of this system is discussed as follows
\urlref{https://arxiv.org/abs/hep-th/9905111}{AGMOO99}.
The D1-branes along one non-compact direction;
The D5-branes extend in the same direction;
and they are coincident in that direction.
The unbroken Lorentz symmetry of this configuration is $SO(1,1)\times SO(4)$
$SO(1,1)$ corresponds to boosts along the string,
and $SO(4)$ is the group of rotations
in the four non-compact directions transverse to the string.
This configuration also preserves eight supersymmetries,
actually $\mathcal{N}=(4,4)$ supersymmetry.
We decompose the supercharges into left and right moving spinors of $SO(1,1)$.

What's the conformal field theory on this system?
\urlref{https://arxiv.org/abs/hep-th/9905111}{AGMOO99}.
The conformal field theory is the IR fixed point of the field theory
living on the D1-D5 branes.
The symmetry of this theory before going into the IR fixed point
is the $\mathcal{N} = (4,4)$ supersymmetry.
This amount of supersymmetry is equivalent to $\mathcal{N}=2$ in $d=4$.
There is a vector multiplet and a hypermultiplet.
In two dimensions both multiplets have the same propagating degrees of freedom,
four scalars and four fermions,
but they have different properties under the $SU(2)_{L} \times SU(2)_{R}$
global R-symmetry.
Under this group, the scalars in the hypermultiplets are in the trivial representation,
while the scalars in the vector multiplet are in the $(\mathbf{2},\mathbf{2})$.
On the fermions these global symmetries act chirally.
The left moving vector multiplet fermions are in the $(\mathbf{1},\mathbf{2})$
and the left moving hypermultiplet fermions are in the $(\mathbf{2},\mathbf{1})$.
The right moving fermions have similar properties with
$SU(2)_L \leftrightarrow SU(2)_R$.
The theory can have a Coulomb branch
where the scalars in the vector multiplets have expectation values,
and a Higgs branch
where the scalars in the hypermultiplets have expectation values.

\begin{question}
A better understanding of the above discussion.
\begin{itemize}
	\item the supermultiplet structure
	\item the global R-symmetry structure
	\item the IR fixed point
	\item the Coulomb branch and the Higgs branch
\end{itemize}
\end{question}

How to interpret this CFT?
\urlref{https://arxiv.org/abs/hep-th/9905111}{AGMOO99}.
The D1-branes can be viewed as instantons of the low energy SYM on the D5-brane.
These instantons live on $M^4$
and are translationally invariant along the time and the $x_5$ direction,
where $x_5$ is the non-compact direction along the D5-branes.
This instanton configuration has moduli,
which are the parameters that parameterize a continuous family of solutions.
All these solutions have the same energy.
Small fluctuation of this configuration (at low energy)
are described by fluctuations of the instanton moduli.
So the low energy dynamics is given by a 1+1 dimensional sigma model
whose target space is the instanton moduli space.

\begin{question}
	Some questions related to the above paragraph:
\begin{itemize}
	\item What is the instanton of the SYM on D5-brane?
	\item Why the D1-brane can be viewed as those instantons?
	\item What's the meaning of the fluctuation of the moduli?
	\item What's the meaning of a 2d sigma model with the moduli target space?
	\item Is this sigma model the 2d SCFT discussed above?
\end{itemize}	
\end{question}

\begin{todo}
Read the papers that study the D1-D5 system:
\urlref{https://arxiv.org/abs/hep-th/9807185}{Cos98},
\urlref{https://arxiv.org/abs/hep-th/9712213}{HW97-a},
\urlref{https://arxiv.org/abs/hep-th/9703163}{HW97-b}
\end{todo}

More on D5-brane bound state:
a bound state of two D5-branes
wrapped on $S^1\times T^4$
with coordinates $x^1,\cdots,x^5$.
\urlref{https://arxiv.org/abs/hep-th/9807185}{Cos98}
Each D5-brane has winding number $N_i$ along $S^1$,
$p_i$ along the $x^2$-direction
and $\overline{p}_i$ along the $x^4$-direction.
Thus, the worldvolume fields take values on the
$U(N_1 p_1 \overline{p}_1 + N_2 p_2 \overline{p}_2)$ Lie algebra.
To get a non-trivial D5-brane configuration
we turn on the worldvolume gauge field
such that the corresponding field strength is diagonal
and self-dual on $T^4$.
The non-vanishing components are taken to be
(assume $\mathrm{tan}\theta_1 > \mathrm{tan}\theta_2$)
\[
	G^0_{23} = G^0_{45} = \frac{1}{2 \pi \alpha'} 
	 \mathrm{diag} \left( \mathrm{tan}\theta_1,\cdots,\mathrm{tan}\theta_1,
	 \mathrm{tan}\theta_2,\cdots,\mathrm{tan}\theta_2\right)  
.\] 
where
\[
\frac{1}{2\pi \alpha'} \mathrm{tan} \theta_i
=\frac{2\pi}{L_2 L_3} \frac{q_i}{p_i} 
= \frac{2\pi}{L_4 L_5} \frac{\overline{q}_i}{p_i}
.\] 
with $q_i,\overline{q}_i$ are integers,
and $L_{\hat{\alpha}} = 2 \pi R_{\hat{\alpha}}$ the length of each $T^4$ circles.

\begin{question}
People are discussing the D5-brane configuration on a compact manifold,
with certain background gauge field.
\begin{itemize}
	\item Why the background gauge field is necessary for the discussion?
	\item What is the $G^0$ field? 
		Why the number of components are related to the D5-brane winding number?
	\item What's special about the value of $\mathrm{tan}\theta$?
\end{itemize}
\end{question}

The background gauge field breaks the gauge invariance
\[
U(N_1 p_1 \overline{p}_1 + N_2 p_2 \overline{p}_2) 
\to U(N_1 p_1 \overline{p}_1) \otimes U(N_2 p_2 \overline{p}_2)
.\] 
Because the branes are wrapped along the $x^1,x^2,x^4$ directions,
the gauge invariance is further broken to $U(1)^{\cdots}$.
