\section{IKKT}
IKKT model
\urlref{https://arxiv.org/abs/hep-th/9612115}{IKKT96}
can be obtained from dimensional reduction of
$d=10,\mathcal{N} = 2$ super Yang-Mills in the large-N limit.

In IKKT96,
the model is written as
\begin{align}
	Z &= \sum_{n=0}^{\infty} \int \mathrm{d} A \mathrm{d} \psi \mathrm{e}^{-S},\\
	S &= \alpha \left\{-\frac{1}{4}\mathrm{Tr}[A_\mu,A_\nu]^2 - \frac{1}{2}\mathrm{Tr}(\bar{\psi}\Gamma^\mu[A_\mu,\psi])\right\} + \beta \mathrm{Tr}1
.\end{align}
They assume this model has a well behaved contitnuum limit
such that the worldsheet action of superstring can be recovered.
Then how can we study the $Z$?
The sum diverges? Each term increases or decreases with $n$?
Could we plot the $n$-dependence qualitatively?
Could we compare the neighbour term?
What's the necessary condition for the existence of the continuum limit?

The commutator $[A_\mu,A_\nu]^2$ may be easier to work with in the canonical basis of Hermitian matrices.
Inside the Cartan subalgebra, the commutator is equal to zero.
As for other commutators, one has the root system in mind:
the commutators of matrices translate into the geometry of root vectors.
The real dimension is $\mathrm{dim}\mathfrak{u}(n) = n^2 $ with the Cartan subalgebra $\mathrm{dim}\mathfrak{h}(n)=n$.

