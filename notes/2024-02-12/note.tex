\section{an RG study of integral}
Integration could be studied using an RG spirit method,
which is similar to the Wilson's RG scheme.
However, there is no natural notion of energy associated with the integral.

Let's take the ``$O(N)$ vector model'' as the example
\urlref{https://arxiv.org/abs/1410.1635}{Zinn-Justin14}
\begin{equation}
	\mathrm{e}^{Z_N} = \left(\frac{N}{2\pi}\right)^{N / 2} \int \mathrm{d}^N \mathbf{x} \mathrm{e}^{-N V(\mathbf{x}^2)}
.\end{equation}
The normalization is chosen to normalize a Gaussian integral.
The factor $N$ before $V(\mathbf{x}^2)$ gives a ``dimension'' of $V$ under the scaling of $N$.
To implement the changing of $N$, one calculates $Z_{N+1}$ by integrating out one component of the vector $\mathbf{x}$
\begin{equation}
	\mathrm{e}^{Z_{N+1}} = \left(\frac{N+1}{2\pi}\right)^{(N+1)/2} \int \mathrm{d}^N \mathbf{x} \int \mathrm{d} y \mathrm{e}^{-(N+1) V(\mathbf{x}^2 + y^2)}
\end{equation}
It's impossible to integrate out $y$ exactly in general,
so one may use the saddle point approximation.
For a generic $\mathbf{x}$, the integrand $\mathrm{e}^{-(N+1)V(\mathbf{x}^2 + y^2)}$ is localized at the critical value of $y$ when $N\to\infty$.
To the leading order, assuming that the saddle point is at $y=0$,the approximation gives \[
	\mathrm{e}^{Z_{N+1}} = \left(\frac{N+1}{2\pi}\right)^{N / 2} \int \mathrm{d}^N \mathbf{x} \mathrm{e}^{-(N+1)V(\mathbf{x}) - \frac{1}{2} \ln 2 V'(\mathbf{x}^2)} \left(1 + O(\frac{1}{N})\right)
.\] 
It should be noted that the saddle point approximation fail when $V'(\mathbf{x}^2) = 0$.
The Gaussian potential $V(\mathbf{x}^2) = (1 /2) \mathbf{x}^2$ will give $ V'(\mathbf{x}^2) = 1 /2$.
In this case, the leading order vanishes.

I'm not clear to what extent the subleading corrections will modify the scaling around the critical point;
however, I guess they are not very important if the critical point by itself requires $N\to\infty$
(similar to the fact that phase transition requires thermodynamic limit).
Due to various methods to solve $N\to\infty$ model, it deserves study;
however, is it possible to obtain a matrix model that has scaling covariance for all $N$?
This is a stringent requirement,
but there is example of matrix model that is independent of $N$.
For example, see the discussion of the Kontsevich integral in
\urlref{https://arxiv.org/abs/hep-th/9303139}{Morozov94}
below the equation (3.70).

Come back to the discussion of the vector model.
The change $N\to N+1 $ induces a change in $\mathbf{x}$ and $V$
\begin{align}
	\mathbf{x} \to \mathbf{x} \left(1- \frac{1}{2N}(1+\gamma)\right),\\
	V \to V + \delta V
\end{align}
such that the partition function scales as
\begin{equation}
	Z_{N+1}(V) = -\frac{1}{2} \gamma + Z_N(V+\delta V) + O(\frac{1}{N})
\end{equation}
The $-\frac{1}{2}\gamma$ term comes form the scaling of the integration measure $\mathrm{d}^N \mathbf{x}$.
The $V+\delta V$ comes from the integration over $y$, also the scaling of $\mathbf{x}$
\[
N \delta V (\rho) = V(\rho) - (1+\gamma) \rho V'(\rho) + \frac{1}{2} \ln 2 V'(\rho) + O(\frac{1}{N})
.\] 
the $\rho$ is a shorthand notation for $\mathbf{x}^2$.
The differential equation for the deformation of $V$ along the RG flow reads then
\begin{equation}
	\lambda \frac{\mathrm{d}}{\mathrm{d} \lambda} V(\rho,\lambda) = V(\rho,\lambda) - \left(1+\gamma(\lambda)\right) \rho V'(\rho,\lambda) + \frac{1}{2} \ln 2V'(\rho,\lambda)
\end{equation}
where $\lambda$ is a continuous parameter $N\to \lambda N$.
Note that $\gamma$ is assumed to depend on $N$.


$V(\rho) = \frac{1}{2} \rho$ is a fix point with $\gamma=0$, the Gaussian fixed point.
It's not obvious in general how to solve the fix point equation due to the non-linear term $\ln V'$.
One possible solution is
\[
V(\rho) = \frac{1}{2} \ln \rho
.\] 
with $\gamma=-1$ such that the second term vanish.
If plugging this back to the integral, one finds the integrand is $\rho^{- N / 2}$.
This cancels with the radius measure in spherical coordinate.
Therefore, this potential corresponds to a trivial vector model.

In general, it turns out that it's easier to solve the equation for $V'$.
Actually, if one define a function $R(\rho)$
\[
R(\rho) \equiv \frac{1}{2 V'(rho)}
.\] 
Then the fix point equation will simply be
\begin{equation}
	\gamma R + R R' - (1 + \gamma) \rho R' = 0
\end{equation}
The $\gamma = -1$ case gives $R'=1$, which corresponding to the solution $V(\rho)=\frac{1}{2}\ln\rho$.
Then how to obtain the solution of, for example $\gamma=-2$?

Note that the finite scaling of $\mathbf{x}$ is
\[
\mathbf{x} \to \mathbf{x} \left(1 + \frac{1 + \gamma}{2} \ln \lambda\right)
.\] 
Instead of power of $\lambda$, it scales as $\ln \lambda$.

