%! Tex root: ../master.tex
\paragraph{the interaction between A and B}

What kind of features that we want our model have?
After integrating out one row and one column, there is no new interaction being generated.
To understand under what situation this property will be satisfied,
let's think about it order by order.
Consider the following $2 \times 2$ matrix
\[
	A = \begin{pmatrix} a_1 & \alpha \\ \alpha^* & a_2 \end{pmatrix},\quad B = \begin{pmatrix} b_1 & \beta \\ \beta^* & b_2 \end{pmatrix}
.\] 
The interaction terms that we are interested in are
\[
	g \mathrm{Tr}(ABAB) \quad h \mathrm{Tr}(AABB)
.\]
Looking at them, one expects that $a_1, (a_1)^2, (a_1)^3, (a_1)^4, \cdots$ (similar for $b_1$) can be generated by integrating out $\alpha,\alpha^*,a_2$.
Now let's figure out which terms in the perturbative expansion will contribute to these ``new interactions''.

It maybe useful to keep track of the terms by drawing diagrams.
For a matrix, let's draw a line with an arrow pointing from one end point to another.
The end points indicate the row and column indices respectively.
Matrix product is represented by joining two end points together,
which means that they always take the same value.
Therefore, the trace is represented by a loop.
One also need to differentiate matrix $A$ and $B$ by using different kinds of lines.

What will be integrated out is the line with end points taking value in a certain row and column.
In the case of $2 \times 2$ matrix, the value may be $2$.
The lines that being integrated out will be connected by a ``propagator'' diagramatically.
For the off-diagonal elements $\alpha,\alpha^*$, the ``propagator'' is represented by double-line diagram;
for the diagonal element $a_1$, it will be represented by a single line because the row and column indices taking the same value therefore can be regarded as a single point.

The first thing to note is that the odd terms $a_1,(a_1)^3,\cdots$ cannot be generated.
We always have even number of $A$ matrix, the terms that could contribute to the odd terms cannot be fully contracted.
The next thing to consider is the $(a_1)^2$ term.
To the first order of $g,h$, one gets from $g\mathrm{Tr}(ABAB)$ a term $g (a_1)^2 (b_1)^2$ and from $h\mathrm{Tr}(AABB)$ a term $h(a_1)^2 \beta^* \beta$.
The term $h (a_1)^2 \beta^* \beta$ will contribute to the modification of the propagator.
Then let's consider the $(a_1)^4$ term.
It will only appear to the second order of $g,h$.
We have the following terms
\[
	\frac{g^2}{2} (a_1 b_1 a_1 b_1)(a_1 b_1 a_1 b_1),\quad \frac{h^2}{2} (a_1 a_1 \beta \beta^*)(a_1 a_1 \beta \beta^*),\quad gh(a_1 a_1 \beta \beta^*)(a_1 b_1 a_1 b_1) 
.\] 
How to understand these terms?
Do they imply that new interactions will appear along the RG flow?
No, for example, the term $(g^2/2) (a_1)^4 (b_1)^4$ could be reproduced by $g \mathrm{Tr}(ABAB)$ interaction by expanding the exponent.
Similarly the term $ gh (a_1)^4 (b_1)^2 (\beta^* \beta)$ could be reproduced by $ g \mathrm{Tr}(ABAB)$ and a proper modification of the quadratic term $ \mathrm{Tr}A^2$.
However, the second term $ (h^2 / 2) (a_1 a_1 \beta \beta^*) (a_1 a_1 \beta \beta^*) $ will generate a new quartic term $(h^2 / 2)\mathrm{Tr}A^4$.

Is it possible to cancel $ ( h^2 / 2) \mathrm{Tr} A^4$ by adding new interactions to the model?
One natural idea is to replace the $B^2$ in $ \mathrm{Tr}(AABB) $ by a Grassmann valued matrix such that the integration will get an extra minus sign.
Schematically we may add a term $h\mathrm{Tr}(AA\Psi \overline{\Psi})$ with its conjugation. 
Then we can look at the term $ (h^2 / 2) (a_1 a_1 \psi \overline{\psi}) (a_1 a_1 \psi \overline{\psi})$.
The contraction of $\psi,\overline{\psi}$ will give $ - (h^2 /2) a_1^4 $.
Whether this cancellation could continue to higher order.
Whether there is a symmetry reason behind this cancellation?

What if we take $h=0$, that is only considering the interaction $ g \mathrm{Tr}(ABAB)$?
Still the $\mathrm{Tr}A^4$ could be generated, but at higher order $g^4$.
For example, consider the contraction of the following term
\[
	\frac{g^4}{4!}(a_1 \beta a_2 \beta^*)(a_1 \beta a_2 \beta^*)(a_1 \beta a_2 \beta^*)(a_1 \beta a_2 \beta^*)
.\] 
Also, it will also generate the ``double trace'' term $ \mathrm{Tr}A^2 \mathrm{Tr}A^2$, which may not be able to be absored into the quadratic term $\mathrm{Tr}A^2$.

\paragraph{Discussion of symmetries}

What's the reason behind the generation of these new terms?
Comparing $\mathrm{Tr}(AABB)$ and $\mathrm{Tr}(A^4) + \mathrm{Tr}(B^4)$, one finds that the first term is invariant under the rescaling $A\to\lambda A, B\to \lambda^{-1}B$, while the second term not.
The reason behind the broken of rescaling invariance is the presence of the quadratic term $\mathrm{Tr}(A^2),\mathrm{Tr}(B^2)$.
What if there is no quadratic term?
Then it's hard to do the perturbative calculation.
But in some interesting cases, the quadratic term may have no influence on the model (they can always be shifted away).
How this can be the case?

Another thing that may be worth to note is that the interaction $\mathrm{Tr}([A,B]^2),\mathrm{Tr}([A,B]^4),\cdots$ is invariant under ``translation'' $A\to A + \lambda \mathds{1},B\to B + \mu \mathds{1}$.
Reversly, the commutator arises naturally if requiring the ``translation invariance''.
However, the quadratic term also breaks this invariance.
If one wants to explore the consequence of these symmetries,
it's important to get around the quadratic term.

Let's try to introduce two fermionic matrices $\psi,\chi$.
It's convenient to define
\begin{equation}
	Z = A + i B,\quad \Psi = \psi + i\chi	
\end{equation}
Also denote $ \overline{Z} = A - iB,\overline{\Psi} = \psi - i\chi$.
Now I will write down an action that having a BRST-like symmetry.
To do this, one needs to introduce two auxillary matrices $H$ (bosonic) and $\eta$ (fermionic).
The BRST-like symmetry for $H,\eta$ is
\[
\delta \eta = H,\quad \delta H = 0
.\] 
For $Z,\Psi$ we will define
\[
	\delta Z = \Psi,\quad \delta\Psi = [H,Z] + i\epsilon Z
.\] 
where $\epsilon$ is an arbitrary real parameter.
The idea behind this definition is that the square $\delta^2$ will give a usual symmetry of matrix models
\[
	\delta^2 Z = [H,Z] + i\epsilon Z,\quad \delta^2\Psi = [H,\Psi]+ i\epsilon \Psi
.\] 
The commutator $[H,\cdot]$ is the generator of the unitary transformation.
The $i\epsilon$ term generate the rotation symmetry
\[
	\delta^2 A = -\epsilon B,\quad \delta^2 B = \epsilon A
.\] 

To construct an action with desired symmetries: unitary, translation, rotation and BRST-like, we write
\begin{equation}
	S[H,\eta,Z,\Psi] = - \frac{1}{2} \mathrm{Tr} H^2 + g\mathrm{Tr}\left[\delta (i [\overline{Z},Z]\eta)\right]
\end{equation}
The second term is $\delta$-exact, so $\delta S = 0$ by definition.
Expanding the action of $\delta$, one gets
\begin{equation}
	- \frac{1}{2} \mathrm{Tr} H^2 + i g \mathrm{Tr} ([\overline{Z},Z]H) + i g \mathrm{Tr} ([\overline{\Psi},Z]\eta + [\overline{Z},\Psi]\eta)
\end{equation}
The rotation $Z\to \mathrm{e}^{i\theta}Z,\Psi \to \mathrm{e}^{i\theta}\Psi$ and translation $Z\to Z + \lambda \mathds{1}$ symmetries are satisfied.

One can also consider adding the following $\delta$-exact term to the action
\[
	i\delta \left(\mathrm{Tr}(\overline{Z}\Psi - Z \overline{\Psi})\right) = 2 i \mathrm{Tr}(\overline{\Psi}\Psi) -2\epsilon \mathrm{Tr}(\overline{Z}Z) - 2i \mathrm{Tr}([\overline{Z},Z]H) 
.\] 
This term breaks the translation symmetry.
As a consequence, the quadratic terms are generated.

