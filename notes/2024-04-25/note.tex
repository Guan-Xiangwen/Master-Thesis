%! Tex root: ../master.tex

\paragraph{contraction game: $\mathcal{A}^8$-term from spinor part}
This means that we are looking at the following term
\[
	\frac{\#}{g^{16}} (\varphi \Sigma \varphi)^{16}
.\] 
Any term after contraction contains a trace of $16$-$\Gamma$ matrices:
\[
\mathrm{Tr}(\underbrace{\Gamma \overline{\Gamma} \cdots 
\Gamma \overline{\Gamma}}_{16})
.\]

\paragraph{an observation}
Maybe starting with a case that we have already done,
and try to develop another method that avoiding the combinatoric complexity.
That is $8$-$\Gamma$-matrices:
\[
\mathrm{Tr}(\Gamma^1 \Gamma^2 \Gamma^3 \Gamma^4 
\Gamma^5 \Gamma^6 \Gamma^7 \Gamma^8)
.\] 
The first thing is to separate it into two groups: odd and even.
We restrict to the case that contraction happens only among the each group.
The result of the contraction is given by the product of $\eta^{\mu\nu}$.
\begin{align*}
	(\eta^{13} \eta^{57} - \eta^{15} \eta^{37} + \eta^{17} \eta^{35})
	(\eta^{24} \eta^{68} - \eta^{26} \eta^{48} + \eta^{28} \eta^{46})
\end{align*}
This structure reminds me of the Pfaffine of a $4\times 4$ matrix.
\begin{equation*}
	H =
	\begin{pmatrix}
		0 & \eta^{24} & \eta^{26} & \eta^{28} \\
		0 & 0 & \eta^{46} & \eta^{48} \\
		0& 0& 0 & \eta^{68} \\
		0& 0& 0& 0
	\end{pmatrix}
\end{equation*}

\paragraph{compare with bosonic part}
In the bosonic part, we are dealing with the contraction of the following structure
\[
	\mathcal{F}^{12} \mathcal{F}^{34}
	= (\mathcal{A}_1 \mathcal{A}_2 - \mathcal{A}_2 \mathcal{A}_1)
	(\mathcal{A}_3 \mathcal{A}_4 - \mathcal{A}_4 \mathcal{A}_3)
.\] 
\[
=\mathcal{A}_1 \mathcal{A}_2 \mathcal{A}_3 \mathcal{A}_4
- \mathcal{A}_1 \mathcal{A}_2 \mathcal{A}_4 \mathcal{A}_3
- \mathcal{A}_2 \mathcal{A}_1 \mathcal{A}_3 \mathcal{A}_4
+ \mathcal{A}_2 \mathcal{A}_1 \mathcal{A}_3 \mathcal{A}_4
.\] 
The index only about the Lorentz index $\mu$, not the matrix index $\mathfrak{a}$.
The contraction must be $\eta^{14} \eta^{23}$
(already specified by the matrix product structure)
\[
	= \eta^{14} \eta^{23}\mathcal{A}_1 \mathcal{A}_2 \mathcal{A}_3 \mathcal{A}_4
-\eta^{14} \eta^{23} \mathcal{A}_1 \mathcal{A}_2 \mathcal{A}_4 \mathcal{A}_3
-\eta^{14} \eta^{23} \mathcal{A}_2 \mathcal{A}_1 \mathcal{A}_3 \mathcal{A}_4
+\eta^{14} \eta^{23} \mathcal{A}_2 \mathcal{A}_1 \mathcal{A}_3 \mathcal{A}_4
.\] 
But then let's rewrite it as
\[
	= (\eta^{14} \eta^{23} - \eta^{13} \eta^{24}
	- \eta^{24} \eta^{13} + \eta^{14} \eta^{23})
	\mathcal{A}_1 \mathcal{A}_2 \mathcal{A}_3 \mathcal{A}_4
.\] 
\[
	= 2(\eta^{14} \eta^{23} - \eta^{13} \eta^{24})
	\mathcal{A}_1 \mathcal{A}_2 \mathcal{A}_3 \mathcal{A}_4
.\] 

A crucial difference between fermionic and bosonic part:
for the fermionic part, there are two separate contractions:
matrix $\mathfrak{a}$ and Lorentz index $\mu$ (from $\Gamma$-trace);
however, for the bosonic part, they happen simultaneously, thus related.
