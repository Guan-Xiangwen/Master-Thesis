\documentclass[a4paper]{article}

% Packages
\usepackage{cmbright}	% computer modern bright font
\usepackage[T1]{fontenc}
			% font encoding: related to the accented characters
\usepackage{
	amsmath,	% math
	amssymb,	% math symbols
	amsthm,		% theorem
	amsfonts,	% use \mathfrak
	dsfont,
}
\usepackage{
	bookmark,	% replaces (and loads) hyperref
	chngcntr,	% use counterwithin
	enumerate,	% enumerate enviornments
	fancyhdr,	% better headers
	float,		% force image placement
	mathrsfs,	% more math scripts
	mathtools,	% allows prescripts
	microtype,	% improved appearence
	multicol,	% use multi-columns
	multirow,	% use multi-rows
	nicefrac, 	% alternative fractions
	relsize,	% bigger math symbols
	stmaryrd,	% power series brackets
	thmtools,	% custom theorem environments
	thm-restate,	% for re-stateable theorems
	totcount,	% define counters
	xcolor,		% define colors
}
\usepackage{tikz-cd}
\usepackage{soul}	% underlining and striking out
\usepackage{parskip}	% handle indentation of paragraphs

% figure support ...? not sure about its function
\usepackage{import}
\usepackage[shortlabels]{enumitem}
\usepackage{xifthen}
\pdfminorversion=7
\usepackage{pdfpages}
\usepackage{transparent}
\newcommand{\incfig}[1]{%
    \def\svgwidth{\columnwidth}
    \import{./figures/}{#1.pdf_tex}
} 

\usepackage{hyperref}
\hypersetup{hidelinks}% hide links (remove color and boarder)
\usepackage{fontawesome}% font containing web-related icons: \faLink below
\usepackage{xifthen}% provides \isempty test: check whether a string is empty or not
\newcommand\pdfref[3]{%
    \href{MasterThesis://open-paper?id=#1&page=#2}{%
    \textup{[\textbf{\ifthenelse{\isempty{#3}}{here}{#3}}]}}% print a text (#3) linking to the address
}% how to understand the first argument of \href: phd://open-paper?id=#1&page=#2
\newcommand\urlref[2]{%
    \href{#1}{\raisebox{0.15ex}{\scriptsize \faLink}\:\textup{\textbf{#2}}}%
}
\newcommand\absolutefileref[2]{%
    \href{run:#1}{\raisebox{0.15ex}{\scriptsize \faFile}\:\textup{\textbf{#2}}}%
}

% this will contain the current date in yyyy-mm-dd format
\def\formatteddate{}
\newcommand\fileref[2]{
    \IfFileExists{./\formatteddate/#1}{
        \absolutefileref{./\formatteddate/#1}{#2}
    }{
        \textcolor{gray}{\absolutefileref{./\formatteddate/#1}{#2}}
    }
}
%\newcommand{\xournal}{\fileref{note.xoj}{Handwritten notes}}% I don't have the Handwritten notes

\DeclareMathOperator{\Tr}{Tr}


\usepackage{mdframed}
\mdfsetup{
    leftmargin=-1em,
    rightmargin=-1em,
    middlelinewidth=1.5pt,
    middlelinecolor=white,
    innertopmargin=1.5\topskip,
    innerbottommargin=1\topskip,
    skipabove=\baselineskip,
    skipbelow=0,
    nobreak=true
}

\newmdenv[
    backgroundcolor = red!10
]{wrong}

\newmdenv[
    backgroundcolor = green!10
]{correct}

\newmdenv[
    backgroundcolor = yellow!10
]{info}

\newmdenv[
    backgroundcolor = blue!10
]{reference}

\newmdtheoremenv[linewidth=0]{question}{Question}
\newmdtheoremenv[linewidth=0]{confusion}{Confusion}
\newmdtheoremenv[linewidth=0]{answer}{Answer}

\newmdtheoremenv[]{idea}{Idea}
\newmdtheoremenv[]{claim}{Claim}
\newmdtheoremenv[]{definition}{Definition}
\newmdtheoremenv[]{lemma}{Lemma}
\newmdtheoremenv[linewidth=0]{remark}{Remark}
\newmdtheoremenv[]{theorem}{Theorem}
\newmdtheoremenv[]{corollary}{Corollary}
\newmdtheoremenv[]{problem}{Problem}
\newmdtheoremenv[]{eg}{Example}
\newmdtheoremenv[linewidth=0]{todo}{TODO}
\newmdtheoremenv[backgroundcolor = blue!10, nobreak=false]{marco}{Marco}

\newmdtheoremenv[backgroundcolor=green!5]{mydefinition}{Definition}
\newmdtheoremenv[backgroundcolor=green!5]{mytheorem}{Theorem}
\newmdtheoremenv[backgroundcolor=green!5]{mylemma}{Lemma}

\renewmdenv[
    bottomline=false,
    topline=false,
    rightline=false,
    fontcolor=black!70
]{quote}

\newmdenv[fontcolor=black!10, linewidth=0]{ditch}


\usepackage{pgfmath}
\usepackage{pgfcalendar}

\let\d=\pgfcalendarshorthand	% \pgfcalendarshorthand{<kind>}{<representations>}, expand to a <representation> of current date depending on <kind>: d, m, y, etc.
\newcommand\formatdate[2]{\pgfcalendar{cal}{#1}{#1}{#2}}	% use \pgfcalendar to represent the date (#1) in a way specified by #2

\newcommand\firstdate{\year-\month-\day+-7}
\newcommand\lastdate{\year-\month-\day}

\begin{document}

\begin{center}
    \huge{Master Thesis Notes}\\[0.4em]
    \Large{Xiangwen Guan}\\[0.2em]
        From \formatdate{\firstdate}{\d d- \d mt} to
        \formatdate{\lastdate}{\d d- \d mt}	% format the \firstdate and \lastdate using \d
\end{center}

\tableofcontents
\bigskip

\pgfcalendar{cal}{\firstdate}{\lastdate}{% rendering the following codes from \fristdate to \lastdate, the date is given to \d
	\IfFileExists{./\d y0-\d m0-\d d0/note.tex}{
		\marginpar{\vspace*{1em}\textsf{ \d w., \d m. \d d-}}
		\addcontentsline{toc}{section}{\d wt, \d d0 \d m.}
		\def\formatteddate{\d{y}0-\d{m}0-\d{d}0}
		%! Tex root: ../master.tex

The idea of the Gaussian expansion method is to expand the action
\[
	S[A_0] = -\frac{1}{g_0^2} \mathrm{Tr}[A_0,A_0]^2
.\] 
around a Gaussian action,
such that one can do the calculation order by order.
There is no unique choice of the Gaussian action beforehand,
so one usually introduce a family of them,
which are parameterized by some free parameters.
Practically,
the calculation result will depend on these free parameters
if one truncates at certain order,
which is understood as an artifact due to the truncation.
The hope is that,
the truncated result could be a good approximation for the true result
at point of the parameter space.

For example, one could take the Gaussian action as
\[
	S_G[A_0] = \frac{N_0}{2v} \mathrm{Tr}A_0^2
.\] 
and write the action $S[A_0]$ as
\[
	S[A_0] = S[A_0] - S_G[A_0] + S_G[A_0]
.\] 
Here the free parameter is $v$,
which controls the variance of the Gaussian distribution.
This is a natural choice because the Lorentzian index $\mu$
is contracted by the metric tensor $\eta_{\mu\nu}$.
However, the translation symmetry mentioned above
$A_0 \to A_0 + \lambda \mathds{1}$
is violated by this Gaussian action.
This means that this symmetry will not be manifest
during the perturbative calculation.

The partition function could then be written as
\begin{equation}
	Z[g_0,N_0] = \int \mathrm{e}^{-S[A_0]} [\mathrm{d}A_0]
	= \int \mathrm{e}^{-(S[A_0] - S_G[A_0])-S_G[A_0]} [\mathrm{d}A_0]
	= \left<\mathrm{e}^{-(S[A_0] - S_G[A_0])} \right>_G
\end{equation}
The goal of our calculation is to find the effective action
$S_{eff}[A']$
such that
\[
	\left<\mathcal{O}(A)\mathrm{e}^{-(S[A_0] - S_G[A_0])} \right>_{G,N_0} = 
	\left<\mathcal{O}(A) e^{-(S_{eff}[A'] - S_G[A'])} \right>_{G,N}
.\] 
$\mathcal{O}(A)$ represents any function of the submatrix $A$.
$A'$ is a matrix related with $A$ by a rescaling $A' = \lambda A$.
This rescaling is usually necessary to make the RG calculation repeatable
in each step of the iteration,
by keeping the structure of the action the same,
especially the normalization of the quadratic term.
However, the difficulty in this calculation is that
we don't have an intrinsic quadratic term with a canonical normalization.
This will lead to an \emph{ambiguity} on how to rescale $A$.
If $\mathcal{O}(A)$ is homogenous in $A$ with power $\Delta$, then
\[
	\mathcal{O}(A) = \mathcal{O}(\lambda^{-1} A')
	= \lambda^{-\Delta} \mathcal{O}(A')
.\] 
The desired equality should be
\[
	\left<\mathcal{O}(A)\mathrm{e}^{-(S[A_0] - S_G[A_0])} \right>_{G,N_0} = 
	\lambda^{-\Delta}
	\left<\mathcal{O}(A') e^{-(S_{eff}[A'] - S_G[A'])} \right>_{G,N}
.\] 

In principle, we should start with the most general form action
that is allowed by assumed symmetries.
This is because all possible terms that are compatible with symmetries
will be generated along the RG flow.
However, this requirement is hard to realize in a practical calculation.
One possible idea is to use certain Schwinger-Dyson equation
to reduce the number of independent operators.
However, at this stage, it's not clear how to implement this idea.

According to the RG method, we write the action in the following way
\begin{align*}
	S[A_0] = S[A] + S'[\alpha,a;A],\\
	S_0[A_0] = S_0[A] + S_0'[\alpha,a].
\end{align*}
Schematically, $S'[\alpha,a;A]$ has the form
\[
	S'[\alpha,a;A] =- \frac{1}{g_0^2}
	\left( \alpha^\dagger A^2 \alpha + \alpha^\dagger a A \alpha
	+ \alpha^\dagger a^2 \alpha + (\alpha^\dagger \alpha)^2\right) 
.\] 
$S_0'[\alpha,a]$ has the form
\[
	S_0'[\alpha,a] = \frac{N_0}{v}
	\left( (\alpha^\mu)^\dagger \alpha_\mu + \frac{1}{2}a^\mu a_\mu\right) 
.\] 

What we need to do to get from $\left<\cdots \right>_{G,N_0}$ to
$\left<\cdots \right>_{G,N}$ is to integrate out $\alpha,\alpha^\dagger$ and $a$.
The contraction rule is given by the Gaussian action
\begin{align*}
	\left<(\alpha^\mu_i)^* \alpha^\nu_j \right>_{G,N_0} = \frac{v}{N_0}
	\eta^{\mu\nu}\delta_{ij}
	\\
	\left<a^\mu a^\nu \right>_{G,N_0} = \frac{v}{N_0}\eta^{\mu\nu}
\end{align*}

The exponential function is expanded as
\[
	\mathrm{e}^{-(S'-S_0')} = 1 - (S' - S_0') + \frac{1}{2} (S' - S_0')^2
	- \frac{1}{6} (S'- S_0')^3 + \frac{1}{24} (S' - S_0')^4 + \cdots
\]
One problem in this calculation is that there is no small parameter
to give a hierarchy for the expansion.
It's not clear whether or not what we obtain by truncating the expansion
will capture certain asymptotic behavior of the integration.

There are many terms in the expansion.
Let's try to organize them in terms of
1. how many $A$
2. the order of $\frac{1}{g^2_0}$.
One expects that only even number of $A$ will appear, say $0,2,4,6,8,\cdots$.
Some shorthand notations
\begin{align*}
	(\overline{\alpha} A^2 \alpha) \equiv -2(\alpha^\mu)^\dagger (2 A_\nu A_\mu - A_\mu A_\nu -  A^2 \eta_{\mu\nu}) \alpha^\nu
	\\
	(\overline{\alpha} a A \alpha) \equiv
	 2 (\alpha^\mu)^\dagger (a_\mu A_\nu + a_\nu A_\mu - 2 a^\rho A_\rho \eta_{\mu\nu}) \alpha^\nu
	 \\
	 (\overline{\alpha} a^2 \alpha) \equiv -2 (\alpha^\mu)^\dagger
	(a_\mu a_\nu - a^\rho a_\rho \eta_{\mu\nu}) \alpha^\nu
	\\
	(\overline{\alpha}\alpha\overline{\alpha}\alpha) \equiv 
	-2 \left[	(\alpha^\mu)^\dagger \alpha^\nu (\alpha_\mu)^\dagger \alpha_\nu 
	- (\alpha^\mu)^\dagger \alpha^\nu (\alpha_\nu)^\dagger \alpha_\mu
	- (\alpha^\mu)^\dagger \alpha_\mu (\alpha^\nu)^\dagger \alpha_\nu \right]
\end{align*}
Then the $S'[\alpha,a;A]$ is written as
\[
	S'[\alpha,a;A] = \frac{1}{g_0^2} \left[ (\overline{\alpha} A^2 \alpha)
	+ (\overline{\alpha} a A \alpha) + (\overline{\alpha} a^2 \alpha) +(\overline{\alpha}\alpha\overline{\alpha}\alpha) \right] 
.\] 

Let's check some contractions:
\[
	(\overline{\alpha} A^2 \alpha) \to 2 \frac{1}{N_0} (D-1) v\mathrm{Tr}A^2
.\] 
\[
	(\overline{\alpha} a a \alpha) \to 2 \frac{N}{N_0^2} D (D-1) v^2
.\] 
\[
	(\overline{\alpha}\alpha \overline{\alpha}\alpha)
	\to 2 \left( \frac{N^2}{N_0^2} + \frac{N}{N_0^2} \right)D^2 v^2   
.\] 
When dealing with multiple brackets contraction $(\cdots)(\cdots)...(\cdots)$
let's focus only on the connected part.
The all possible contractions can be recovered by exponentiate the connected part.
When $(\overline{\alpha}\alpha)$ or $(aa)$ from $S_0'$ comes into play
\[
	(\overline{\alpha} A^2 \alpha) (\overline{\alpha} \alpha)
	\to 2 \frac{1}{N_0} (D-1) v\mathrm{Tr}A^2
.\] 
This is the same as $(\overline{\alpha}A^2\alpha)$.
It's easy to calculate also the following contraction
\[
	(\overline{\alpha}A^2\alpha)(\overline{\alpha}\alpha)^k
	\to 2 k! \frac{1}{N_0} (D-1) v \mathrm{Tr}A^2
.\] 
It's a general result that attaching $(\overline{\alpha}\alpha)$
and considering only connected contractions will only modify the result
by a combinatoric number.
But the exact number depends on the way you contract.
Attaching $(aa)$ has a similar effect.

Now let's consider the contraction between two $S'$.
The followings are the possibilities
\[
	(\overline{\alpha}A^2\alpha) (\overline{\alpha}A^2\alpha)
	\to
	4 \left( \frac{v}{N_0} \right)^2 \left[ (3+D) \mathrm{Tr}(A^\mu A_\mu A^\nu A_\nu) - 4 \mathrm{Tr}(A^\mu A^\nu A_\mu A_\nu) \right] 
.\] 
\[
	(\overline{\alpha} A^2 \alpha) (\overline{\alpha} a a \alpha)
	\to
	4\left( \frac{v}{N_0} \right)^3 (D-1)^2 \mathrm{Tr}A^2 
.\] 

While integrating out $v,a$, the $S'[v,a;A]$ will provide new terms of $A$.
The strategy for integration is as usual:
first expand the exponential function to a certain order
according to the $N_0\to\infty$ reigme that we are interested in.
Also there are connected terms as well as disconnected terms in the expansion.
If one is interested in the free energy $F=-\ln Z$,
one could only consider the connected terms.
(This point need to be justified although it's quite standard in QFT):

One should first be careful about how $S$ depends on $N_0$.
If one wants to take the 't Hooft limit $\lambda = g_0^2 N_0$ fixed as $N_0\to\infty$.
This will lead to the prefactor that is proportional to $N_0$.
Then expanding the exponential will not give terms with a proper $N_0$ hirarchy.

Let's try to do the calcuation to the order of $\frac{1}{\lambda}$ and $\frac{1}{\lambda^2}$.
For $S',S_0'$, let's ignore $a$ terms temporarily
\begin{align*}
	S'[v;A] &= - \frac{2}{g_0^2} (v^\mu)^\dagger
	(2 A_\nu A_\mu - A_\mu A_\nu - 2 A^2 \eta_{\mu\nu}) v^\nu \\
	S'_0[v] &= \frac{N_0}{2 v} (v^\mu)^\dagger v_\mu
\end{align*}
Denote
\[
	M_{\mu\nu}\equiv 2 A_\nu A_\mu - A_\mu A_\nu - 2 A^2 \eta_{\mu\nu}
.\] 
The expansion then gives
\begin{align*}
	1 + \frac{2}{g_0^2} (v^\mu)^\dagger M_{\mu\nu} v^\nu
	+ \frac{N_0}{2v} (v^\mu)^\dagger v_\mu
	+ \frac{2}{g_0^4} (v^\mu)^\dagger M_{\mu\nu} v^\nu
	(v^\rho)^\dagger M_{\rho\sigma} v^\sigma \\
	+ \frac{N_0}{g_0^2 v} (v^\mu)^\dagger M_{\mu\nu} v^\nu (v^\rho)^\dagger v_\rho
	+ \frac{N_0^2}{8 v^2} (v^\mu)^\dagger v_\mu (v^\nu)^\dagger v_\nu
\end{align*}
Focus on the connected contraction, using the contraction rule
\[
	\left<(v^\mu)^*_i v^\nu_j \right> = \frac{v}{N_0}\eta^{\mu\nu}\delta_{ij}
.\] 
For example, the term $ \frac{2}{g_0^2} (v^\mu)^\dagger M_{\mu\nu} v^\nu$
will be contracted to give
\[
	\frac{2}{g_0^2} \frac{v}{N_0} \eta^{\mu\nu} \mathrm{Tr} M_{\mu\nu}
	= \frac{2v}{g_0^2 N_0} (1-2D) \mathrm{Tr}A^2
.\] 
The term
\[
	\frac{N_0}{g_0^2 v} (v^\mu)^\dagger M_{\mu\nu} v^\nu
	(v^\rho)^\dagger v_\rho
.\] 
has two possible contractions, but one is disconnected.
The connected one gives
\[
\frac{v}{g_0^2 N_0} (1-2D) \mathrm{Tr} A^2
.\] 
So at this expansion order,
we will have a $\mathrm{Tr}A^2$ term that contributed to the free energy,
arising from integrating out the variable $v,v^\dagger$.
The coefficient is $ (3v)(1-2D)/(g_0^2 N_0)$.
This shift in $\mathrm{tr}A^2$ makes it impossible to take the partition function back to the original form.
This implies the RG method is not applicable?
Note also that this shift depends on $v$,
which implies that it's an artifact follows from the truncation of the exponential function.
Maybe we need to calculate higher order terms to see whether there is indeed a ``physical'' shift in $\mathrm{tr} A^2$.

	}{}
}
\end{document}
