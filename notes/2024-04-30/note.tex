%! Tex root: ../master.tex

Just like identifying the supergravity solution of extremal $p$-brane as the geometry generated by the D$p$-brane in string theory, there are also solutions that carry both electric and magnetic charges could be identified as the geometry of D$p$/D$(6-p)$ brane bound states.
\textcolor{red}{example here for D$1$/D$5$ bound state}

\paragraph{the ansatz for D$(-1)$/D$7$}
\begin{equation*}
    F_1 = \alpha (y) \mathrm{d} x
    + i \beta(y) \mathrm{d} y
\end{equation*}
The magnetic piece of $F_1$: $\alpha(y)\mathrm{d}x$?
The electric piece of $F_1$: $i \beta(y) \mathrm{d}y$?

$F_1^2 = g^{\mu\nu} (F_1)_\mu (F_1)_\nu = g^{x x} \alpha^2 - g^{ y y} \beta^2$.
$g^{xx}$ and $g^{yy}$ are positive definite in Euclidean signature?
Note that electirc and magnetic contributes oppositely.

$F_9 = \hodge F_1 = \beta (\cdots) \mathrm{d}x \wedge \mathrm{d}\theta
+ i \alpha (\cdots) \mathrm{d}y \wedge \mathrm{d}\theta$.
The imaginary number $i$?

D$7$-branes are wrapped over $\mathbb{T}^8$.
$F_9$, the dual of $F_1$, has ``legs'' along $\mathbb{T}^8$ and transverse.
The source of $F_1$ or $F_9$? Magnetic source and electric source,
D$(-1)$ or D$7$?

\paragraph{the D$1$/D$5$ example}
The metric in string frame (appearing in \pdfref{Dou03} section 4.2.1)
\begin{equation}
	\mathrm{d}s^2 = \frac{1}{\sqrt{H_1 H_5}} (-\mathrm{d}x_0^2 + \mathrm{d}x_1^2)
	+ \sqrt{H_1 H_5}(\mathrm{d}r^2 + r^2 \mathrm{d}\Omega_3^2)
	+\sqrt{\frac{H_1}{H_5}} (\mathrm{d}x_6^2 + \cdots + \mathrm{d}x_9^2)
\end{equation}
$x_0,x_1$: directions along the world-volume of D$1$ and D$5$;
$x_6,\cdots,x_9$: directions along the world-volume of D$5$, transverse to D$1$;
$r,\Omega_3$: transverse directions.
Note a pattern: longitudinal direction has metric factor $1/\sqrt{H}$;
transversal direction has metric factor $\sqrt{H}$.
Interpretation: D$1$ smeared over D$5$?

RR-fluxes:

field theory, D$1$/D$5$ CFT?
supergravity solution preserves $8$ susy.
