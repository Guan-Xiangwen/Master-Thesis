\subsection{a nonlinear RG equation for matrix model}
This part is a review of \cite{higuchi_renormalization_1995}. Consider the following matrix integral
\begin{equation}
    Z_N(g_j) = \int \mathrm{d}\phi \mathrm{exp}[-N\mathrm{Tr} V(\phi)],
\end{equation}
where
\[V(\phi) = \sum_{k\geq 1}\frac{g_k}{k}\phi^k.\]
The free energy is then defined as
\begin{equation}
    F(N,g_j) = -\frac{1}{N^2} \mathrm{log} \left(\frac{Z_N(g_j)}{Z_N(g_2=1,\text{others}=0)}\right).
\end{equation}
Later, the RG equation will be of the form
\begin{equation}
    \left[N\frac{\partial}{\partial N} + 2\right]F(N,g) = G\left(g,\frac{\partial F}{\partial g}\right) + O(\frac{1}{N})
\end{equation}
The normalization $1/N^2$ of $f(N,g)$ gives the factor $2$ in the above RG equation, which is designed to reproduce the scaling law of string partition function \cite{brezin_renormalization_1992}.

Starting from $Z_{N+1}(g)$, perform the integration over the last row and column, then we obtain an induced matrix model with rank $N$. The partition function of the induced model is the same as $Z_{N+1}(g)$. In \cite{higuchi_renormalization_1995}, the integration gives
\begin{align*}
    Z_{N+1}(g) = \left(\frac{\pi}{N+1}\right)^N \int \mathrm{d}\phi \mathrm{exp}[-(N+1)\mathrm{tr}V(\phi)] \\
    \cdot \int\mathrm{d}\alpha \mathrm{exp}[-(N+1)V(\alpha) - \mathrm{tr}~\mathrm{log}(\mathds{1}+g(\phi+\alpha\mathds{1}))]
\end{align*}
This is different from $Z_N(g)$, and the RG equation formulate the difference. The $\alpha$ integral is performed by the saddle point method. Denote the saddle point value $\alpha_s = \alpha_s(g,\phi)$, which is determined by the following saddle point equation
\begin{equation}
    \alpha_s + g \alpha_s^2 + \frac{g}{N}\mathrm{tr}\frac{1}{\mathds{1}+g(\phi+\alpha_s\mathds{1})} = 0.
\end{equation}
At this point $\alpha_s$ is not solved in terms of $g$ and $\phi$, but latter we can solve it with the help of resolvent and loop equations. 

The factorization property of large-N limit: for $U(N)$-invariants $\mathcal{O},\mathcal{O}'$, we have
\begin{equation}
    \langle \mathcal{O}\mathcal{O}' \rangle = \langle \mathcal{O} \rangle \langle \mathcal{O}' \rangle + O(N^{-2})
\end{equation}
can be used to put the average $\langle \cdots \rangle$ inside any polynomial functions of gauge invariant quantities. This leads to a great simplification which enables us to derive an RG equation.

To derive the RG equation, we can start by considering the ratio
\[\frac{Z_{N+1}(g)}{Z_N(g)} = \cdots.\]
The above consideration enables us to express the right hand side as an exponential function
\[\frac{Z_{N+1}(g)}{Z_N(g)} = \left(\frac{\pi}{N+1}\right)^N \mathrm{exp}[-\langle \mathrm{tr}V(\phi)\rangle - N V(\langle \alpha_s \rangle) - \langle \mathrm{tr}~\mathrm{log}(\mathds{1}+g(\phi+\alpha_s \mathds{1})) + O(N^0)\rangle]\]
This, combined with the definition of free energy, gives the following prototype of the RG equation
\begin{equation}
    \left[N\frac{\partial}{\partial N} + 2\right] F(N,g) = -\frac{1}{2} + \langle \frac{1}{N}\mathrm{tr}V(\phi)\rangle + V(\langle \alpha_s\rangle) + \langle \frac{1}{N} \mathrm{tr}~\mathrm{log}(\mathds{1}+g(\phi+\langle \alpha_s\rangle\mathds{1}))\rangle + O(\frac{1}{N}).
\end{equation}
The right hand side is a function of $\langle \cdots \rangle$ in the rank $N$ model, which can be generated by the derivatives of free energy. $\langle \alpha_s \rangle$ is understood as $\langle \alpha_s (g,\phi)\rangle$, for which the concrete form still need to be solved by resolvent method. Although it's not obvious at this step, the terms we retain on the right hand side is of order $O(N^0)$. This will be verified by the loop equations which essentially relates different correlators
\[\langle \frac{1}{N} \mathrm{tr} \phi^i \rangle,\]
there will be no other $N$ factors appear in the loop equation, which indicates all such correlators are of the same $N$ order. For $\langle \alpha_s \rangle$, it turns out that, from again loop equations,
\[\langle \alpha_s \rangle = \langle \frac{1}{N}\mathrm{tr}\phi \rangle\]
This fact is consistent with how $\alpha$ appears in the matrix decomposition.

Let's write down the nonlinear RG equation we obtain through this procedure. To be specific, \cite{higuchi_renormalization_1995} takes the cubic potential
\begin{equation}
    \left(N\frac{\partial}{\partial N} + 2\right) F(N,g) = G(g,\frac{\partial F}{\partial g}) + O(\frac{1}{N})
\end{equation}
with
\[G(g,a) = -\frac{g}{2}a + \frac{1}{2}\bar{\alpha}(g,a)^2 + \frac{g}{3}\bar{\alpha}(g,a)^3 + \mathrm{log}(1 + g\bar{\alpha}(g,a)) + \int_{-\infty}^{-1/g-\bar{\alpha}(g,a)}\mathrm{d}z \left(W(z,g,a) - \frac{1}{z}\right)\]
\[\bar{\alpha}(g,a) = -g + 3 g^2 a \equiv \langle \alpha_s \rangle\]
\[W(z,g,a) = \frac{1}{2}\left(z + gz^2 - \sqrt{(z+gz^2)^2 - 4(1+gz-g^2+3g^3 a)}\right)\]
$W(z,g,a)$ is the average value of the resolvent
\[W(z,g,a) = \langle \frac{1}{N}\mathrm{tr}\frac{1}{z\mathds{1}-\phi}\rangle\]
and $a$ is the notation for
\[a_j = \frac{1}{j}\langle \frac{1}{N}\mathrm{tr}\phi^j \rangle = \frac{\partial F}{\partial g_j}\]
The resolvent here can be understood as a way to encode all higher interactions $\mathrm{tr}\phi^k$ contribution to the RG equation in terms of what appears in the original potential $V(\phi)$.

The information of critical point $g_*$ and the critical exponent $\gamma_0$ can be extracted from the RG equation. In \cite{higuchi_renormalization_1995}, $g_*,\gamma_0$ is characterized by the singular scaling behavior of the free energy. The leading order of free energy in large N expansion is $O(N^0)$, so let's denote it as $F^0(g)$. Then $F^0(g)$ is assumed to contain an analytic part and non-analytic part around the critical point $g_*$,
\begin{equation}
    F^0(g) = \sum_{k=0}^{\infty} a_k(g-g_*)^k + \sum_{k=0}^{\infty} b_k(g-g_*)^{k+2-\gamma_0},\quad \gamma_0\notin \mathbb{Z}.
\end{equation}
The right hand side of the RG equation depends on $g$ and $\partial F^0(g)/\partial g$. Let's denote at the critical point $\partial F^0(g)/\partial g\vert _{g=g_*} = a_1$. Then $G(g,a)$ is assumed to have the following expansion form
\begin{equation}
    G(g,a) = \sum_{n=0}^{\infty} \sum_{k=0}^{\infty} \beta_{n,k} (g-g_*)^k (a-a_1)^n.
\end{equation}
Then the idea is to compare the coefficients of various powers of $g-g_*$ on both sides. But the problem is that there are non-analytic parts in $F^0(g)$, which does not appear in $G(g,a)$ \textcolor{red}{To be continue.......}

\subsection{notes}
A notion of \textbf{canonical dimension} of coupling constants, like $g_k \mathrm{Tr} \phi^k$, is defined in \cite{lahoche_revisited_2020}. In the vicinity of the Gaussian fixed point, we expect $g_k,k\geq 2$ has the following scaling
\[g_k \sim N^{-d_k + \mathcal{O}(g_2,g_3,\cdots)}\]
then $d_k$ is called the canonical dimension. Let's discuss how this scaling is obtained. Consider the following partition function
\[Z_{N+1}(g_3,g_4,\cdots) = \int [\mathrm{d}\phi]\mathrm{exp}\left(-\frac{1}{2}\mathrm{Tr}\phi_{N+1}^2 - \frac{g_3}{3}\mathrm{Tr} \phi_{N+1}^3 - \frac{g_4}{4}\mathrm{Tr} \phi_{N+1}^4 + \cdots\right).\]
Following the conventional method, the terms involving $v,v^\dagger$ appear in the potential
\[\mathrm{exp}\left\{\cdots-v^\dagger v - g_3 v^\dagger (\phi_N+\alpha\mathds{1}) v - g_4 \left[v^\dagger (\phi_N ^2 + \alpha \phi_N + \alpha^2\mathds{1}) v + \frac{1}{2}(v^\dagger v)^2\right]+\cdots\right\}\]
To the first order of the coupling constant, the integral over $v,v^\dagger$ gives terms like
\[- g_3 \mathrm{tr}(\phi_N + \alpha\mathds{1}) - g_4 \mathrm{tr}(\phi_N^2 + \alpha\phi_N + \alpha^2 \mathds{1})+\cdots\]
If we re-exponentiate them, there will be new quadratic terms like $-g_4 \mathrm{tr}\phi^2$. To keep the normalization of the quadratic term, we need to rescale $\phi_N$ as
\[\phi_N' = (1+g_4)\phi_N\]
In all of these calculations, we keep only to the first order of $g_4$. Therefore, at the Gaussian fix point $g_4=0$, there is no scaling of $\phi_N$. Look at the coefficient of $\mathrm{Tr}(\phi_N')^4$, define it as $g_4'/4$, $g_4'$ is equal to
\[g_4' = g_4 (1 - 4 g_4)\]
In the theory, written in this way (with no factor $N$ in the action), all couplings $g$ and the matrix $\phi$ are dimensionless, if we take $N$ as the only dimensional scale.
\textcolor{red}{To be continue.......}
