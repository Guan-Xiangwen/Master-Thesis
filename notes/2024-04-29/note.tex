%! Tex root: ../master.tex

\section{D$p$/D$(6-p)$ bound states}

\begin{info}
	\emph{references} The notes from Sebastien \pdfref{RRT-draft}.
	The $\mathrm{AdS}_1 \times \mathrm{S}^1$ paper \pdfref{SKT22}.
	A review of duality in supergravity solution \pdfref{DKL94}.
	String theory discussion Polchinski.
	The review of brane-intersections and corresponding susy \pdfref{Dou03}.
\end{info}

\begin{info}
	\emph{terminology}: dyon. the particle that carries electric and magnetic charges.
	electric charge. the Noether charge associated with the equation of motion.
	magnetic charge. the topological charge associated with the Bianchi identity.
	Montonen-Olive duality. A strong-weak duality for the gauge field theory living on the D$3$-brane.
	It can be understood as a consequence of $SL(2,\mathbb{Z})$ duality (but in our context, it's not the $SL(2,\mathbb{Z})$.
	brane-probe condition. a brane probe (without backreaction) the susy (BPS) background generated by the same type of branes should feel no force.
	moduli space. 
\end{info}

The bosonic truncation of type IIA/B supergravity (for a single R-R field, Einstein frame)
\begin{equation}
S = \int \sqrt{|g|} \left( R - \frac{1}{2} (\partial\phi)^2 - \frac{1}{2} \frac{1}{(p+2)!} \mathrm{e}^{\frac{(3-p)}{2}\phi} F_{p+2}^2 \right)
\end{equation}
to understand: $F^2$ term leads to a dilaton gradient,
diverge near the ``would-be horizon''?;
electric charge and magnetric charge contribute to $F^2$,
with opposite sign?
Vanish for D$1$/D$5$? Near horizon or overall? Constraints on charge relations?

Understand more about the electric-magnetic duality for D$p$-branes.

The R-R charges carried by D$p$-brane is the magnetic dual to that carried by D$(6-p)$-brane.
The R-R field strength is $F_{p+2}$ for D$p$-brane.
$p=3$ as a special case (self-dual?)

\paragraph{$p$-brane and $F_{p+2}$ gauge field}
Some notions of gauge field $F_{p+2}$ of $p$-branes.
In particular, how the D$p$-brane carries electric charge, while D$(6-p)$-brane carries magnetic charge.

Consider a $p$-brane that couples to a $(p+1)$-form gauge potential $A$.
The gauge transformation is $A \to A + \mathrm{d} \lambda$
(\textcolor{red}{maybe add a factor $(p+1)$ before $\lambda_{\mu_2\cdots\mu_{p+1}}$}?)
\[
	A_{\mu_1 \mu_2 \cdots \mu_{p+1}} \to A_{\mu_1 \mu_2 \cdots \mu_{p+1}}
	+ \partial_{[\mu_1} \lambda_{\mu_2 \cdots \mu_{p+1}]}
.\] 
The field strenght $F = \mathrm{d} A$ is defined as
\[
	F_{\mu_1 \mu_2 \cdots \mu_{p+2}} = (p+2) \partial_{[\mu_1}
	A_{\mu_2 \cdots \mu_{p+2}]}
.\] 
By definition, the Bianchi identity is satisfied.
\[
\mathrm{d} F = 0
.\] 
The $p$-brane is the source of $A$, a $p+1$ form $J$.
The equation of motion is
\[
\mathrm{d} * F = * J
.\] 
The Hodge dual $*$ is defined as
\[
	(* J)^{\mu_1 \cdots \mu_{9-p}}
	\equiv \frac{1}{(p+1)!} \varepsilon^{\mu_1\cdots \mu_{10}}
	J_{\mu_{10-p}\cdots \mu_{10}}
.\] 
with $\varepsilon^{01\cdots 9} = 1$.
The equation of motion tells us that there is a conserved ``electric charge''.
The Bianchi identity tells us that there is no dual ``magnetic charge''.
The way to introduce the ``magnetric charge'' is to add a source term to the Bianchi identity.
This can be done by modifying the definition of $F$ as follows
\[
F = \mathrm{d}A + \omega
.\] 
$\omega$ could give a source term to the Bianchi identity
\[
\mathrm{d}F = \mathrm{d}\omega = X
.\] 
$X$ is a $(p+3)$ form. $*X$ is a $(7-p)$ form, and is interpreted as the magnetic charge density carried by a $(6-p)$-brane.
The electric and magnetic charge is obtained from the integration
\[
	e \equiv \int_{S^{8-p}} * F = \int_{M^{9-p}} *J
.\] 
\[
	g \equiv \int_{S^{p+2}} F = \int_{M^{p+3}} X
.\] 

Let's put this into the context of type IIB supergravity.
D$5$-brane is a magnetic source of the R-R $2$-form potential.
D$1$-brane is the electric source.

\paragraph{fluxes}
Flux is important for keeping supersymmetries?
How to understand this point?

\paragraph{supersymmetry of the bound states}
Supersymmetry: a single D$p$-brane preserve half of the supersymmetry
(out of $32$ real supercharges).
The bound states preserve less. (more details...
how it matches with that of supergravity solution?
There are methods to determine the preserved susy in some brane intersection configurations.
For example, by checking the projection conditions for both brane, whether some susy generators survive.

Two susy: world-volume and space-time.
Pull-back of gamma-matrices on the world-volume:
$
	\gamma_\mu \equiv (\partial_{\sigma^\mu}
	X^M) \Gamma_M
$

\paragraph{harmonic rules?}
Basic requirements for D$p$-brane intersections(?):
the harmonic function rules.

Open string boundary condition: Dirichlet, Neumann, Mixed.
Supersymmetric bound state: $4$ mixed boundary conditions?
(where this condition comes from?)

\paragraph{D-brane profile}
Configuration: wrap the D$(6-p)$ brane over a torus to compactify the dimension $10\to 4$? and same time put it on the same footing as the D$p$-brane in the non-compactified directions?
Wrap D$7$ brane over $T^8$.
Dissolve it in the flux of $F_5$.
Smear $D(-1)$ branes over $T^8$.

D$1$/D$5$ geometry: 
$\mathrm{AdS}_3 \times \mathrm{S}^3 \times \mathrm{T}^4$.
What's the relation between the geometry and the D$p$-brane configuration?

Lorentz v.s. Euclidean: the supergravity solution of D$(-1)$/D$7$;
D$(-1)$ only exist in Euclidean?
(Is it really the case? Why it's the case? Implication on supergravity solution?)
Be careful about the Hodge duality of $F_n$, the $G_{D-n}$,
the sign of kinetic term of the field strength:
wrong sign in Euclidean signature (generally even number of time directions)

Tha metric ansatz $\mathrm{AdS}_1 \times \mathrm{S}^1 \times \mathrm{T}^8$
\begin{equation}
	\mathrm{d}s^2 = L_y^2 \mathrm{d}y^2 + L_x^2 \mathrm{d}x^2
	+ \sum_{i=1}^8 L_i^2 (\mathrm{d}\theta^i)^2	
\end{equation}
In \pdfref{SKT22}, the $y$-direction is understood as ``Euclidean time'':
$ y = i t$ with $t$ is the Lorentz time.
This corresponds to the $\mathrm{AdS}_1$ factor.
