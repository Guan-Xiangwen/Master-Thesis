The RG method of matrix model seems having the ability to apply to a complicate model,
combining with the perturbative calculation method.
The complication may arise from two aspects:
1. the matrix product usually generate complicate interaction;
2. the integration is hard to perform in general;
However, our hope is not doing the calculation,
but understand whether or not new interaction terms will arise in a qualitative level.
Whether symmetries will prevent the action from generating the new terms.

$2 \times 2$ matrix is the simplest matrix.
It's far away from the large N limit, but some interesting properties may show already at this level.

It's not hard to write down the terms in $\mathrm{tr}(ABBA\cdots)$.
If one writes down the matrix in the following form
\[
	A = \begin{pmatrix}
		a_1 & \alpha \\
		\alpha^* & a_2
	\end{pmatrix}
.\] 
Then the next step is to integrate out $\alpha,\alpha^*$ and $a_2$.
We don't need to actually do the integration;
We only need to know what new interactions can be generated from the perturbation calculation.
If possible, it's also important to know the coupling constants of the new interactions.
