%! TeX root: ../master.tex

The RG method of matrix model seems having the ability to apply to a complicate model,
combining with the perturbative calculation method.
The complication may arise from two aspects:
1. the matrix product usually generate complicate interaction;
2. the integration is hard to perform in general;
However, our hope is not doing the calculation,
but understand whether or not new interaction terms will arise in a qualitative level.
Whether symmetries will prevent the action from generating the new terms.

$2 \times 2$ matrix is the simplest matrix.
It's far away from the large N limit, but some interesting properties may show already at this level.

It's not hard to write down the terms in $\mathrm{tr}(ABBA\cdots)$.
If one writes down the matrix in the following form
\[
	A = \begin{pmatrix}
		a_1 & \alpha \\
		\alpha^* & a_2
	\end{pmatrix}
.\] 
Then the next step is to integrate out $\alpha,\alpha^*$ and $a_2$.
We don't need to actually do the integration;
We only need to know what new interactions can be generated from the perturbation calculation.
If possible, it's also important to know the coupling constants of the new interactions.

There are some typical terms that could appear in $\mathrm{Tr}(A^4)$ and $\mathrm{Tr}(A^3 B)$
\begin{align}
	\alpha^* \alpha (a_1)^2 & \alpha^* \alpha a_1 a_2 & \alpha^* \alpha(a_2)^2\notag \\
	(\alpha^* \beta + \beta^* \alpha) a_1 a_2 & \alpha^* \alpha (b_1 a_2 + b_2 a_1) & \alpha^* \alpha (\alpha^* \beta + \beta^* \alpha)
\end{align}
Performing the integration over $\alpha^*,\alpha$ and $a_2$ perturbatively,
the question is that what the correction to the first order.
An easy way to think about the perturbative calculation is through the Wick contraction.
It gives non-zero result only if the $a_2$ is of even order,
and $\alpha^*,\alpha$ could form pairs.
For example, $\alpha^* \alpha (a_1)^2$ will give a term that is proportional to $(a_1)^2$, while $(\alpha^* \alpha) a_1 a_2$ will give zero.
There is no essence difficulty in doing these calculations.
Getting the correct combinatoric factors is important.
The number of terms explodes in higher order, so maybe some patterns will
be helpful to organize the calculation.

Another problem of the perturbative calculation is the choice of quadratic term.
For example, a quadratic term in $\alpha$ has the form $\alpha \alpha^* (\cdots)$, where the $(\cdots)$ does not contain $\alpha$.
But $(\cdots)$ could contain other variables,
The consequence is that the propogator does not have a fixed value.
This may cause difficulty on the calculation, for example, giving a non-linear term in variables that need to be integrated out.
The origin of those nonlinear terms is the interaction terms in the matrix integral.
So it's possible to treat them perturbatively.
However, if one is interested in the non-perturbative aspect,
then it's necessary to use the exact propagator.
It's possible to study the non-perturbative effect because one works with the large $N$ limit.
However our interest is not the large $N$ limit, but the property at finite $N$.
It seems like only perturbation effect is doable when $N$ is finite.
Then at certain order of $g$, the terms will involve different order of $N$.
It's very important to difference our calculation with that was done in the literatures.
