%! Tex root: ../master.tex
Study the RG flow of the ``superfield'' matrix model.
This model is mentioned in
\urlref{https://arxiv.org/abs/hep-th/9601041}{Ple96}

The matrix-valude superfield is constructed as
\begin{equation}
\Phi = \phi + \overline{\psi} \theta + \overline{\theta} \psi + \theta \overline{\theta} F.	
\end{equation}
$\theta,\overline{\theta}$ are the coordinates of the superspace.
$\phi,F$ are bosonic $N\times N$ matrix, and we assume them to be Hermitian.
$\psi,\overline{\psi}$ are fermionic $N\times N$ matrix, whose entries are Grassmann variables.
Recall that the complex conjugate of the product of Grassmann variables is defined as $ (\xi \eta)^* = \eta^* \xi^*$.
To keep $\Phi$ Hermitian $\Phi^\dagger = \Phi$, we require that
\[
	\psi^\dagger  = \overline{\psi},\quad \theta^* = \overline{\theta}
.\] 
The measure over the superfield will be the usual Berezian integral of the matrix entries
\begin{equation}
\mathrm{d}\Phi = \mathrm{d}\phi \mathrm{d}F \mathrm{d}\overline{\psi} \mathrm{d}\psi.	
\end{equation}

The supersymmetric action can be constructed as
\begin{equation}
	S[\Phi] =\mathrm{Tr} \int \mathrm{d}\theta \mathrm{d}\overline{\theta}
	\left\{ - D_\theta \Phi D_{\overline{\theta}} \Phi + \sum_{k=0}^\infty g_k \Phi^k \right\}.
\end{equation}
The super-derivative acts from the right.
The matrix model partition function is
\begin{equation}
	Z_\Phi [g_k] = \int [\mathrm{d}\Phi] \mathrm{exp} \left( - N S[\Phi] \right) 	.
\end{equation}
We have $D_\theta \Phi = \overline{\psi} - \overline{\theta} F$ and $D_{\overline{\theta}} \Phi = -\psi + \theta F$.
Their product will contribute to the action only if the measure $\mathrm{d}\theta \mathrm{d}\overline{\theta}$ is saturated.
This will give a term $\mathrm{Tr}F^2$ in the action.
We calculate the $\Phi^k$ term as
\begin{align*}
	\Phi^k &= \left( \phi + \overline{\psi}\theta + \overline{\theta} \psi + \theta \overline{\theta} F \right)^k \\
		   &= \phi^k + \left( \sum_{a+b=k-1} \phi^a \overline{\psi} \phi^b \right) \theta + \overline{\theta} \left( \sum_{a+b=k-1} \phi^a \psi \phi^b\right)  \\
		   &+ \theta \overline{\theta}\left(\sum_{a+b+c=k-2} \phi^a \overline{\psi} \phi^b \psi \phi^c + \sum_{a+b=k-1}\phi^a F \phi^b\right).
\end{align*}
By taking trace and keeping only the $\theta \overline{\theta}$ term, we get in the action a term
\[
	- \mathrm{Tr}[V'(\phi) F] - \sum_{k=0}^\infty k g_k \sum_{a+b=k-2} \mathrm{Tr}(\phi^a \overline{\psi} \phi^b \psi)
.\] 

Let's assume a quartic potential $V(\phi) = \frac{1}{2} \phi^2 + \frac{g}{4} \phi^4$,
\begin{align*}
	S[\Phi] &= \mathrm{Tr} F^2 - \mathrm{Tr} (\phi F) - g \mathrm{Tr}(\phi^3 F) 	 \\
			& - \mathrm{Tr}(\overline{\psi}\psi) - g \left[ \mathrm{Tr}(\phi^2 \overline{\psi} \psi) + \mathrm{Tr} (\phi \overline{\psi} \phi \psi) + \mathrm{Tr}(\overline{\psi} \phi^2 \psi) \right] .
\end{align*}
Although it's easy to do the integral over $F$, we will not do that.
Because it will lead to a non-linear supersymmetry transformation rule.
However, let's define $F' = F - \frac{1}{2} \phi$ and rewrite the action in terms of $F$ such that a quadratic term in $\phi$ will appear in the action
\begin{align*}
	S[\Phi] &= \mathrm{Tr} F'^2 - \frac{1}{4} \mathrm{Tr} \phi^2 - \frac{g}{2} \mathrm{Tr} \phi^4 - g \mathrm{Tr} (\phi^3 F') 	 \\
			& - \mathrm{Tr}(\overline{\psi}\psi) - g \left[ \mathrm{Tr}(\phi^2 \overline{\psi} \psi) + \mathrm{Tr} (\phi \overline{\psi} \phi \psi) + \mathrm{Tr}(\overline{\psi} \phi^2 \psi) \right] .
\end{align*}
The supersymmetry variation reads
\begin{align*}
	\delta \phi &= \overline{\varepsilon} \psi + \overline{\psi} \varepsilon \\
	\delta \psi &= -\varepsilon (F' + \frac{1}{2}\phi) \\
	\delta \overline{\psi} &= - \overline{\varepsilon} (F' + \frac{1}{2}\phi) \\
	\delta F' &= -\frac{1}{2} (\overline{\varepsilon}\psi + \overline{\psi}\varepsilon)
\end{align*}

Think about applying the RG strategy on this action.
If we denote $\phi_{i,N}=v_i,\phi_{N,i}=v_i^*$, the following term will appear
\[
-g v^\dagger F' \phi v - g v^\dagger \phi F' v
\] 
from the interaction $-g \mathrm{Tr} (\phi^3 F')$.
It will contribute a $\mathrm{Tr}(\phi F')$ term to the effective action.
Why the supersymmetry cancellation does not happen here?
This is because the supersymmetry transformation acts on all matrix entries simultaneously: it does not realize solely on the variables that being integrated out.

\begin{idea}
Study the effect of a ``local'' supersymmetry transformation: that is, those only act on a part of the matrix.	
\end{idea}

\begin{idea}
Carry out the calculation anyway, to see what you get.	
\end{idea}

For the RG strategy, let's decompose the matrix
\begin{align*}
	F'_0 = \begin{pmatrix} F' & f \\ f^\dagger & a \end{pmatrix},\quad \phi_0 = \begin{pmatrix} \phi & v \\ v^\dagger & \alpha \end{pmatrix} \\
	\psi_0 = \begin{pmatrix} \psi & \chi \\ \omega^\dagger & \beta \end{pmatrix},\quad \overline{\psi}_0 = \begin{pmatrix} \overline{\psi} & \omega \\ \chi^\dagger & \overline{\beta} \end{pmatrix}
\end{align*}
The interactions decompose correspondingly
\begin{align*}
	\frac{g}{2} \mathrm{Tr} \phi_0^4 &= 
	\frac{g}{2}\mathrm{Tr}\phi^4 
	+ 2 g v^\dagger \phi^2 v + g (v^\dagger v)^2 
	+ 2g \alpha v^\dagger \phi v + 2g \alpha^2 v^\dagger v	
	+ \frac{g}{2} \alpha^4
	\\
	g \mathrm{Tr}(\phi_0^3 F_0') &= 
	g \mathrm{Tr}(\phi^3 F') 
	+ g (v^\dagger \phi^2 f + f^\dagger \phi^2 v) 
	+ g (v^\dagger \phi F' v + v^\dagger F' \phi v) 
	\\
	&+ g (\alpha v^\dagger \phi f + \alpha f^\dagger \phi v) 
	+ g a v^\dagger \phi v + g \alpha v^\dagger F' v
	+ g (v^\dagger v v^\dagger f + v^\dagger v f^\dagger v)
	\\
	&+ g(\alpha^2 v^\dagger f + \alpha^2 f^\dagger v) 
	+ 2 g a \alpha v^\dagger v + g a \alpha^3
	\\
	g \mathrm{Tr}(\phi_0^2 \overline{\psi}_0 \psi_0 
	+ \overline{\psi}_0\phi_0^2 \psi_0) 
	&=g \mathrm{Tr}(\phi^2 \overline{\psi} \psi 
	+ \overline{\psi} \phi^2 \psi)
	+ g (v^\dagger \overline{\psi} \psi v 
	+ v^\dagger \psi \overline{\psi} v) 
	- g (\omega^\dagger \phi^2 \omega 
	- \chi^\dagger \phi^2 \chi) 
	\\
	&+ g (\chi^\dagger \psi \phi v 
	+ v^\dagger \phi \overline{\psi} \chi 
	- v^\dagger \phi \psi \omega 
	-\omega^\dagger \overline{\psi} \phi v)
	\\
	&+ g (\alpha \chi^\dagger \psi v 
	- \alpha v^\dagger \psi \omega 
	+ \alpha v^\dagger \overline{\psi} \chi 
	- \alpha \omega^\dagger \overline{\psi} v)
    \\ 
	&+ g (\overline{\beta} \omega^\dagger \phi v 
	+ \overline{\beta} v^\dagger \phi \chi 
	- \beta v^\dagger \phi \omega 
	- \beta \chi^\dagger \phi v) 
	\\
	&+ g (v^\dagger \omega \omega^\dagger v 
	- v^\dagger \chi \chi^\dagger v 
	+ v^\dagger v \chi^\dagger \chi 
	- v^\dagger v \omega^\dagger \omega)
	\\
	&+ g (\alpha \overline{\beta} \omega^\dagger v 
	+ \alpha \overline{\beta} v^\dagger \chi 
	+ v^\dagger \omega \beta \alpha 
	+ \chi^\dagger v \alpha \beta) 
	\\
	&+ g (2 v^\dagger v \overline{\beta} \beta 
	+ \alpha^2 \chi^\dagger \chi 
	- \alpha^2 \omega^\dagger \omega) 
	+ g \alpha^2 \overline{\beta} \beta
	\\
	g \mathrm{Tr} (\phi_0 \overline{\psi}_0 \phi_0 \psi_0)
	&= g\mathrm{Tr} (\phi \overline{\psi} \phi \psi)
	+ g(v^\dagger \overline{\psi} \phi \chi
	+ \chi^\dagger \phi \psi v
	- v^\dagger \psi \phi \omega
	- \omega^\dagger \phi \overline{\psi} v) 
	\\
	&+ g(\alpha \chi^\dagger \phi \chi
	- \alpha \omega^\dagger \phi \omega
	+ \overline{\beta} v^\dagger \psi v
	- \beta v^\dagger \overline{\psi} v)
	+g (\chi^\dagger v \omega^\dagger v
	+ v^\dagger \omega v^\dagger \chi)
	\\
	&+ g(\alpha \overline{\beta} \omega^\dagger v
	-\alpha \beta v^\dagger \omega
	+ \alpha \overline{\beta} v^\dagger \chi
	- \alpha \beta \chi^\dagger v)
	+ g \alpha^2 \overline{\beta}\beta. 
\end{align*}
To the first order of $g$, the non-vanishing contributions are
\[
2g v^\dagger \phi^2 v 
+ g (v^\dagger \phi F' v + v^\dagger F' \phi v) 
- g(\omega^\dagger \phi^2 \omega - \chi^\dagger \phi^2 \chi)
.\] 
The quadratic terms are $\frac{1}{2}(v^\dagger v)$
and $\chi^\dagger \chi - \omega^\dagger \omega$.

\begin{idea}
There are two ways through which the supersymmetry could help us in the calculation: 1. build up a model in which the supersymmetry is realized ``locally''; 2. develop another RG method which is compatible with the global supersymmetry;

We focus on susy because we believe it's a crucial ingredient to build a conformal matrix model.
\end{idea}
