%! Tex root: ../master.tex
\newpage

\begin{todo}
	What should I do this week?
	\begin{itemize}
		\item Explore the auxillary field method:
			Do the integration with the help of the auxillary field $H$;
			What's the effective action?
		\item Extend the matrix RG calculation to the next orders:
			What's the method to do a systematic expansion?
		\item How to deal with the model with fermionic matrix?
		\item Think about the radial coordinate $r$ and the matrix size $N$:
			is there a match between the $r$ scaling of supergravity
			and the $N$ scaling of the matrix model?
		\item Learn more about the D(-1)/D7 system:
			What's the matrix model?
			What's the supergravity solution?
	\end{itemize}
	Some new stuff to learn:
	\begin{itemize}
		\item The Generalized Conformal Symmetry (GCS):
			What's the counterpart of conformal symmetry for D$p$-brane
			background $p<3$.
		\item Relevent, irrelevent and marginal deformation of CFT:
			What's the definition?
			What's the significance in the context of AdS/CFT?
	\end{itemize}
\end{todo}

Some important problems to keep in mind:

\begin{problem}
	The 't Hooft coupling $\lambda$ in the supergravity solution;
	How does it change along the holographic RG flow?
\end{problem}

\begin{problem}
	Near horizon geometry from field theory perspective:
	radial coordinte = energy scale;
	transverse direction versus longitudinal direction;
\end{problem}

\begin{problem}
	Always be careful about the difference between the ``bare quantity''
	(in the context of QFT, without radiative correction dressing)
	and the ``renormalized quantity''
	(in our context, it's more proper to call it ``effective quantity'').
	We are not sure about the value of the bare quantity:
	they can be set arbitrarily such that the renormalized quantity
	reproduces the measured value.
	It makes sense to ask ``how the effective quantity changes with
	certain scale'';
	It makes no sense to ask ``how the bare quantity changes with
	certain scale".
\end{problem}

\begin{problem}
	How could the matrix RG result tells us something about the holography.
\end{problem}

\newpage
The near horizon limit $r\to 0$:
It's not proper to still use $r$ as coordinate
because all the things happen for $r\to 0$; 
Instead we define a finite parameter $U\equiv r / \alpha'$
($\alpha'\to 0$)
as the proper coordinate to describe the near horizon geometry.

The dilaton field profile $\Phi(r)$ is related to the running of coupling
in the dual Yang-Mills theory.
\[
	\mathrm{e}^{\Phi(r)} = \mathrm{e}^{\Phi_0} H_p(r)^{\frac{3-p}{4}}
.\] 
The holography happens in the following limit
\begin{equation}
	U \equiv \frac{r}{\alpha'} = \text{fixed},\quad \alpha'\to 0
\end{equation}
$U$ has the dimension of energy.
In this limit
\[
H_p(r) \to H_p(U) = \frac{d_p g_{\text{YM}}^2 N}{\alpha'^2 U^{7-p}},\quad
d_p = 2^{7-2p} \pi^{\frac{9-3p}{2}} \Gamma \left(\frac{7-p}{2}\right)
.\] 
Here the Yang-Mills coupling is defined for the asymptotic infinity $r\to\infty$:
\[
	g_{\text{YM}}^2 = (2\pi)^{p-2} g_s \alpha'^{(p-3) / 2}, \quad g_s \equiv \mathrm{e}^{\Phi_0}
.\] 
This is not the Yang-Mills coupling for the field theory on the D$p$-brane.

\begin{wrong}
	The following identification is wrong?
	A factor of $N$ should be added?
\end{wrong}

Previously we identify $\tilde{g}_{\text{YM}}^2$ (use this to refer to the coupling on the D$p$-brane)
\[
	\tilde{g}_{\text{YM}}^2 (U) = \frac{1}{(2\pi\alpha')^2 T_p}
	\left( \frac{d_p g_{\text{YM}}^2 N}{\alpha'^2 U^{7-p}} \right)^{\frac{3-p}{4}} 
.\] 
This does not lead to the correct $U$ scaling of the Yang-Mills coupling.

Think about the the corresponding geometry as AdS,
but the AdS radius changes with $U$:
\[
	H_p(U)^{-\frac{1}{2}} = \left( \frac{r}{L} \right)^2 
.\] 
\[
	L(U) = \left( \frac{\alpha'^2 d_p g_{\text{YM}}^2 N}{U^{3-p}} \right) ^{\frac{1}{4}}
.\] 
The $\tilde{g}$ is then written as
\[
	\tilde{g}_{\text{YM}}^2 (U) = \frac{1}{(2\pi\alpha')^2 T_p}
	\left( \frac{\alpha' U}{L}\right)^{p-3} 
.\] 
We see that if $L$ is kept fixed, the $\tilde{g}$ has the correct $U$ scaling.

Interestingly, one can change $U$ and $N$ simultaneously such that $L$ kept fixed.
This may provide a relation between the scaling of $N$ and the scaling of $U$.
In particular, $p=-1$ case, the field theory is a matrix model.
The only scale is the matrix size $N$.
In the 't Hooft limit, the effective coupling $\lambda \sim N^{-1}$.
With the restriction that $N \sim U^4$,
one could recover $\lambda \sim U^{-4}$.
