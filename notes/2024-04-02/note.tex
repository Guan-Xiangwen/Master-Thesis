%! Tex root: ../master.tex

The holographic dual between supergravity and QFT:
What's the QFT correspondence of the dilaton profile in the supergravity
solution?

The relation (2.21) in \pdfref{KST08} is non-trivial if the QFT is a matrix model
\begin{equation}\label{dilaton-coupling}
	\mathrm{e}^{\phi} = \frac{1}{N} c_d [g^2_{\text{eff}}(u)]^{\frac{7-p}{2(5-p)}}
\end{equation}
The reason is that the meaning of
``the effective dimensionless coupling constant'' $g_{\text{eff}}^2(E)$
\[
	g_{\text{eff}}^2(E) = g_d^2 N E^{p-3}
\] 
is not clear.
The following is the use of dual frame in the DBI action
\[
	S_{\text{dual}} = - \frac{1}{(2\pi)^{p-2}\alpha'^{(p-3) / 2}}
	\int \mathrm{d}^{p+1}x (N \mathrm{e}^\phi)^{\frac{2(p-5)}{7-p}}
	(N u^{p-3}) \frac{1}{4} \mathrm{Tr} F^2
.\] 
Note that one recovers the correct classical scaling of Yang-Mills theory from the AdS geometry $\sim N u^{p-3}$.
However, the dilaton profile does not play a significant role in determining the scaling of the field theory.
Look at \eqref{dilaton-coupling}, the power reads $p=3: \# = 1$,
$p=2: \# = 5 / 6$, $p=1: \# = 3 / 4$, $p=0: \# = 7 / 10$,
$p=-1: \# = 2 / 3$.
We are interested in the $p=-1$ case
\[
	N \mathrm{e}^\phi = c (g^2 N)^{\frac{2}{3}} u^{-\frac{8}{3}}
.\] 
This seems a modified version of
\begin{equation}
	g_{YM}^2 = (2\pi)^{p-2} g_s {\alpha'}^{(p-3) / 2}	
\end{equation}
if one identifies the $\mathrm{e}^\phi$ with $g_{\text{s}}$.
But it's not clear how to understand this modification.

What's the interpretation of the Weyl rescaling of metric from the QFT point of view?

What's the meaning of ``the geometry is conformal to AdS'' from the QFT point of view?

The running of coupling $g(\mu)$ in (maximally) super-Yang-Mills theory:
the classical scaling is protected by supersymmetry?
How this is reflected in the IKKT model?
To understand this point, I should be equipped with the correct method to deal with the fermionic integration.
Especially the technique to deal with the non-quadratic fermionic terms
in the matrix model integration.

In \pdfref{IKKT96}, a double scaling limit is discussed.
The limit is derived from $10\mathrm{d}$ super-Yang-Mills,
where the mass scale $m=g_0^{1 / 3}a$ is kept fixed.
This is the limit which the author expects to describe type IIB string theory.
What's the limit that we want in the holographic setting?

\begin{problem}
	The ultimate goal is to reproduce the dilaton profile of D-instanton solution from the IKKT matrix model.
	There are some sub-problems along this way
	\begin{itemize}
		\item What's the most natural coordinate to describe the dilaton profile $\mathrm{e}^{\phi}$? (gravity side)
		\item What's the RG flow of the IKKT matrix model? (matrix model side)
		\item One expects that there are different fixed points around different vacua, which have different scaling dimensions. (matrix model side)
		\item There are two sorts of effective action in \pdfref{IKKT96}:
			the effective potential for certain vacuum solution $W$;
			the effective action for the partition function $S_{\text{eff}}\equiv -\log (\frac{Z(n)}{Z_0(N)})$.
	\end{itemize}
\end{problem}

\newpage

Understand the fermionic integration in the IKKT model.

The Grassmann integration formula
\begin{equation}
	\int [\mathrm{d}\psi]
	\mathrm{e}^{\int \mathrm{d}^d x \psi_i \Delta_{ij} \psi_j}
	= \mathrm{Pf} \Delta
\end{equation}
We want to apply this formula to the following action
\[
\mathrm{Tr} \left( \overline{\varphi} \Gamma^\mu
[A_\mu,\varphi]\right) 
.\] 
To make the quadratic structure clear,
let's work in a particular basis.

In $10 \mathrm{d}$, spinor space is of $32$ (complex) dimension.
A particular basis can be constructed using ``creation and annihilation operators'' (I refer to Pol (Appendix B)):
\begin{align*}
	\Gamma^{0\pm} &= \frac{1}{2} (\pm \Gamma^0 + \Gamma^1)\\
	\Gamma^{a\pm} &= \frac{1}{2} (\Gamma^{2a} \pm i \Gamma^{2a+1}),\quad
	a=1,2,3,4
\end{align*}
Here $\Gamma^-$ is annihilation and $\Gamma^+$ is creation.
The representation is constructed as usual.
We will label the states as
\[
	\zeta^{(\mathbf{s})},\quad \mathbf{s}\equiv (s_0,s_1,s_2,s_3,s_4)
,\] 
with $s_a = \pm \frac{1}{2}$.
The highest weight state is $(-\frac{1}{2},-\frac{1}{2},-\frac{1}{2},-\frac{1}{2},-\frac{1}{2})$.

In the model, $\varphi$ is Majorana-Weyl spinor.
Let's impose the conditions.

For the Weyl condition, let's use the following chirality operator
\[
	\Gamma \equiv i^{-k} \Gamma^0 \Gamma^1 \cdots \Gamma^9
.\]
It turns out that $\Gamma$ is diagonal in the above basis,
with eigenvalues
\[
\Gamma = 2^5 s_0 s_1 s_2 s_3 s_4
.\] 
In the following, we will use the chirality $\Gamma = +1$.
The (complex) dimension then is $16$.

The complex conjugate spinor $\zeta^{(s)*}$ is defined by
the complex conjugate of Gamma-matrices $\Gamma^{\mu*}$.
Note that $\Gamma^{a\pm *}$ acting on $\zeta^{(s)*}$
is the same as $\Gamma^{a\pm}$ acting on $\zeta^{(s)}$
because the coefficients are real numbers.
However, this means that $\Gamma^{\mu*}$ will act differently.
In particular, $\zeta$ and $\zeta^*$ have different Lorentz transformation
property.

The Majorana condition 
\[
\zeta^* = B\zeta,\quad B = \Gamma^3 \Gamma^5 \Gamma^7 \Gamma^9
.\] 
This makes sense because $B^{-1}\zeta^*$ has the same Lorentz property as $\zeta$.
One can write
\[
	B = (\Gamma^{1+} - \Gamma^{1-})(\Gamma^{2+} - \Gamma^{2-})
	(\Gamma^{3+} - \Gamma^{3-})(\Gamma^{4+} - \Gamma^{4-})
.\] 
So it will flip $s_1,s_2,s_3,s_4$ with a sign depending on the number of fliping $+\to -$.
For example
\[
	B \zeta^{(+++++)} = (-1)^4 \zeta^{(+----)} = \zeta^{(+----)}
.\] 
We will also use the following identity
\[
B^* B = 1
.\] 

Let's define the Majorana basis
\begin{align*}
	\xi^{(1)} = \frac{1}{\sqrt{2}} (\zeta^{(+++++)} + B^{-1}\zeta^{(+++++)*}),\quad
	\xi^{(2)} = \frac{1}{\sqrt{2}} (\zeta^{(+++--)} + B^{-1}\zeta^{(+++--)*})
	\\
	\xi^{(3)} = \frac{1}{\sqrt{2}} (\zeta^{(++-+-)} + B^{-1}\zeta^{(++-+-)*}),\quad
	\xi^{(4)} = \frac{1}{\sqrt{2}} (\zeta^{(+-++-)} + B^{-1}\zeta^{(+-++-)*})
	\\
	\xi^{(5)} = \frac{1}{\sqrt{2}} (\zeta^{(-+++-)} + B^{-1}\zeta^{(-+++-)*}),\quad
	\xi^{(6)} = \frac{1}{\sqrt{2}} (\zeta^{(++--+)} + B^{-1}\zeta^{(++--+)*})
	\\
	\xi^{(7)} = \frac{1}{\sqrt{2}} (\zeta^{(+-+-+)} + B^{-1}\zeta^{(+-+-+)*}),\quad
	\xi^{(8)} = \frac{1}{\sqrt{2}} (\zeta^{(-++-+)} + B^{-1}\zeta^{(-++-+)*})
	\\
	\xi^{(9)} = \frac{1}{\sqrt{2}} (\zeta^{(+--++)} + B^{-1}\zeta^{(+--++)*}),\quad
	\xi^{(10)} = \frac{1}{\sqrt{2}} (\zeta^{(-+-++)} + B^{-1}\zeta^{(-+-++)*})
	\\
	\xi^{(11)} = \frac{1}{\sqrt{2}} (\zeta^{(--+++)} + B^{-1}\zeta^{(--+++)*}),\quad
	\xi^{(12)} = \frac{1}{\sqrt{2}} (\zeta^{(+----)} + B^{-1}\zeta^{(+----)*})
	\\
	\xi^{(13)} = \frac{1}{\sqrt{2}} (\zeta^{(-+---)} + B^{-1}\zeta^{(-+---)*}),\quad
	\xi^{(14)} = \frac{1}{\sqrt{2}} (\zeta^{(--+--)} + B^{-1}\zeta^{(--+--)*})
	\\
	\xi^{(15)} = \frac{1}{\sqrt{2}} (\zeta^{(---+-)} + B^{-1}\zeta^{(---+-)*}),\quad
	\xi^{(16)} = \frac{1}{\sqrt{2}} (\zeta^{(----+)} + B^{-1}\zeta^{(----+)*})
\end{align*}
For example let's check for $\xi^{(1)}$:
\[
	B \xi^{(1)} = \frac{1}{\sqrt{2}} 
	(B \zeta^{(+++++)} + \zeta^{(+++++)*})
	= \frac{1}{\sqrt{2}}
	(\zeta^{(+----)} + \zeta^{(+++++)*})
.\] 
\[
	(\xi^{(1)})^* = \frac{1}{\sqrt{2}}
	(\zeta^{(+++++)*} + (B^{-1})^* \zeta^{(+++++)})
	= \frac{1}{\sqrt{2}}
	(\zeta^{(+++++)*} + B \zeta^{(+++++)})
.\] 

Decompose $\varphi$ in the bove basis
\[
	\varphi = \sum_{k=1}^{16} c_{(k)} \xi^{(k)} 
\] 
where $c_{(k)}$ are real coefficients.
The real dimension is $16$.

The spinor product is constructed using the Majorana conjugation. 
It is defined as:
\[
	\overline{\varphi} \equiv \varphi^{\text{T}} C,\quad C\equiv B \Gamma^0
.\] 
We have
\[
C = \Gamma^0 \Gamma^3 \Gamma^5 \Gamma^7 \Gamma^9
.\] 
It acts on the spinor basis $\zeta^{(\mathbf{s})}$ will flip all spins $s_a$.

We also need to generalize $\varphi$ to the adjoint representation of $\mathfrak{u}(N)$.
\[
	\varphi = \varphi_{\mathfrak{a}} T^{\mathfrak{a}}
	= \left( \sum_{k} b_{(k),\mathfrak{a}} \xi^{(k)} + i c_{(k),\mathfrak{a}} \chi^{(k)}\right) T^{\mathfrak{a}},\quad \mathfrak{a} = 1,\cdots,N^2
.\] 
$T^{\mathfrak{a}}$ are Hermitian matrices.

It's useful to define
\[
B_a \equiv A_{2a} - i A_{2a+1},\quad a=1,2,3,4
\] 
and
\[
B_0 \equiv A_0 + A_1,\quad \overline{B}_0 \equiv -A_0 + A_1
.\] 
such that
\[
\Gamma^\mu A_\mu = \sum_{a=0}^4 (\Gamma^{a+}B_a + \Gamma^{a-}\overline{B}_a)
.\] 
