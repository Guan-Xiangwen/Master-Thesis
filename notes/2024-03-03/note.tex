%! Tex root: ../master.tex

To include supersymmetry in the matrix model,
we are forced to consider the Grassmann-valued matrix.
This leads us to think about how to build a matrix upon the Grassmann algebra.
In particular, think about how the supersymmetry transformation is realized as a transformation inside the Grassmann algerba (interchange odd and even elements).

First, one should difference between the elements of Grassmann algebra and the Grassmann variables.
The later are the generators of the Grassmann algebra.
The differential and integration are defined for the Grassmann variables.
Then

\begin{question}
	The fermionic matrix has entries of the Grassmann variables or the general odd elements of the Grassmann algebra?
\end{question}
It seems natural to allow general odd elements and take the number of the Grassmann variables to infinity.
\begin{question}
	How to define the integration over such a Grassmann-valued matrix?
\end{question}

The supersymmetry of a matrix model may imply the vanishing of some correlation functions.
This may be regarded as the matrix analog of the Ward identity.
An example is the superfield model mentioned in
\urlref{https://arxiv.org/abs/hep-th/9601041}{Ple96}.
The reason behind the vanishing is the cancellation between bosonic and fermionic degrees of freedom.
However, the prove given in the article based on the diagonalization of the bosonic matrix.
One would like to see the cancellation in a direct way:
how the cancellation works when calculating the effective action.
The first step would be a careful study of the action under a parameterization of the model.

One problem of the RG calculation is that
the susy transformation of the reduced matrix involving the variables that being integrated out.
This implies that the parameterization is not done in a way that the susy is preserved.
To avoid the problem, maybe we should use the formalism where the susy is realized linearly.

\begin{idea}
	Do the RG calculation for a model where the supersymmetry is realized linearly.
\end{idea}

The supersymmetry Ward identity gives a relation between correlation functions
\begin{equation}
	\frac{1}{2} \left<\mathrm{Tr} V'(\phi) \phi^{n-1} \right>
	= \sum_{a+b=n-2} \left<\mathrm{Tr} \phi^a \psi \phi^b \overline{\psi} \right>.
\end{equation}
This identity maybe used to reduce the result of the calculation.
\begin{problem}
Try to derive this Ward identity.	
\end{problem}
\begin{problem}
How does the measure $\mathrm{d}\phi \mathrm{d}\psi \mathrm{d}\overline{\psi}$ transforms under the susy?	
\end{problem}
First note that the super-determinant must involve $\overline{\epsilon}\epsilon$ such that the integral is bosonic.
So the measure is invariant to the first order of $\epsilon,\overline{\epsilon}$.
The ``superfield'' $\Phi = \phi + \overline{\psi}\theta + \overline{\theta}\psi + \theta \overline{\theta} F$ could be the starting point for deriving the Ward identity.
Calculate $\Phi^n$
\begin{align*}
	\Phi^n &= \left( \phi + \overline{\psi}\theta + \overline{\theta} \psi + \theta \overline{\theta} F \right)^n \\
		   &= \phi^n + \left( \sum_{a+b=n-1} \phi^a \overline{\psi} \phi^b \right) \theta + \overline{\theta} \left( \sum_{a+b=n-1} \phi^a \psi \phi^b\right)  \\
		   &+ \theta \overline{\theta}\left(\sum_{a+b+c=n-2} \phi^a \overline{\psi} \phi^b \psi \phi^c + \sum_{a+b=n-1}\phi^a F \phi^b\right).
\end{align*}
This leads us to consider the following reparameterization (induced by $\Phi^n$, this reparameterization should be supersymmetric.)
\begin{align*}
	\phi &\to \phi' = \phi + \varepsilon \phi^n \\	
	\overline{\psi} &\to \overline{\psi}' = \overline{\psi} + \varepsilon\sum_{a+b=n-1} \phi^a \overline{\psi} \phi^b \\
	\psi &\to \psi' = \psi + \varepsilon \sum_{a+b=n-1} \phi^a \psi \phi^b \\
	F &\to F' = F + \varepsilon\left( \sum_{a+b+c=n-2} \phi^a \overline{\psi} \phi^b \psi \phi^c + \sum_{a+b=n-1} \phi^a F \phi^b\right)
\end{align*}
Consider how the measure $\mathrm{d}\phi \mathrm{d}\overline{\psi} \mathrm{d}\psi \mathrm{d}F$ changes to the first order of $\varepsilon$.
The usual coordinate transformation formula $\mathrm{det} \mathrm{Jac}$ is generalized to that involving Grassmann variables $\mathrm{Ber} \mathrm{Jac}$.
The Berezinian satisfies the formula
\begin{equation}
	\mathrm{Ber} \left( \mathds{1} + \varepsilon M \right) = 1 + \varepsilon \mathrm{STr} M.	
\end{equation}
Here the $\mathrm{STr}$ is the super-trace.
It can be proved that there is no change of the measure.

Now the following terms arise from the variation of the action
\begin{align*}
	\varepsilon \mathrm{Tr}[V''(\phi)\phi^n F]-2 \varepsilon \sum_{a+b=n-1} \mathrm{Tr} (\phi^a F \phi^b F) + \varepsilon \sum_{a+b=n-1} \mathrm{Tr}[V'(\phi)\phi^a F \phi^b] \\
	-2 \varepsilon \sum_{a+b+c=n-2} \mathrm{Tr} (\phi^a \overline{\psi} \phi^b \psi \phi^c F) + \varepsilon \sum_{a+b+c=n-2} \mathrm{Tr} [V'(\phi) \phi^a \overline{\psi} \phi^b \psi \phi^c] \\
	+\varepsilon \sum_{k=0}^\infty k g_k \sum_{a+b=k-2} \Big[ a \mathrm{Tr}(\phi^{a+n-1}\overline{\psi}\phi^b\psi) + b \mathrm{Tr}(\phi^a \overline{\psi}\phi^{b+n-1}\psi)   \\
	+ 2 \sum_{c+d=n-1} \mathrm{Tr}(\phi^{a+c}\overline{\psi}\phi^{b+d}\psi)\Big]
\end{align*}
The second line will vanish after integrating over $F$.
The first line will give $-2\varepsilon \sum_{a+b=n-1} \mathrm{Tr}\phi^a \mathrm{Tr}\phi^b + \frac{\varepsilon}{2} \mathrm{Tr}[V''(\phi)\phi^n V'(\phi)]$.
\begin{remark}
The above calculation gives a fairly complicate identity.
It's also not clear whether it will be useful or not.
Also, this does not lead to the Ward identity.
\end{remark}
