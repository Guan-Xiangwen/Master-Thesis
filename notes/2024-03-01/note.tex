%! Tex root: ../master.tex

It's not clear that what's the meaning of conformal invariance for a matrix model, but we have the motivation to believe that there exists such a notion.
From the point view of AdS/CFT, the matching between the conformal group,
for example in $4d$ the conformal group is $SO(4,2)$, and the AdS isometry group, for example in $5d$ the isometry group is $O(4,2)$.
However, such a ``group argument'' does not make sense for AdS1/CFT0.

Let's review that how the AdS/CFT is understood in string theory.
The central object is the D-brane whose dynamics can be described by an effective action of field theory.
The fields live on the world-volume of D-brane: there are massless scalar fields and vector fields.
Their appearance is understood as the massless excitation of open strings with end points on the D-brane.
The form of the action is obtained by matching with the calculation from the perturbative string theory.
This field theory will become the CFT side.

\question{
But why the field theory of D-branes must be a CFT? Does this fact relate to the dynamics of the D-branes? Or is there a ``string theory'' argument for the appearance of conformal invariance?
}

\answer{
	Because the open string excitation is massless when the two end points locate on the coincident D-branes.
}

D-branes carry the R-R charges, 
which means that it will couple to the higher form gauge fields.
Therefore they can be understood physically 
as the sources of those gauge fields.
The D(-1)-brane couples to the 0-form potential in the IIB theory $C_0$.
The energy of that coupling is given by the value of $C_0$ at the point where the D(-1)-brane locating.
Therefore the D(-1)-brane is properly interpreted as an instanton.

We are interested in the D-instanton,
a natural question is that how the D-instantons interact with themselves
and other objects like higher dimensional D-branes.
This question could be answered in the perturbative string theory.
One calculates the string amplitudes for exchanging the closed string states (graviton, dilaton and the R-R state) between the D-branes.
The fundamental physical properties of the D-branes 
(the tension and the R-R charge)
could be related to the fundamental constants in string theory.

\question{
	When discussing the interaction and action of D-branes, we have in mind a picture where the D-brane locating at a particular space-time position.
	In this way, rather than taking them as a fundamental degrees of freedom, we regard them as absolute objects (like black hole in GR).
	In such a setting, how should we understand the physics under the D-brane action?
	A clearer name seems to be the ``action of D-brane fluctuation'',
	and the fluctuation has its origin at the open string oscillation.
	Or maybe one should call it the ``open string action restricted on the D-branes''?
}

There is an interesting explanation of the collective coordinates $X^\mu$ of the D-branes: 
the Goldstone bosons for the spontaneous translation symmetry breaking.
But why it's called the ``spontaneous''?

\idea{
	Why not trying to look further at the ``Ward identity'' of the matrix model and think about the possible implications on our calculations?
}
