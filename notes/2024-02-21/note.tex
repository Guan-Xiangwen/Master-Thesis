%! Tex root: ../master.tex

\paragraph{rethink the model}
In constructing the previous model, the way we introduce the matrix $\eta$ is suspicious.
Specifically $\delta^2 \eta = 0$, but it should generate the unitary transformation $\delta^2 \eta = [H,\eta]$.
Remember that the idea behind the BRST construction is ``taking the square root of gauge transformation''.
However, it seems impossible to satisfy $\delta^2 H = 0$ simultaneously.
It seems like the only way to solve this problem is to introduce again an auxillary matrix such that $\delta H = [X,\eta]$.
That is the gauge transformation is parameterized by the auxillary matrix $X$.

What's the motivation to introduce $\eta$ in our model?
I want to introduce a term $\mathrm{Tr}([\overline{Z},Z]H)$ in the action,
from doing the variation $\delta(\cdots)$.
To obtaine $H$ from $\delta$, I introduce $\eta$ such that $\delta \eta = H$.
This is a bad move as having been said above: this breaks the unitary invariance when the model involving the matrix $\eta$.

\paragraph{one possible solution to the previous problem?}
If we introduce another auxillary matrix $X$ to parameterize the unitary transformation,
and modify $\delta H = [X,\eta]$ with $\delta X = 0$. (Remark. $H,X,\eta$ should all be anti-hermitian to keep each term ``real'' under the hermitian conjugation $\dagger$)
The model build may still start from the commutator term
\[
	- \frac{1}{2} \mathrm{Tr}H^2 + ig \mathrm{Tr}\left([\overline{Z},Z]H\right)
.\] 
Then one wants to find $\delta$-exact terms to generate these two terms.
The candidates are
\begin{align*}
	i g \delta \mathrm{Tr}\left([\overline{Z},Z]\eta\right) - \frac{1}{2} \delta \mathrm{Tr}\left(\eta H\right)
\end{align*}
Other terms are generated along
\[
	i g \mathrm{Tr}\left([\overline{\Psi},Z]\eta + [\overline{Z},\Psi]\eta\right) + \frac{1}{2} \mathrm{Tr}\left(\eta [X,\eta]\right)
.\] 
Note that $(1 / 2)\mathrm{Tr}(\eta [X, \eta]) = \mathrm{Tr} (\eta X \eta)$ gives a quadratic term of $\eta$.
To deal with the interaction terms $ig\mathrm{Tr}([\overline{Z},Z]H)$ and 
$ig \mathrm{Tr}([\overline{\Psi},Z]\eta + [\overline{Z},\Psi]\eta)$,
it's important to introduce quadratic term for $Z,\overline{Z}$ and $\Psi,\overline{\Psi}$.
This is done by adding to following $\delta$-exact term
\[
	i \kappa \delta \mathrm{Tr}\left(\overline{Z}\Psi - Z \overline{\Psi}\right) = 2i \kappa \mathrm{Tr}(\overline{\Psi}\Psi) - 2\kappa \epsilon \mathrm{Tr}(\overline{Z}Z) - 2i\kappa \mathrm{Tr}\left([\overline{Z},Z]X\right)
.\] 
Here $X$ is not a dynamical matrix, and it should be anti-hermitian.
In summary, let's collect all terms in the action
\begin{align*}
	- \frac{1}{2} \mathrm{Tr} H^2 + \mathrm{Tr} \eta X \eta \\
	+ \frac{i}{2} \kappa \mathrm{Tr} \overline{\Psi}\Psi - \frac{1}{2} \kappa \epsilon \mathrm{Tr} \overline{Z} Z - \frac{i}{4} \kappa \mathrm{Tr}\left([\overline{Z},Z]X\right) \\
	+ i g \mathrm{Tr} \left([\overline{Z},Z]H\right) + i g \mathrm{Tr}\left([\overline{\Psi},Z]\eta + [\overline{Z},\Psi]\eta\right)
\end{align*}
It can be verified all the terms are real under $\dagger$ (It acts on the matrix, will not take $i\to -i$).

Compare to previous wrong model, it has several differences.
First is the appearance of the $X$ matrix.
It it is taken to be zero, the model will become very similar to what we have studied previously,
but with an additional term $ ig \mathrm{Tr}\left([\overline{Z},Z]H\right) $.
However, the previous calculation is wrong because $\eta$ should be anti-hermitian, rather than hermitian.
This means that there is no $\theta$ variable to be integrated out;
also $h^*$ and $h$ terms should have a relative minus sign.
Let's list all terms that are relavent to the integration
\begin{align*}
	- v^* A \beta + \beta^* A v + v^* B \alpha - \alpha^* B v + \alpha^* H \beta - \beta^* H \alpha \\
	x^* A h - h^* A x - y^* B h + h^* B y \\
	\beta^* \psi h - h^* \psi \beta - \alpha^* \chi h + h^* \chi \alpha \\
	- \alpha^* \eta x - x^* \eta \alpha + \beta^* \eta y + y^* \eta \beta
\end{align*}

It's interesting to take $X = i z \mathds{1}$.
$z$ is a real constant to be determined later.
One consequence is
\[
	\mathrm{Tr}\left(\eta X \eta\right) \sim - i z h_i^* h_i
.\] 
Remember that $\eta_{(N+1),i} = -h_i^*,~ \eta_{i,(N+1)} = h_i$.
Another consequence is
\[
	\frac{\kappa}{2} \mathrm{Tr}\left([A,B]X\right) \sim \frac{\kappa}{2} iz (\beta_i^* \alpha_i - \alpha_i^* \beta_i)
.\] 
So the quadratic term for $\alpha,\beta$ is
\[
 - \frac{\kappa \epsilon}{2} (\alpha^*_i \alpha_i + \beta_i^* \beta_i) 
 + \frac{i \kappa z}{2} (\beta_i^*\alpha_i - \alpha_i^* \beta_i)
.\] 
This motivates us to redefine
\[
	\tilde{\alpha} = \frac{1}{\sqrt{2}} (\alpha + i\beta),\quad 
	\tilde{\beta} = \frac{1}{\sqrt{2}} (\alpha - i\beta)
.\] 
Then the quadratic term becomes
\[
	- \frac{\kappa}{2} \left[(\epsilon + z) \tilde{\alpha}^*_i \tilde{\alpha}_i 
	+ (\epsilon - z) \tilde{\beta}^*_i \tilde{\beta}_i\right]
.\] 

The interaction of $A,B$ with the ``bosonic legs'' in terms of $\tilde{\alpha},\tilde{\beta}$ reads
\[
	\frac{1}{\sqrt{2}} \left( i v^* A \tilde{\alpha} + i \tilde{\alpha}^* A v
	- i v^* A \tilde{\beta} - i \tilde{\beta}^* A v
+ v^* B \tilde{\alpha} - \tilde{\alpha}^* B v
+ v^* B \tilde{\beta} - \tilde{\beta}^* B v\right)
.\] 
It's possible to obtain a new operator $\mathrm{Tr}(AB)$ through the contraction.
To the lowest order there are four terms that contribute to $\mathrm{Tr}(AB)$:
\[
	\frac{i}{2}\left(\tilde{\alpha}^* A v v^* B \tilde{\alpha}
	- v^* A \tilde{\alpha} \tilde{\alpha}^* B v
+v^* A \tilde{\beta} \tilde{\beta}^* B v
- \tilde{\beta}^* A v v^* B \tilde{\beta}\right)
.\] 
We see that there is actually no $\mathrm{Tr}(AB)$ generated at this order.
For the interaction of $A,B$ with the ``fermionic legs''
\[
x^* A h - h^* A x - y^* B h + h^* B y
.\] 
There are two terms that contribute to the $\mathrm{Tr}(AB)$
\[
	x^* A h h^* B y + h^* A x y^* B h
.\] 
The order of contraction is $y x^*$ and $y^* x$.
So there is also no contribution at this order.

(TODO: more discussions on the propagators $\alpha^* \alpha$ and $\beta^* \beta$, and possible cancellation or suppression of new interaction generations.)
