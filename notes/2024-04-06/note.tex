% Tex root: ../master.tex

Calculate $\Gamma^{a\pm} \xi^{(k)}$:
\begin{equation*}
	\Gamma^{a\pm} \xi^{(k)} = \frac{1}{\sqrt{2}} \Gamma^{a\pm}
	\left( \zeta^{(\mathbf{s})} + B^{-1} \zeta^{(\mathbf{s})*} \right) 
	=\frac{1}{\sqrt{2}} \Gamma^{a\pm}
	\left( \zeta^{(\mathbf{s})} + B^{-1}(B\Gamma^{a\pm}B^{-1}) \zeta^{(\mathbf{s})*} \right) 
\end{equation*}
\[
B\Gamma^{a\pm}B^{-1} = (\Gamma^{a\mp})^*
.\]
\begin{equation}
	\Gamma^{a\pm} \xi^{(k)}  = \frac{1}{\sqrt{2}} 	\left(\Gamma^{a\pm}
 \zeta^{(\mathbf{s})} + B^{-1} (\Gamma^{a\mp}\zeta^{(\mathbf{s})})^* \right) 
\end{equation}
Note that only one of the two terms is non-vanishing.
Also, these two terms have different Lorentz transformations,
it's consistent to add them together only if one of them is zero.
The spinor product
$
	\overline{\xi^{(l)}} \Gamma^{a\pm} \xi^{(k)} \neq 0
$
only if $\xi^{(l)}$ has the opposite Lorentz transformation.

\paragraph{anti-Majorana}
Just like imagniary number, an anti-Majorana spinor is defined naturally as
\[
B \chi  =  - \chi^*
.\] 
Note that $i\chi$ is just a Majorana spinor.
It's easy to construct an anti-Majorana basis as
\[
	\chi^{(k)} = \frac{1}{\sqrt{2}} (\zeta^{(\mathbf{s})}
	- B^{-1} \zeta^{(\mathbf{s})*})
.\] 
Is it true that
\[
\frac{i}{\sqrt{2}} (\zeta^{(\mathbf{s})}
	- B^{-1} \zeta^{(\mathbf{s})*})
 = \frac{1}{\sqrt{2}} (\zeta^{(\mathbf{s})}
	+ B^{-1} \zeta^{(\mathbf{s})*})
.\] 
It can be proved that $\overline{\chi}\xi$ is a pure imaginary number
(with $\chi$ anti-Majorana, $\xi$ Majorana)
\[
	\overline{\chi}\xi = \overline{\chi} B^{-1}\xi^* 
	= \chi^{\text{T}} C B^{-1} \xi^*
.\] 
One can proceed by using
\[
	C B^{-1} = B^{-1} B C B^{-1} = B^{-1} B (B\Gamma^0) B^{-1}
	= B \Gamma^0 B^{-1} = (\Gamma^0)^*
.\]
So (the Dirac conj. is the minus of the Majorana conj. for $\chi$)
\[
\overline{\chi}\xi = (\chi^\dagger \Gamma^0 \xi)^*
= - (\overline{\chi}\xi)^*
.\] 
