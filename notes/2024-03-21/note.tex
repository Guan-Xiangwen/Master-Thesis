%! Tex root: ../master.tex

\begin{todo}
	Learn something about ``membrane regularization''.
\end{todo}

%%%%%%%%%%%%%%%%%%%%%%%%%%
%  Green-Schwarz Action  %
%%%%%%%%%%%%%%%%%%%%%%%%%%


In \pdfref{IKKT96},
the matrix model (IKKT model) is proposed as a regularization of
the Green-Schwarz type IIB superstring.

The Green-Schwarz action \pdfref{GS84} used in \pdfref{IKKT96} is
\begin{align}\label{GS-IKKT96}
	S_{\text{GS}} = -T \int \mathrm{d}^2\sigma
	\sqrt{-\frac{1}{2}\Sigma^2}
	+ i \epsilon^{ab} \partial_a X^\mu
	\left( \overline{\theta}^1 \Gamma_\mu \partial_b \theta^1 
	+ \overline{\theta}^2 \Gamma_{\mu} \partial_b \theta^2\right) \notag\\
	+ \epsilon^{ab} (\overline{\theta}^1 \Gamma^\mu \partial_a \theta^1)
	(\overline{\theta}^2\Gamma_\mu\partial_b\theta^2)
\end{align}
with
\begin{align*}
	\Sigma^{\mu\nu} &= \epsilon^{ab} \Pi^\mu_a \Pi^\nu_b,\\
	\Pi^\mu_a &= \partial_a X^\mu - i \overline{\theta}^1 \Gamma^\mu \partial_a \theta^1 + i \overline{\theta}^2 \Gamma^\mu \partial_a \theta^2
\end{align*}
Here $\theta^1,\theta^2$ is understood as $10 \mathrm{d}$
Majorana-Weyl spinor with the same chirality.

%%%%%%%%%%%%
%  Remark  %
%%%%%%%%%%%%

\eqref{GS-IKKT96} differs from the one proposed in \pdfref{GS84}
by a redefinition $\theta^2\to i\theta^2$.

\eqref{GS-IKKT96} uses the Nambu-Goto form $\sqrt{-\frac{1}{2}\Sigma^2}$;
\pdfref{GS84} uses the Polyakov form $\frac{1}{2}\sqrt{-g}g^{\alpha\beta}
\Pi^\mu_{\alpha}\Pi_{\mu\beta}$.

\begin{question}
	I don't understand how this action is derived in \pdfref{GS84}.
	The $\kappa$ symmetry seems play an important role there.
	The motivation of \pdfref{GS84} is to derive a covariant action
	for superstring theory (compare to the light-cone gauge).
	I don't understand the idea of $\theta^2\to i\theta^2$
	used in \pdfref{IKKT96}.
	It is said that this prescription will give the correct supersymmetry
	algebra.
\end{question}

%%%%%%%%%%%%%%%%%%%%%%%%%%%%%%
%  Space-time Supersymmetry  %
%%%%%%%%%%%%%%%%%%%%%%%%%%%%%%

There is a $\mathcal{N}=2$ space-time supersymmetry
\begin{align}
	\delta_{\text{SUSY}} \theta^1 &= \epsilon^1,\notag\\
	\delta_{\text{SUSY}} \theta^2 &= \epsilon^2,\notag\\
	\delta_{\text{SUSY}} X^\mu &= i \overline{\epsilon}^1 \Gamma^\mu \theta^1
- i \overline{\epsilon}^2 \Gamma^\mu \theta^2.
\end{align}

\begin{align*}
	\delta_{\text{SUSY}} \Pi_{a}^\mu = \delta_{\text{SUSY}}
	\left( \partial_a X^\mu - i \overline{\theta}^1 \Gamma^\mu \partial_a\theta^1 + i \overline{\theta}^2 \Gamma^\mu \partial_a \theta^2\right) = 0
\end{align*}

\begin{gather*}
	\delta_{\text{SUSY}}
	\left[ i \epsilon^{ab} \partial_a X^\mu
	(\overline{\theta}^1 \Gamma_\mu \partial_b \theta^1
+ \overline{\theta}^2 \Gamma_\mu \partial_b \theta^2)
+ \epsilon^{ab} (\overline{\theta}^1 \Gamma^\mu \partial_a \theta^1)
(\overline{\theta}^2 \Gamma_\mu \partial_b\theta^2)\right] \\
= - \epsilon^{ab}
( \overline{\epsilon}^1 \Gamma^\mu \partial_a \theta^1
-  \overline{\epsilon}^2 \Gamma^\mu \partial_a \theta^2)
	(\overline{\theta}^1 \Gamma_\mu \partial_b \theta^1
+ \overline{\theta}^2 \Gamma_\mu \partial_b \theta^2)\\
+i \epsilon^{ab} \partial_a X^\mu
	(\overline{\epsilon}^1 \Gamma_\mu \partial_b \theta^1
+ \overline{\epsilon}^2 \Gamma_\mu \partial_b \theta^2) \\
+ \epsilon^{ab} \left[(\overline{\epsilon}^1 \Gamma^\mu \partial_a \theta^1)
(\overline{\theta}^2 \Gamma_\mu \partial_b\theta^2) + 
(\overline{\theta}^1 \Gamma^\mu \partial_a \theta^1)
(\overline{\epsilon}^2 \Gamma_\mu \partial_b\theta^2 )\right] \\
= -\epsilon^{ab} (\overline{\epsilon}^1\Gamma^\mu\partial_a\theta^1)
(\overline{\theta}^1\Gamma_\mu\partial_b\theta^1)
+\epsilon^{ab} (\overline{\epsilon}^2\Gamma^\mu\partial_a\theta^2)
(\overline{\theta}^2 \Gamma_\mu\partial_b\theta^2) \\
+i \epsilon^{ab} \partial_a X^\mu
	(\overline{\epsilon}^1 \Gamma_\mu \partial_b \theta^1
+ \overline{\epsilon}^2 \Gamma_\mu \partial_b \theta^2) 
\end{gather*}

An identity mentioned in \pdfref{GS84}
\[
\gamma_\mu \psi_1 \overline{\psi}_2 \gamma^\mu \psi_3
+ \gamma_\mu \psi_2 \overline{\psi}_3 \gamma^\mu \psi_1
+ \gamma_\mu \psi_3 \overline{\psi}_1 \gamma^\mu \psi_2
=0
.\] 
Apply it
\[
- (\overline{\epsilon}^1 \Gamma^\mu \partial_a \theta^1)
(\overline{\theta}^1 \Gamma_\mu \partial_b \theta^1)
= (\overline{\epsilon}^1 \Gamma^\mu \theta^1)
(\partial_b \overline{\theta}^1 \Gamma_\mu \partial_a \theta^1)
+ (\overline{\epsilon}^1 \Gamma^\mu \partial_b \theta^1)
(\partial_a \overline{\theta}^1 \Gamma_\mu \theta^1)
.\] 
Play with the last term:
\[
	\epsilon^{ab}(\overline{\epsilon}^1 \Gamma^\mu \partial_b \theta^1)
	(\partial_a \overline{\theta}^1 \Gamma_\mu \theta^1)
	= -\epsilon^{ab}(\overline{\epsilon}^1 \Gamma^\mu \partial_b\theta^1)
	(\overline{\theta}^1 \Gamma_\mu \partial_a \theta)
	= \epsilon^{ab}(\overline{\epsilon}^1 \Gamma^\mu \partial_a \theta^1)
	(\overline{\theta}^1 \Gamma_\mu \partial_b \theta)
.\] 
This gives back to the first term!
So we have
\[
	\epsilon^{ab} (\overline{\epsilon}^1 \Gamma^\mu \partial_a \theta^1)
	(\overline{\theta}^1 \Gamma_\mu \partial_b \theta^1)
	=  \frac{1}{2} \epsilon^{ab} (\overline{\epsilon}^1 \Gamma^\mu \theta^1)
	(\partial_a \overline{\theta}^1 \Gamma_\mu \partial_b \theta^1)
.\] 
This further implies
\[
	\epsilon^{ab} (\overline{\epsilon}^1 \Gamma^\mu \partial_a \theta^1)
	(\overline{\theta}^1 \Gamma_\mu \partial_b \theta^1)
	=
	\frac{1}{3}\partial_a \left[ \epsilon^{ab} (\overline{\epsilon}^1
	\Gamma^\mu \theta^1) (\overline{\theta}^1 \Gamma_\mu
\partial_b \theta^1)\right]
.\] 
Conclusion: $\delta_{\text{SUSY}}$ gives a total derivative term.

%%%%%%%%%%%%%%%%%%%%
%  kappa-symmetry  %
%%%%%%%%%%%%%%%%%%%%

In \pdfref{IKKT96},
the $\kappa$-symmetry is given
\begin{align*}
	\delta_{\kappa} \theta^1 &= \alpha^1,\\
	\delta_{\kappa} \theta^2 &= \alpha^2,\\ 
	\delta_{\kappa} X^\mu &= i \overline{\theta}^1 \Gamma^\mu \alpha^1
	- i \overline{\theta}^2 \Gamma^\mu \alpha^2.
\end{align*}
with
\[
	\alpha^1 = (1 + \tilde{\Gamma}) \kappa^1,\quad
	\alpha^2 = (1 - \tilde{\Gamma}) \kappa^2
.\] 
$\kappa^{1,2}$ are local fermionic parameters.
The $\tilde{\Gamma}$ is
\[
	\tilde{\Gamma} = \frac{1}{2\sqrt{-\frac{1}{2}\Sigma^2}} \Sigma_{\mu\nu}
	\Gamma^{\mu\nu}
.\] 

\[
	\tilde{\Gamma}^2 = - \frac{1}{2\Sigma^2}
	\Sigma_{\mu\nu} \Sigma_{\rho\sigma}
	\Gamma^{\mu\nu} \Gamma^{\rho\sigma}
.\] 

\[
	\Gamma^{\mu\nu} \Gamma_{\rho\sigma}
	= \Gamma^{\mu\nu}_{~~~\rho\sigma}
	+ 4 \Gamma^{[\mu}_{~~[\sigma} \delta^{\nu]}_{~~\rho]}
	+ 2 \delta^{[\mu}_{~~[\sigma} \delta^{\nu]}_{~~\rho]}
.\] 

\[
	\tilde{\Gamma}^2 = -\frac{1}{2\Sigma^2}
	\left( \Sigma_{\mu\nu}\Sigma_{\rho\sigma}\Gamma^{\mu\nu\rho\sigma} 
	+ 4 \Sigma_{\mu\rho}\Sigma^\rho_{~\nu}\Gamma^{\mu\nu}
+ 2 \Sigma_{\mu\nu}\Sigma^{\nu\mu}\right) 
.\] 

\[
	\tilde{\Gamma}^2 \stackrel{?}{=} 1
.\] 

\[
	\Sigma_{\mu\rho}\Sigma^\rho_{~\nu}\Gamma^{\mu\nu}
	\stackrel{?}{=} 0
.\] 
