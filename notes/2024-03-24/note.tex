%! Tex root: ../master.tex

Clarify the near horizon geometry of the extremal $p$-brane solution
and the energy scaling of the Yang-Mills coupling.

The metric of the solution:
\begin{equation}
	\mathrm{d}s^2 = H_p^{-\frac{1}{2}} \left( - \mathrm{d}t^2
	+ \sum_{i=1}^p \mathrm{d} x_i^2\right) 
	+ H_p^{\frac{1}{2}} \left( \mathrm{d}r^2
	+ r^2 \mathrm{d}\Omega_{8-p}^2\right) .
\end{equation}
where
\[
	H_p(r) = 1 + \frac{Q}{r^{7-p}}
.\] 
The $r\to 0$ limit:
\begin{equation}
	\mathrm{d}s^2 \to Q^{-\frac{1}{2}} r^{\frac{7-p}{2}}
	\left( - \mathrm{d}t^2 + \sum_{i=1}^p \mathrm{d}x_i^2 \right) 
	+ Q^{\frac{1}{2}} r^{-\frac{7-p}{2}} 
	\left( \mathrm{d}r^2 + r^2 \mathrm{d}\Omega_{8-p}^2 \right) 
\end{equation}

When $p=3$, the $(t,r,x_i)$ part of the metric is the same as
the $\mathrm{AdS}_5$ metric in the Poincare coordinate;
$Q$ controls the $\mathrm{AdS}$ radius $\alpha$.
\[
\mathrm{d}s^2 = - \frac{r^2}{\alpha^2}\mathrm{d}t^2
+ \frac{\alpha^2}{r^2} \mathrm{d}r^2
+ \frac{r^2}{\alpha^2} \mathrm{d}\vec{x}^2
.\] 

Define a length $L$ as
\[
	\left( \frac{r}{L} \right)^2  \equiv Q^{-\frac{1}{2}} r^{\frac{7-p}{2}}
.\] 

How the Yang-Mills theory looks like in the $r\to 0$ geometry?
The leading order D$p$-brane action is the Yang-Mills theory we are talking about.

The bosonic DBI action for a single D$p$-brane [Pol (8.7.2)]:
introduce coordinates $\xi^a,a=0,\cdots,p$ on the brane.
The fields on the brane are the embedding coordinates $X^\mu(\xi)$
and the gauge fields $A_a(\xi)$.
The action for these fields is
\begin{equation}
	S_{p} = - T_p \int \mathrm{d}^{p+1} \xi \mathrm{e}^{-\Phi}
	\left[ - \mathrm{det} (G_{ab} + B_{ab} + 2\pi\alpha' F_{ab}) \right] ^{1 / 2}
\end{equation}
$T_p$ is related to the D$p$-brane tension,
$G_{ab}(\xi)$ and $B_{ab}(\xi)$ are the induced fields on the brane
\[
	G_{ab}(\xi) = \frac{\partial X^\mu}{\partial \xi^a} \frac{\partial X^\nu}{\partial \xi^b} G_{\mu\nu}(X(\xi)),\quad
	B_{ab}(\xi) = \frac{\partial X^\mu}{\partial \xi^a}
	\frac{\partial X^\nu}{\partial \xi^b} B_{\mu\nu}(X(\xi))
.\] 

There is a T-duality argument for the appearance of $F_{ab}$.

\[
	\epsilon^{a_0\cdots a_p} = \pm 1,\quad \epsilon^{01\cdots p}=1
.\] 
Expand the determinant (set $B_{ab}=0$)
\[
	\mathrm{det} (G_{ab} + 2\pi\alpha' F_{ab})
	= \frac{1}{(p+1)!}\epsilon^{a_0\cdots a_p}\epsilon^{b_0\cdots b_p}
	(G_{a_0 b_0} + 2\pi\alpha' F_{a_0 b_0})
	\cdots
	(G_{a_p b_p} + 2\pi\alpha' F_{a_p b_p})
.\] 
Be careful: $\epsilon^{a_0\cdots a_p}$ is symbol, not tensor.
\[
	\mathrm{det}(G_{ab} + 2\pi\alpha' F_{ab})
	= \mathrm{det}G \left( 1 + \frac{1}{2} (2\pi\alpha')^2 F^{ab}F_{ab}+ O(\alpha'^3)
 \right).\] 
The first order DBI action gives
\begin{equation}
	S_p = - T_p \int (\mathrm{d}^{p+1}\xi )\mathrm{e}^{-\Phi}
	\sqrt{-G} \left( 1 + (2\pi\alpha')^2 \frac{1}{4}F^{ab}F_{ab} \right) 
	+ O(\alpha'^3)
\end{equation}
The Yang-Mills action is
\begin{equation}
	S_{\text{YM}} = - T_p (2\pi\alpha')^2\int (\mathrm{d}^{p+1}\xi)
	\mathrm{e}^{-\Phi}\sqrt{-G} \frac{1}{4}F^{ab}F_{ab}.
\end{equation}

Identify the Yang-Mills coupling with the D$p$-brane tension
for a constant dilaton field $\Phi(r) = \Phi_0$:
\[
	g_{\text{YM}}^2 = \frac{1}{(2\pi\alpha')^2\tau_p},\quad \tau_p \equiv
	T_p \mathrm{e}^{-\Phi_0}
.\] 

The extremal D$p$-brane dilaton profile: $\Phi(r) \equiv \Phi_0 + \tilde{\Phi}(r)$:
\begin{equation}
	\mathrm{e}^{-\tilde{\Phi}(r)} = H_p(r)^{\frac{p-3}{4}}
\end{equation}
where \[
	H_p(r) = 1 + \frac{Q}{r^{7-p}}
.\] 
We are interested in $r\to 0$ limit (near horizon)
\begin{align*}
	S_{\text{YM}} = -T_p (2\pi\alpha')^2 \mathrm{e}^{-\Phi(r)}
	\int \mathrm{d}^{p+1}\xi \sqrt{-G} \left( \frac{1}{4}F^{ab}F_{ab} \right) 
	\\
	\to -T_p (2\pi\alpha')^2 \mathrm{e}^{-\tilde{\Phi}(r)}
	\int \mathrm{d}^{p+1}\xi \sqrt{-G} \left( \frac{1}{4}F^{ab}F_{ab} \right) 
\end{align*}
This leads to the following identification
\begin{equation}
	g_{\text{YM}}^2 = \frac{1}{(2\pi\alpha')^2 T_p}
	\left( \frac{Q}{r^{7-p}} \right)^{\frac{3-p}{4}}
\end{equation}

\begin{equation}
	\left( \frac{Q}{r^{7-p}} \right)^{-\frac{1}{2}} \equiv \left( \frac{r}{L} \right)^2
\end{equation}
The ``AdS radius'' $L$ depends on $r$.
$Q$ is independent of $r$
\begin{equation*}
	Q  = d_p (2\pi)^{p-2} \mathrm{e}^{\Phi_0} N {\alpha'}^{\frac{7-p}{2}}
\end{equation*}

The only way to recover the field theory scaling is to use the following identification
\[
	g_{\text{YM}}^2 = \frac{1}{(2\pi\alpha')^2 T_p}
	\left( \frac{r}{L} \right)^{D-4}
.\] 
The dimension $[g_{\text{YM}}^2] = [L^{D-4}]$.

No idea of how to interpret it.
