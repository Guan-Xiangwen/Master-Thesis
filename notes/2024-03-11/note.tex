%! Tex root: ../master.tex
\begin{todo}
Get some ideas on the ``emergent geometry'' from the IKKT model.	
\end{todo}

The simplest example is
the semicircle law of the Gaussian matrix model.
If one take the eigenvalue distribution $\rho(\lambda)$
as the distribution of ``points'' that constitute the geometry.

\begin{question}
Another notion (closely related) of emergent geometry arises from
the continuum limit of the matrix model,
which is not simply $N\to\infty$ but also requiring the ``free energy''
develop scaling behavior near certain critical point.

Clarfiy this.
\end{question}

% the research statement:  <11-03-24, Xiangwen> %

%The main question to ask when writing the research statement is:
%what kind of question you are willing to think about?

%There are some general questions that stay with you,
%which has not been answered in a clear way.
%That's not necessarily a deep question,
%but may be just some ``fussy and nitpicky'' questions.
%Or just some superficial questions that are due to the lack of thinking.
%Or just for fun.

%When looking through the researches,
%keep in mind the following idea:
%Is there any possible connections with what you are familiar with?

% Manuela (Trinity College, Dublin):  <11-03-24, Xiangwen> %

%Keywords (maybe useful to think about)
%- the structure of CFT in general dimensions.
%- AdS/CFT

%I appreciate the significance of mathematical physics.
%Although the problems may not interest a general physicist,
%scrutinizing and formalizing the ideas from physics 
%in a mathematical way has its own value.
%One reason is that sometimes it's hard to untangle math and physics.

%One area where the math and physics meet together a lot
%is the conformal field theory.
%To me, it seems like a subject where the simplicity and complicity meet together, physics and mathematical ideas mix together.
%One thing that CFT attracts me a lot is its universality:


%In any case, it's worth to understand the general structure of CFT better.
%It's of course also interesting that CFT appears in the setting of string theory in a geometric background.
%This somewhat unexpected appearance could show some natures of gravity
%or providing a new way to understand the classes of CFT.
%There are something in CFT that I definitely want to learn more.
%For example, it seems like that the OPE encodes important information about the field theory operators,
%but I don't have a good understanding on why and how it holds.
%
%I like to do researches on the topic like CFT.
%CFT, although being a special case of QFT, providing insights that can not be obtained otherwise.
%For example, it's hard to learn the vacuum structure of a general QFT,
%but in CFT the vacuum energy simply encoded in a number,
%which in turn related to many other interesting things.
%CFT is rich, even the free massless field theory could be studied on various geometries showing interesting structures.
%The most down to earth calculation in string theory can not be done without using the 2d CFT.
%I'm curious about how the powerful techniques, like the OPE,
%could be extended to the higher dimensions $d>2$.
%
%%%%%%%%%%%%%%%%%%
%%  Danger Zone  %
%%%%%%%%%%%%%%%%%%
%
%Among many other things,
%such a study could help people to see how many physics results
%have already cointained in the 
%It encourages people to think about the physics idea deeply and clearly.
%but it strives for a better understanding based on which new ideas could be developed.
