%! Tex root: ../master.tex

\begin{question}
	In \pdfref{DKL94},
	two kinds of solution was discussed:
	the elementary solution and the topological solution.
	The elementary $p$-brane solution carries the ``electric'' Noether charge $e_d$;
	The topological $\tilde{p}$-brane solution carries the ``magnetic'' charge $g_{\tilde{d}}$.
	There is a Dirac quantization rule $e_d g_{\tilde{d}}=2\pi n$.
	How to understand these two dual solutions?
	Whether there is a singularity in these two solutions?
\end{question}

\begin{info}
	The action used in \pdfref{DKL94}.
	Consider an antisymmetry tensor potential of rank $d$,
	$A_{M_1 M_2 \cdots M_d}$,
	in $D$ dimensional space-time $M=0,\cdots,(D-1)$.
	The field strength is $F_{d+1}=dA_d$.
	The following action captures the interaction between
	the gravity $g_{MN}$
	and the dilaton $\phi$:
	\begin{equation}
		I_{D}(d) = \frac{1}{2\kappa^2} \int d^D x
		\sqrt{-g} \left( R - \frac{1}{2}(\partial \phi)^2
		-\frac{1}{2(d+1)!} e^{-a(d)\phi} F_{d+1}^2\right) .
	\end{equation}
	The $a(d)$ is an yet undetermined constant.
\end{info}

Let's compare it with the 
NS-NS sector of the type IIB supergravity action 
written above \pdfref{Joh03}
\[
S_{\text{IIB,NS-NS}} = \frac{1}{2\kappa_0^2}
	\int d^{10} x (-G)^{1 / 2} 	
	e^{-2\Phi} \left[ R + 4 \partial_\mu\Phi \partial^\mu\Phi
		-\frac{1}{12}(H^{(3)})^2\right] 
.\] 
One notices that the dilaton field coupling is different.
One may take one form to another by
the Weyl transformation on the metric:
\begin{info}
	Let's consider a Wely transformation $G\to\tilde{G}$
	\begin{equation}
		\tilde{G}_{\mu\nu} (x) = \mathrm{exp}(2\omega(x)) G_{\mu\nu}(x).
	\end{equation}
	One has the following formula
	for the Ricci scalar
	\begin{equation}
		\tilde{R} = \mathrm{exp}(-2\omega)
		\left[ R - 2(D-1)\nabla^2\omega
		-(D-2)(D-1)\partial_\mu \omega \partial^\mu \omega\right] .
	\end{equation}
	where $D$ is the space-time dimension.
\end{info}

Let's write the $S_{\text{IIB,NS-NS}}$ in terms of the
$\tilde{G}$ and $\tilde{R}$.
\[
	(-G)^{1 / 2} e^{-2\Phi} R = 
	(- \tilde{G})^{1 / 2} e^{-2\Phi - D\omega}
	\left[ e^{2\omega} \tilde{R}
	+ 2(D-1)\nabla^2\omega 
	+ (D-2)(D-1)\partial_\mu\omega \partial^\mu\omega\right] 
.\] 
Then one can choose
\[
\omega = \frac{2}{2-D} \Phi
.\] 
to decouple $R$ with the dilaton $\Phi$.
With this choice
\[
	(-G)^{1 / 2} e^{-2\Phi} R = 
	(- \tilde{G})^{1 / 2}\tilde{R}
	+ (- \tilde{G})^{1 / 2}	e^{- \frac{4}{2-D}\Phi}
	\left[ \frac{4(D-1)}{2-D}\nabla^2\Phi 
	- 2(D-1)\partial_\mu\Phi \partial^\mu\Phi\right]
.\] 
$D=2$ is special in the sense that
one can not decouple the dilaton field
by choosing the Wely scaling factor $\omega(x)$.
Now the dilaton field terms in the action
\[
	(-\tilde{G})^{1 / 2} e^{- \frac{4}{2-D}\Phi}
	\left[ (6-2D) \partial_\mu\Phi \partial^\mu\Phi 
	+\frac{4(D-1)}{2-D}\nabla^2\Phi  \right] 
.\] 
Then one may use the following identity
\[
	e^{- \frac{4}{2-D} \Phi} \cdot \frac{4(D-1)}{2-D}\nabla^2\Phi
	= \frac{16(D-1)}{(2-D)^2} e^{- \frac{4}{2-D} \Phi} \partial_\mu \Phi \partial^\mu \Phi
	- (D-1) \nabla^2 e^{- \frac{4}{2-D} \Phi}
.\] 
to get rid of $\nabla^2\Phi$ term.
The last term on the r.h.s. is a boundary term.

\begin{question}
We will get a term
\[
	-2 \frac{D^3-7D^2+8D-4}{(2-D)^2}e^{- \frac{4}{2-D}\Phi}
	\partial_\mu\Phi \partial^\mu\Phi
.\] 
How to get the usual kinetic term $-\frac{1}{2}\partial_\mu\tilde{\Phi}\partial^\mu\tilde{\Phi}$ out of it?
\end{question}

It's important to note that
the space-time indices $\mu$ should be raised
using the new metric $\tilde{G}^{\mu\nu}$.
$G^{\mu\nu} = e^{2\omega} \tilde{G}^{\mu\nu}= e^{ 4 / (2-D)\Phi} \tilde{G}^{\mu\nu}$.
This cancels the exponent factor before $\partial_\mu\Phi\partial^\mu\Phi$.
However, for the gauge field term $(H^{(3)})^2$,
there are three indices to raise
which leads to the following dilaton coupling
\[
	-\frac{1}{12} e^{\frac{8}{2-D}\Phi} (H^{(3)})^2	
.\] 

The NS-NS sector $\phi,g_{MN}$ and $A_{M_1 M_2 \cdots M_d}$
could be sourced by a $(d-1)$-brane.
One could write down the action that describing
how the brane couples to the fields.
(TODO)

\begin{question}
First, the $(d-1)$-brane that coupling to the NS-NS sector
is not the Dirichlet brane (confirm this point);
because the D-brane sources the R-R sector instead of the NS-NS sector.

However, in the action that is used in \pdfref{DKL94};
it's not clear that the form field $A_{M_1 M_2 \cdots M_d}$
is in the NS-NS sector or the R-R sector,
or could be suitable for both sectors.
\end{question}

The variation of $I_{D}(d)$ with respect to
$\delta A^{M_1\cdots M_d},\delta g^{MN}$ and $\delta\phi$

\begin{info}
The convention for the forms:
\[
	(F_{d+1})_{N M_1 \cdots M_d} = (d+1) \partial_{[N} A_{M_1\cdots M_d]}
.\] 
with
\[
\partial_{[N} A_{M_1\cdots M_d]} = \frac{1}{d+1}
\left( \partial_{N} A_{M_1\cdots M_d} +\partial_{M_1} A_{M_2\cdots N} +\cdots+\partial_{M_d} A_{N\cdots M_{d-1}}  
 \right) 
.\] 
\end{info}

The variation of the form action with respect to $\delta A^{M_1\cdots M_d}$ is
\begin{align*}
	\delta \left[e^{-a \phi} F_{d+1}^2\right] &= 2 e^{-a \phi} (F_{d+1})_{N M_1 \cdots M_d}
	\delta (F_{d+1})^{N M_1 \cdots M_d} \\
	&= \partial^N \left[ e^{-a\phi} F_{N M_1\cdots M_N} \right] 
	\delta A^{M_1\cdots M_N}
	+ \nabla^N (\cdots)_N
\end{align*}
The last term contributes to a boundary term.
Then the equation of motion reads
\begin{equation}
	\partial^N \left( e^{-a\phi} F_{N M_1\cdots M_N} \right) = \cdots 	
\end{equation}
The r.h.s. comes from the possible source term.

\begin{info}
	To do the variation of $\delta g^{MN}$,
	the following formula are necessary
	\begin{equation}
		\delta g_{MN} = - g_{MR} g_{NS} \delta g^{RS}
	\end{equation}
	and
	\begin{equation}
		\delta \sqrt{-g} = - \frac{1}{2} \sqrt{-g}
		g_{MN}\delta g^{MN}
	\end{equation}
	and
	\begin{equation}
		\delta R = R_{MN} \delta g^{MN}
		+ \nabla_{P} (g^{MN}\delta \Gamma^P_{MN}
		-g^{NP}\delta\Gamma^M_{MN})
	\end{equation}
	and
	\begin{equation}
		\delta \Gamma^S_{MN} = \frac{1}{2} g^{SL}
		\left( \nabla_{N} (\delta g)_{ML} + \nabla_{M} (\delta g)_{NL}
		- \nabla_L (\delta g)_{MN}\right) 
	\end{equation}
\end{info}

Calculations:
\[
	-\frac{1}{2}\delta \left[ \sqrt{-g} (\partial \phi)^2\right] 
	=-\frac{1}{2} \sqrt{-g}\left[ -\frac{1}{2} g_{MN}(\partial\phi)^2
	+\partial_M\phi \partial_N\phi\right] \delta g^{MN}
.\] 
and
\begin{align*}
	- \frac{1}{2(d+1)!} \delta \left( \sqrt{-g} e^{-a(d) \phi} F_{d+1}^2 \right)  \\
	= - \frac{\sqrt{-g}}{2} \frac{1}{d!}
	\left( F_{M M_1\cdots M_d} F_{N}^{~~M_1\cdots M_d} 
	-\frac{1}{2(d+1)} g_{MN} F_{d+1}^2\right) 
	e^{-a(d)\phi} \delta g^{MN}
\end{align*}
The variation of the Ricci scalar part
simple gives the Einstein tensor.
The equation of motion reads then
\begin{align*}
	\sqrt{-g} \Bigg\{ R_{MN} - \frac{1}{2} g_{MN}R
	-\frac{1}{2} \left[ \partial_M\phi \partial_N\phi-\frac{1}{2} g_{MN}(\partial\phi)^2
	\right]  \\
- \frac{1}{2(d!)}\left[ F_{M M_1\cdots M_d} F_{N}^{~~M_1\cdots M_d} 
	-\frac{1}{2(d+1)} g_{MN} F_{d+1}^2\right] 
	e^{-a(d)\phi} 
\Bigg\} = \cdots
\end{align*}
The r.h.s. comes from the possible source terms.
