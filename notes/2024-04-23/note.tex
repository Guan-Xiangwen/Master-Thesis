%! Tex root: ../master.tex

Notations
\[
	(\Sigma_{(\alpha)})^{\mu\nu}_{\mathfrak{a}\mathfrak{b}}
	\equiv \left[
	2 a_\rho (\mathcal{A}^\rho J)_{\mathfrak{a}\mathfrak{b}}
	+ \mathcal{A}^2_{\mathfrak{a}\mathfrak{b}}
\right] \eta^{\mu\nu}
- 2 \mathcal{F}^{\mu\nu}_{\mathfrak{a}\mathfrak{b}}
.\] 
note. this term couples to the fluctuations $\alpha_{mu \mathfrak{a}}$
through the form $\alpha_{\mu \mathfrak{a}} 
(\Sigma_{(\alpha)})^{\mu\nu}_{\mathfrak{a}\mathfrak{b}}
\alpha_{\nu \mathfrak{b}}$.
Also
\[
	(\Sigma_{(\varphi)})_{\alpha \mathfrak{a};\beta \mathfrak{b}}
	\equiv (C\Gamma^\mu)_{\alpha\beta}
	\mathcal{A}_{\mu \mathfrak{a}\mathfrak{b}}
.\] 
\[
	(\Sigma_{(\text{g})})_{\mathfrak{a}\mathfrak{b}}
	\equiv 2 a_\mu (\mathcal{A}^\mu J)_{\mathfrak{a}\mathfrak{b}}
	+ \mathcal{A}^2_{\mathfrak{a}\mathfrak{b}}
.\] 
Maybe simply use number to denote the index:
$\Sigma_{12}$ for example.

The action to expand
\begin{align}
	- S = 
	- \frac{1}{2} \alpha (\Delta_{\mathrm{B}}) \alpha
	+ \frac{1}{2} \varphi (\Delta_{\mathrm{F}}) \varphi
	- b ( \Delta_{ \mathrm{G}}) c \notag \\
	- \frac{1}{2g^2} \alpha (\Sigma_{(\alpha)}) \alpha
	+ \frac{1}{2g^2} \varphi (\Sigma_{(\varphi)}) \varphi
	- \frac{1}{g^2} b ( \Sigma_{(\text{g})}) c \notag \\
	- \frac{1}{g^2} \varphi (C\Gamma^\mu) \Psi \alpha_\mu
	+ \frac{1}{4 g^2} \mathrm{Tr}([\alpha_\mu,\alpha_\nu]
	[\alpha^\mu,\alpha^\nu])
\end{align}

First consider the contribution getting from the first two lines:
$ \mathrm{e}^x = 1 + x + \frac{1}{2}x^2 + \frac{1}{6}x^3 + \frac{1}{24}x^4 + \cdots$,
$x$-term
\[
	- \frac{1}{2g^2} \alpha_1 \Sigma_{(\alpha)1,2} \alpha_2
	+ \frac{1}{2g^2} \varphi_1 \Sigma_{(\varphi)1,2} \varphi_2
	- \frac{1}{g^2} b_1 \Sigma_{(\text{g})1,2} c_2
.\] 
Contract by the propagators:
\[
	- \frac{1}{2 g^2} \Sigma_{(\alpha)1,2} 
	(\Delta_{\mathrm{B}}^{-1})_{2,1}
	+ \frac{1}{2 g^2} \Sigma_{(\varphi)1,2}
	(\Delta_{\mathrm{F}}^{-1})_{2,1}
	- \frac{1}{g^2} \Sigma_{(\text{g})1,2}
	(-\Delta_{\mathrm{g}}^{-1})_{2,1}
.\] 
$\frac{1}{2}x^2$-term
\begin{align*}
	\frac{1}{8 g^4} \alpha_1 \Sigma_{(\alpha)1,2} \alpha_2
	\alpha_3 \Sigma_{(\alpha)3,4} \alpha_4
	+ \frac{1}{8 g^4} \varphi_1 \Sigma_{(\varphi)1,2} \varphi_2
	\varphi_3 \Sigma_{(\varphi)3,4} \varphi_4 \\
	+ \frac{1}{2g^4} b_1 \Sigma_{(\text{g})1,2} c_2
	b_3 \Sigma_{(\text{g})3,4} c_4
	-\frac{1}{4g^4} \alpha_1 \Sigma_{(\alpha)1,2} \alpha_2
	\varphi_3 \Sigma_{(\varphi)3,4} \varphi_4 \notag\\
	- \frac{1}{2 g^4} \varphi_1 \Sigma_{(\varphi)1,2} \varphi_2
	b_3 \Sigma_{(\text{g})3,4} c_4
	+ \frac{1}{2 g^4} \alpha_1 \Sigma_{(\alpha)1,2} \alpha_2
	b_3 \Sigma_{(\text{g})3,4} c_4
\end{align*}
Consider only ``connected'' contractions
\begin{align*}
	\frac{1}{8g^4}\left[
		\Sigma_{(\alpha)1,2} (\Delta_{\mathrm{B}}^{-1})_{2,3}
	\Sigma_{(\alpha)3,4} (\Delta_{\mathrm{B}}^{-1})_{4,1}
		+\Sigma_{(\alpha)1,2} (\Delta_{\mathrm{B}}^{-1})_{1,3}
	\Sigma_{(\alpha)3,4} (\Delta_{\mathrm{B}}^{-1})_{4,2}
\right] \notag \\
\frac{1}{8g^4} \left[
	\Sigma_{(\varphi)1,2} (\Delta_{\mathrm{F}}^{-1})_{3,2}
	\Sigma_{(\varphi)3,4} (\Delta_{\mathrm{F}}^{-1})_{4,1}
	+\Sigma_{(\varphi)1,2} (\Delta_{\mathrm{F}}^{-1})_{1,3}
	\Sigma_{(\varphi)3,4} (\Delta_{\mathrm{F}}^{-1})_{4,2}
\right] \notag \\
\frac{1}{2g^4}\left[
	- \Sigma_{(\text{g})1,2} (-\Delta_{\mathrm{g}}^{-1})_{2,3}
	\Sigma_{(\text{g})3,4} (-\Delta_{\mathrm{g}}^{-1})_{4,1}
\right]
\end{align*}
Note: how to think about the order and the sign?
First order the contraction pair $\left<(x_1 y_1)(x_2 y_2)\cdots \right>$.
Imagine couple them with sources in the action
$u_1 x_1 + v_1 y_1 + u_2 x_2 + v_2 y_2 + \cdots$.
Then $\left<(x_1 y_1) (x_2 y_2) \cdots \right>$ is obtained by taking the derivatives
$\partial_{u_1} \partial_{v_1} \partial_{u_2} \partial_{v_2} \cdots$ to the generating functional.
The plus or minus sign of the source term doesn't matter because we always have an even number of them.
The propagator appears from the same action (rewriting)
having the form $ x_i \Delta_{ij} y_j \to v_i (\Delta^{-1})_{ij} u_j$.
Be careful about the position of $u,v$.
So acting the derivative we get $(\Delta^{-1})_{v_1 u_1}
(\Delta^{-1})_{v_2 u_2}\cdots$.
The only sign in this procedure comes from the initial ordering.

$\frac{1}{6}x^3$-term (for simplification, only show the terms leading to some connected contractions)
(also, the last line of the action is ignored)
\begin{align*}
	- \frac{1}{48 g^6} \alpha_1 \Sigma_{(\alpha)1,2} \alpha_2
	\alpha_3 \Sigma_{(\alpha)3,4} \alpha_4
	\alpha_5 \Sigma_{(\alpha)5,6} \alpha_6 \notag\\
	\frac{1}{48 g^6} \varphi_1 \Sigma_{(\varphi)1,2} \varphi_2
	\varphi_3 \Sigma_{(\varphi)3,4} \varphi_4
	\varphi_5 \Sigma_{(\varphi)5,6} \varphi_6 \notag \\
	- \frac{1}{6 g^6} b_1 \Sigma_{(\text{g})1,2} c_2
	b_3 \Sigma_{(\text{g})3,4} c_4
	b_5 \Sigma_{(\text{g})5,6} c_6
\end{align*}
Contractions
\[
	(23)(45)(16) \text{ with all possible interchanges }
	(1 \leftrightarrow 2)(3 \leftrightarrow 4)(5 \leftrightarrow 6)
.\] 
For the ghost part, only one possibility: all interchanges happen at once.
\begin{align*}
	- \frac{1}{48 g^6} \Sigma_{(\alpha)1,2} (\Delta_{\mathrm{B}}^{-1})_{2,3}
	\Sigma_{(\alpha)3,4} (\Delta_{\mathrm{B}}^{-1})_{4,5}
	\Sigma_{(\alpha)5,6} (\Delta_{\mathrm{B}}^{-1})_{6,1}\notag\\
	\frac{1}{48 g^6} \Sigma_{(\varphi)1,2} (\Delta_{\mathrm{F}}^{-1})_{3,2}
	\Sigma_{(\varphi)3,4}(\Delta_{\mathrm{F}}^{-1})_{5,4}
	\Sigma_{(\varphi)5,6} (\Delta_{\mathrm{F}}^{-1})_{6,1} \notag\\
	-\frac{1}{6g^6}\Sigma_{(\text{g})1,2} (-\Delta_{\mathrm{g}}^{-1})_{2,3}
	\Sigma_{(\text{g})3,4}(-\Delta_{\mathrm{g}}^{-1})_{4,5}
	\Sigma_{(\text{g})5,6}(-\Delta_{\mathrm{g}}^{-1})_{6,1}\notag\\
	+ \text{ all possible interchanges } \cdots
\end{align*}

$\frac{1}{24}x^4$-term (for simplification, only show the terms leading to some connected contractions)
(also, the last line of the action is ignored)
\begin{align*}
	\frac{1}{384 g^8} \alpha_1 \Sigma_{(\alpha)1,2} \alpha_2
	\alpha_3 \Sigma_{(\alpha)3,4} \alpha_4
	\alpha_5 \Sigma_{(\alpha)5,6} \alpha_6 
	\alpha_7 \Sigma_{(\alpha)7,8} \alpha_8\notag\\
	\frac{1}{384 g^6} \varphi_1 \Sigma_{(\varphi)1,2} \varphi_2
	\varphi_3 \Sigma_{(\varphi)3,4} \varphi_4
	\varphi_5 \Sigma_{(\varphi)5,6} \varphi_6 
	\varphi_7 \Sigma_{(\varphi)7,8} \varphi_8\notag \\
	\frac{1}{24 g^6} b_1 \Sigma_{(\text{g})1,2} c_2
	b_3 \Sigma_{(\text{g})3,4} c_4
	b_5 \Sigma_{(\text{g})5,6} c_6
	b_7 \Sigma_{(\text{g})7,8} c_8
\end{align*}
More contraction possibilities...
$\text{number}=6(3)\times 2^4 = 96(48)$.
(todo. find a systematic way to consider all connected contractions)

\begin{correct}
Further calculation for the fermionic $x^3$-term
Contractions:
\[
	\frac{8}{48g^6}\Sigma_{12}\Delta_{23}\Sigma_{34}\Delta_{45}
	\Sigma_{56}\Delta_{61}
	,\quad
	\Delta = \overline{C\Gamma}\cdot a J,~\Sigma = C\Gamma\cdot \mathcal{A}
.\] 
next:
\[
\frac{1}{6 a^6} \mathrm{Tr}(\Gamma^1 \overline{\Gamma}^2 \Gamma^3
\overline{\Gamma}^4 \Gamma^5 \overline{\Gamma}^6)
a^2 a^4 a^6 \mathrm{Tr}(\mathcal{A}^1  \mathcal{A}^3 \mathcal{A}^5 J)
.\] 
\[
= \frac{32}{3a^6}\mathrm{Tr}(a\cdot \mathcal{A} a \cdot \mathcal{A}
a\cdot \mathcal{A} J)
- \frac{8}{ a^4} \mathrm{Tr}(a\cdot \mathcal{A} \mathcal{A}^2 J)
.\] 
Something wrong or not?
\end{correct}
