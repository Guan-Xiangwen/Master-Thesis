A review of the matrix model renormalization group method is
\urlref{http://arxiv.org/abs/1410.1635}{Zinn14}.
Not only matrix model, but also simpler cases are contained in this review.
Some solutions of the fix points (the potential that is invariant under the RG flow) are given.
References that predate this review are given at the end of this note.
More recently there are some articles about the ``functional renormalization group approach'' applied to the matrix model.
For example
\urlref{http://arxiv.org/abs/1909.03327}{LS19}.

\section{continum limit}
The RG method of matrix model was proposed to study the continuum limit.
Below I review the definition of it.

A critical point can be defined for matrix integral in the $N\to\infty$ imit. 
Consider the partition function $Z(g;N)$ 
\begin{equation}
	\mathrm{e}^{Z} = \int [\mathrm{d}M] \mathrm{e}^{-\frac{N}{g}V[M]}
.\end{equation}
and denote the leading order in $N\to\infty$ as $Z_0(g)$. 
This corresponds to the planar diagrams.
The critical point $g_*$ then is conventionally defined as the point where
\[
Z_0(g)\sim (g-g_*)^{2-\gamma}
.\]
The critical exponent $\gamma$ is introduced in such a way to make contact with that of string susceptibility (in the contex of 2d gravity).
This definition follows
\urlref{https://arxiv.org/abs/hep-th/9306153}{Francesco92}.

The motivation for introducing the critical point is that
at this point the matrix model is argued to generate the partition function of a continuous 2d surface.
This can be proved order by order:
the order in $N$ labels the genus and the order in $g$ labels the area.
It is around the critical point that the ``macroscopic'' area contributes significantly to the partition function.
However, this ``order by order argument'' leads to a non-perturbative statement which is skeptical...
One should also note that $Z$ is the $\ln$ of the matrix integral, or the matrix partition function.
It's the free energy in the matrix model sense.

What's the value of $\gamma$? The typical value that is calculated in conformal field theory of 2d gravity is $\gamma = -1 / m, m=2,3,\cdots$.
The value $m$ is related to the central charge.
In these cases, $g_*$ is the end point of branch cuts of $Z_0(g)$.
Let's remark on the position of $g_*$.
The matrix integral is usually well defined on the positive real axis $g>0$.
However, to locate the critical point,
one needs to continue $g$ to the complex plane.
The $g_*$ generally locate at the extended regime.
A typical example for quartic action $V(M)=\frac{1}{2}M^2 + \frac{1}{4}M^4$ will give a critical point $g_*=-\frac{1}{12}$.
The calculation is sketched in
\urlref{https://arxiv.org/abs/hep-th/9306153}{Francesco92}.
I will review it in the following.

\section{methods of calculating the critical point (the resolvent method)}
It's possible to determine $g_*$ by the resolvent method.

This method bases on the assumption that
the $N\to\infty$ limit of matrix model can be studied according to a contitnuous function $\rho(\lambda)$.
It's defined as
\begin{equation*}
    \rho(\lambda) := \lim_{N\to\infty} \frac{1}{N}\left\langle \sum_{i=1}^N \delta(\lambda - \lambda_i) \right\rangle.
\end{equation*}
$\langle\cdots\rangle$ denotes the average over the probability distribution;
$\lambda_i$ are the eigenvalues of the matrices in the ensemble.
By definition the moment $\langle \frac{1}{N} \mathrm{Tr} M^k \rangle$ can be calculated by $\int \lambda^k \rho(\lambda)\mathrm{d}\lambda$.

I'd like to add here a remark on $\rho(\lambda)$.
It's by no mean a complete characterization of the matrix model.
In particular, it's impossible to rewrite the $\mathrm{exp}(-S[M])$ in terms of $\rho(\lambda)$.
Interesting quantities, like ``average level spacing'' or trace product $\langle\mathrm{Tr} M^k \mathrm{Tr} M^j\rangle$, can not be calculated from $\rho(\lambda)$.
A proper understanding of $\rho(\lambda)$ is
rather than telling us how the probability distribution looks like,
it tells us what's the ``typical matrix'' of the matrix model in $N\to\infty$.

$\rho(\lambda)$ is analized by an auxillary function $\omega(z)$, the average trace of the resolvent,
\begin{equation}
	\omega(z) := \left\langle \mathrm{tr}\frac{1}{z - M} \right\rangle
.\end{equation}
It turns out that $\omega(z)$ is the Stieltjes transformation of $\rho(\lambda)$.
Reversely, $\rho(\lambda)$ is determined by the discontinuity of $\omega(z)$ acrossing the branch cut.
The analyticity of $\omega(z)$ will be determined by the saddle point equation of $N\to\infty$,
It turns out that the singular part of $\omega(z)$ has the form of branch cuts of square root.
The end points of the branch cuts are essentially determined by the action $S[M]$.
After solving for the $\omega(z)$,
one can use the saddle point approximation to evaluate $Z_0(g)$,
for which the ``on shell action'' is evaluated at $\rho(\lambda)$. 

Let's illustrate the result for the quartic potential $V(\lambda)=\frac{1}{2}\lambda^2+\frac{1}{4}\lambda^4$.
There is only one branch cut $[-a,a]$, and the end point turns out to be \[
	a^2(g) = \frac{2}{3}(-1+\sqrt{1+12 g})
.\] 
$a$ decreases to $0$ as $g$ decreasing to $0$.
The matrix model is ill defined for $g\leq 0$,
however the partition function can be defined by continuation.
Then for $g<0$, the branch cut is on the imaginary axis.
Until $g=g_*=-\frac{1}{12}$, one expects a branch cut singularity,
which turns out to be the critical point of $Z_0(g)$.
I will simply cite the equations that can be used to study the singularity of $Z_0(g)$
\begin{align}
	\frac{\partial}{\partial g} \left(g^3 \frac{\partial Z_0(g)}{\partial g}\right) \equiv g u(g),\\
	\frac{\partial gu(g)}{\partial g} = 2 N^2 g \frac{\ln a(g)}{\partial g}.
\end{align}
By integrating one gets the leading singularity of $Z_0(g)$ is
\begin{equation}
	Z_0(g) \sim - \frac{4}{15} N^2 x^{\frac{5}{2}} + \cdots
\end{equation}
with $x=1 - \frac{g}{g_c}$.

It's also possible to determine $g_*$ with the orthogonal polynomial method.
Assume that one finds the family of orthogonal polynomials with the normalization
\begin{equation}
	\int (\mathrm{d}\lambda) \mathrm{e}^{-\frac{N}{g}\mathrm{tr}V(\lambda)}p_n(\lambda) p_m(\lambda) = s_n \delta_{nm}
.\end{equation}
To obtain the continuum limit, one must first take $N\to\infty$,
then replacing $s_n/s_{n-1}=f(\xi),\xi=n/N$ with a continuous function.
The matrix partition function $Z(g)$ then can be calculated from $f(\xi)$ through
\begin{equation}
	\lim_{N\to\infty} \frac{1}{N^2} \ln Z(g) \sim \int \mathrm{d}\xi (1-\xi) \ln f(\xi)
.\end{equation}
We use $\sim$ to ignore additive constant which is irrelavent to the asymptotic $g\to g_*$.
It turns out that around the critical point $g\approx g_*$,
$f(\xi)$ should have the following scaling
\begin{equation}
	f(\xi) - f_* \sim (g_* - g\xi)^{-\gamma}
.\end{equation}
where $f_*$ is some constant determined by the potential.
This will give the expected scaling of $\ln Z(g)$.

The definition of $g_*$ above based on $N\to\infty$.
Is it possible to define it for finite $N$?

\section{RG equation for continuum limit}

In this section, I review the RG method proposed in 
\urlref{https://arxiv.org/abs/hep-th/9206035}{BZ92}.
This is proposed as another way to find the critical point of continuum limit,
so $N\to\infty$ is imposed in the calculation.
They use an even potential as example $V(\lambda)=\frac{1}{2}\lambda^2+\frac{g}{4}\lambda^4$,
whose critical point is $g_*=-\frac{1}{12}<0$.
It's hard to make sense of the matrix integral with $g<0$, so it should be understood as analytic continuation.

How does this method give the critical point?
The critical point is defined as the point at where the matrix model free energy$Z_N(g)$ has a scaling law.
Then this scaling law is formulated as a Callan-Symanzik like equation
\[
	\left[N \frac{\partial}{\partial N}- \beta(g)\frac{\partial}{\partial g}+\gamma(g)\right]Z_N(g) = r(g)
.\]
At the fix point $\beta(g_*)=0$, the scaling law can be recovered.
Such an equation will arise from the RG calculation, from which one could obtain the functions $\beta(g),\gamma(g),r(g)$.
Then one can find the point $g_*$ where the free energy exhibits proper scaling law.

Another point is that
an RG method could allow us to study more general matrix models,
especially those without an eigenvalue representation.
This application of the RG method appears in
\urlref{https://arxiv.org/abs/1306.3019}{KT13}
in which the ``Yang-Mills two-matrices model'' is studied.
This model is characterized by a term $[A_1,A_2]^2$ in the action.
Such a term can arise from the Yang-Mills Lagrangian $F^2=(\mathrm{d}A)^2$ by ignoring the derivative terms.
The gauge group of this model is $U(N)$.

\section{simpler example: the vector model}
Integration could be studied using an RG spirit method,
which is similar to the Wilson's RG scheme.
However, there is no natural notion of energy associated with the integral.

Let's take the ``$O(N)$ vector model'' as the example
\urlref{https://arxiv.org/abs/1410.1635}{Zinn14}
\begin{equation}
	\mathrm{e}^{Z_N} = \left(\frac{N}{2\pi}\right)^{N / 2} \int \mathrm{d}^N \mathbf{x} \mathrm{e}^{-N V(\mathbf{x}^2)}
.\end{equation}
The normalization is chosen to normalize a Gaussian integral.
The factor $N$ before $V(\mathbf{x}^2)$ gives a ``dimension'' of $V$ under the scaling of $N$.
To implement the changing of $N$, one calculates $Z_{N+1}$ by integrating out one component of the vector $\mathbf{x}$
\begin{equation}
	\mathrm{e}^{Z_{N+1}} = \left(\frac{N+1}{2\pi}\right)^{(N+1)/2} \int \mathrm{d}^N \mathbf{x} \int \mathrm{d} y \mathrm{e}^{-(N+1) V(\mathbf{x}^2 + y^2)}
\end{equation}
It's impossible to integrate out $y$ exactly in general,
so one may use the saddle point approximation.
For a generic $\mathbf{x}$, the integrand $\mathrm{e}^{-(N+1)V(\mathbf{x}^2 + y^2)}$ is localized at the critical value of $y$ when $N\to\infty$.
To the leading order, assuming that the saddle point is at $y=0$,the approximation gives \[
	\mathrm{e}^{Z_{N+1}} = \left(\frac{N+1}{2\pi}\right)^{N / 2} \int \mathrm{d}^N \mathbf{x} \mathrm{e}^{-(N+1)V(\mathbf{x}) - \frac{1}{2} \ln 2 V'(\mathbf{x}^2)} \left(1 + O(\frac{1}{N})\right)
.\] 
It should be noted that the saddle point approximation fail when $V'(\mathbf{x}^2) = 0$.
The Gaussian potential $V(\mathbf{x}^2) = (1 /2) \mathbf{x}^2$ will give $ V'(\mathbf{x}^2) = 1 /2$.
In this case, the leading order vanishes.

It's not clear to what extent the subleading corrections will modify the scaling around the critical point;
however, I guess they are not very important if the critical point by itself requires $N\to\infty$
(similar to the fact that phase transition requires thermodynamic limit).
Due to various methods to solve $N\to\infty$ model, it deserves study;
however, is it possible to obtain a matrix model that has scaling covariance for all $N$?
This is a stringent requirement,
but there is example of matrix model that is independent of $N$.
For example, see the discussion of the Kontsevich integral in
\urlref{https://arxiv.org/abs/hep-th/9303139}{Morozov94}
below the equation (3.70).

Come back to the discussion of the vector model.
The change $N\to N+1 $ induces a change in $\mathbf{x}$ and $V$
\begin{align}
	\mathbf{x} \to \mathbf{x} \left(1- \frac{1}{2N}(1+\gamma)\right),\\
	V \to V + \delta V
\end{align}
such that the partition function scales as
\begin{equation}
	Z_{N+1}(V) = -\frac{1}{2} \gamma + Z_N(V+\delta V) + O(\frac{1}{N})
\end{equation}
The $-\frac{1}{2}\gamma$ term comes form the scaling of the integration measure $\mathrm{d}^N \mathbf{x}$.
The $V+\delta V$ comes from the integration over $y$, also the scaling of $\mathbf{x}$
\[
N \delta V (\rho) = V(\rho) - (1+\gamma) \rho V'(\rho) + \frac{1}{2} \ln 2 V'(\rho) + O(\frac{1}{N})
.\] 
the $\rho$ is a shorthand notation for $\mathbf{x}^2$.
The differential equation for the deformation of $V$ along the RG flow reads then
\begin{equation}
	\lambda \frac{\mathrm{d}}{\mathrm{d} \lambda} V(\rho,\lambda) = V(\rho,\lambda) - \left(1+\gamma(\lambda)\right) \rho V'(\rho,\lambda) + \frac{1}{2} \ln 2V'(\rho,\lambda)
\end{equation}
where $\lambda$ is a continuous parameter $N\to \lambda N$.
Note that $\gamma$ is assumed to depend on $N$.


$V(\rho) = \frac{1}{2} \rho$ is a fix point with $\gamma=0$, the Gaussian fixed point.
It's not obvious in general how to solve the fix point equation due to the non-linear term $\ln V'$.
One possible solution is
\[
V(\rho) = \frac{1}{2} \ln \rho
.\] 
with $\gamma=-1$ such that the second term vanish.
If plugging this back to the integral, one finds the integrand is $\rho^{- N / 2}$.
This cancels with the radius measure in spherical coordinate.
Therefore, this potential corresponds to a trivial vector model.

In general, it turns out that it's easier to solve the equation for $V'$.
Actually, if one define a function $R(\rho)$
\[
R(\rho) \equiv \frac{1}{2 V'(\rho)}
.\] 
Then the fix point equation will simply be
\begin{equation}
	\gamma R + R R' - (1 + \gamma) \rho R' = 0
\end{equation}
The $\gamma = -1$ case gives $R'=1$, which corresponding to the solution $V(\rho)=\frac{1}{2}\ln\rho$.
Then how to obtain the solution of, for example $\gamma=-2$?

Note that the finite scaling of $\mathbf{x}$ is
\[
\mathbf{x} \to \mathbf{x} \left(1 + \frac{1 + \gamma}{2} \ln \lambda\right)
.\] 
Instead of power of $\lambda$, it scales as $\ln \lambda$.

\section{a matrix integral calulation}
I don't follow the calculation in the reference.
There the calculation requires taking $N\to\infty$ because the use of saddle point equation.
I try to avoid the use of saddle point approximation.
However, there are still some problems to solve.

The model under consideration is the one-matrix $\phi^4$ model
\begin{equation}
    \zeta_N(g) = \int \mathrm{d}\phi_N \mathrm{exp}(-S_N[\phi_N,g]),
\end{equation}
with an action
\begin{equation}
    S_N[\phi_N,g] = \mathrm{Tr} (\frac{1}{2}\phi_N^2 + \frac{g}{4}\phi_N^4).
\end{equation}

Start with the rank $N+1$ model, and decomposing the matrix $\phi_{N+1}$ as
\begin{equation}
    \phi_{N+1} = \begin{pmatrix}
\phi_N & v\\ 
 v^\dagger & \alpha
\end{pmatrix}.
\end{equation}
Then the action can be expanded as
\begin{align}
	S_{N+1}[\phi_{N+1},g] = &\mathrm{Tr}\left( \frac{1}{2}\phi_N^2+\frac{g}{4}\phi_N^4\right) + v^\dagger v + \frac{1}{2}\alpha^2 \notag \\
	&+ g\left( v^\dagger \phi_N^2 v + \alpha v^\dagger \phi_N v + \alpha^2v^\dagger v + \frac{1}{2}(v^\dagger v)^2 + \frac{1}{4}\alpha^4\right ).
\end{align}
Now the trace is over $N\times N$ matrix.

Recall that the matrix model is $U(N)$-invariant;
it may be useful to first gauge away some variables in $v,a$.
Let's consider how the gauge transformation acting on $\phi_N, v$ and $a$.
The infinitesimal transformation reads
\begin{align}
    \delta_t\phi_N &= i (vt^\dagger - tv^\dagger) \notag\\
	\delta_t v &= i(\phi_N - \alpha\mathds{1})t\notag\\
\delta_t v^\dagger &= it^\dagger (\alpha\mathds{1}-\phi_N) \notag\\
\delta_t \alpha &= i(v^\dagger t - t^\dagger v)
\end{align}
where $t$ ($N\times 1$ vector) is the components of the following generator
\[
 T = \begin{pmatrix}
	 0 & t \\
	 t^\dagger & 0
 \end{pmatrix}
.\] 
It's possible to choose $t$ to gauge away $v$ if $\phi_N - \alpha \mathds{1}$ is not degenerate.
So let's impose the gauge fixing condition as $v=0,v^\dagger=0$.

How to implement this condition in the integral?
Essentially what we should do is to change the integration variable from $v$ to $t$.
$t$ is understood as the parameter for the coset $U(N+1)/U(N)$ through exponent.
Around infinitesimal neighbour of $v=0$, $v(t)$ is a linear function given by the infinitesimal gauge transformation
\[
	v(t) = - i (\phi_N -\alpha \mathds{1})t
.\] 
Then the ``coordinate transformation'' will give a Jacobian that is proportional to $\mathrm{det} (\phi_N - \alpha \mathds{1})$.
The same factor will be obtained from $v^\dagger$.
Therefore we get
\[\left[\mathrm{det}(\phi_N - \alpha \mathds{1})\right]^2\]
in the integrand, which can be re-exponentiate to give a term in the action
\begin{equation}
    S_{N+1}[\phi_{N+1},g] = \mathrm{Tr}\left( \frac{1}{2}\phi_N^2+\frac{g}{4}\phi_N^4\right) + \frac{1}{2}\alpha^2 +  \frac{g}{4}\alpha^4 - 2 \mathrm{Tr} \mathrm{ln}(\phi_N - \alpha \mathds{1}).
\end{equation}
Note that this formula is equivalent to the eigenvalue representation if we diagonalize $\phi_N$.

However, the above calculation based on the linearized version of gauge transformation.
The full gauge transformation is more complicate.
We can proceed as following.
Consider a $(N+1)\times (N+1)$-matrix $\phi_{N+1}(\phi_N,0,a)$ with $v=0$.
Then do a gauge transformation, parameterized by $t$, on this matrix $\phi_{N+1}\to \tilde{\phi}_{N+1} = U(t) \phi_{N+1} U^\dagger(t)$.
The idea is that the integration over $\tilde{\phi}_{N+1}$ can be replaced with an integration over $\phi_{N+1}$ and $t$,
with a proper Jacobian that taking into account the functional dependence $\tilde{\phi}_{N+1}(\phi_{N+1},t)$.
In the exponent, because of the gauge invariance, one can replace $\tilde{\phi}$ directly with $\phi$.
To write down the Jacobian, one needs to solve for the $\tilde{v}(t)$, which is fixed by the condition $v=0$.
It turns out that the $t$-dependence of the Jacobian can be factorized out, so the $t$-integral can be performed independently.
This factorization property also implies that the linearized result giving the exact Jacobian for the $\phi$ factor,
although we don't know the $t$-factor.

In summary, we have
\begin{equation}
	\int \mathrm{d}\phi_N \mathrm{d}\alpha  [\mathrm{det}(\phi_N-\alpha\mathds{1})]^2 \mathrm{e}^{-\frac{1}{2}\alpha^2 - \frac{g}{4}\alpha^4-S_N[\phi_N,g]}
\end{equation}
or
\begin{equation}
	\int \mathrm{d} \phi_N \mathrm{d}\alpha \mathrm{e}^{-\frac{1}{2} \alpha^2 - \frac{g}{4}\alpha^4 - S_N[\phi_N,g] + 2 \mathrm{Tr}\ln (\phi_N - \alpha \mathds{1})} 
\end{equation}
The $\alpha$ integral is difficult to understand.
But I try to study it as follows.

The determinant is a characteristic polynomial $\mathrm{det}( \alpha\mathds{1} - \phi_N)\equiv p_N (\alpha)$ of $\phi_N$ in $\alpha$ with coefficients given by the following formula
\begin{equation}
    p_N (\alpha) = \sum_{k=0}^N \alpha^{N-k}(-1)^k \mathrm{Tr} \left(\wedge^k \phi_N \right),
\end{equation}
with
\[\mathrm{Tr} \left(\wedge^k \phi \right) = \frac{1}{k!}\mathrm{det}\begin{vmatrix}
\mathrm{Tr} \phi & k-1 & 0 & \cdots & 0\\ 
\mathrm{Tr} \phi^2 & \mathrm{Tr} \phi & k-2 & \cdots & 0 \\ 
\vdots & \vdots & \ddots  &  &\vdots \\ 
\mathrm{Tr} \phi^{k-1} & \mathrm{Tr} \phi^{k-2} &  & \ddots  & 1\\ 
\mathrm{Tr} \phi^k & \mathrm{Tr} \phi^{k-1} & \mathrm{Tr} \phi^{k-2} & \cdots & \mathrm{Tr} \phi
\end{vmatrix}\]
The leading term of $p_N(\alpha)$ is $\alpha^N$, and the last term is $(-1)^N \mathrm{det}\phi_N$. Also, let's write down first few terms to get a feeling
\[p_N(\alpha) = \alpha^N - \alpha^{N-1}\mathrm{Tr}\phi + \frac{1}{2}\alpha^{N-2}\left[(\mathrm{Tr} \phi)^2 - \mathrm{Tr} \phi^2\right] - \frac{1}{6}\alpha^{N-3}\left[(\mathrm{Tr}\phi)^3 - 3\mathrm{Tr}\phi \mathrm{Tr}\phi^2 + \mathrm{Tr}\phi^3\right] + \cdots\]
If we square it as in the integrand, we get
\begin{align}
	p_N^2(\alpha) =& \alpha^{2N} - 2 \alpha^{2N-1} \mathrm{Tr}\phi + \alpha^{2N-2}\left[2(\mathrm{Tr}\phi)^2 - \mathrm{Tr} \phi^2 \right] \notag\\
				   & -\alpha^{2N-3} \left[\frac{4}{3}(\mathrm{Tr}\phi)^3 -2 (\mathrm{Tr}\phi)(\mathrm{Tr}\phi^2) + \frac{1}{3}\mathrm{Tr}\phi^3\right] \cdots
\end{align}
One can only consider even terms if the potential is even.
For each term, $\alpha$ is decoupled from $\phi$, therefore can be integrated out.

To simpilfy the result, the following equation maybe useful
\begin{align}
	2\left\langle \mathrm{Tr}\phi^2 \right\rangle + \left\langle (\mathrm{Tr}\phi)^2 \right\rangle &= \left\langle \mathrm{Tr}(\phi^4 + g \phi^6) \right\rangle \notag\\
	2\left\langle \mathrm{Tr}\phi^3 \right\rangle + 2\left\langle (\mathrm{Tr}\phi)(\mathrm{Tr}\phi^2) \right\rangle &= \left\langle \mathrm{Tr}(\phi^5 + g \phi^7) \right\rangle
\end{align}
The $\langle \rangle$ indicates the matrix integral.
These equations follow from changing the integral variable $\phi \to \phi + \epsilon \phi^{3,4}$, with infinitesimal $\epsilon$.
The left hand side comes from the change of measure; while the right hand side follows from the change of the action.
For example, the $\alpha^{2N-2}$ term becomes
\[
	\alpha^{2N-2}\left[\mathrm{Tr}(-5\phi^2 + 2\phi^4 + 2g\phi^6)\right]
.\] 

As for the integration over $\alpha$
\[
	\int (\mathrm{d}\alpha) \alpha^{2N - k} \mathrm{e}^{- \frac{1}{2} \alpha^2 - \frac{g}{4} \alpha^4}
.\] 
The same trick allows us to derive that
\[
	(2N-k) \langle \alpha^{2N-k-1}\rangle = \langle \alpha^{2N-k + 1} \rangle + g \langle \alpha^{2N-k+3} \rangle
.\] 
For example $k=1$
\[
	\langle \alpha^{2N-2} \rangle = \frac{1}{2N-1} (\langle \alpha^{2N} \rangle + g \langle \alpha^{2N+2}\rangle)
.\] 
$k=3$
\[
	\langle \alpha^{2N-4} \rangle = \frac{1}{2N-3}(\langle \alpha^{2N-2} \rangle + g \langle \alpha^{2N} \rangle)
.\] 
For small $g$, it's reasonable that we only keep the $\alpha^{2N},\alpha^{2N-2}$ terms in large $N$ limit.
For a general $g$, it's not clear that whether or not $\alpha^{2N-2},\alpha^{2N-4}$ are in the same order or not.

\section{related references}

In this section I will give a short remark on references.
The idea stems from \urlref{https://arxiv.org/pdf/hep-th/9206035}{BZ92}.
A more systematic way to extend the calculation to subleading orders was presented at \urlref{https://arxiv.org/abs/hep-th/9304090}{Aya93}.
The idea of using the reparameterization identity to reduce the number of coupling constants in the flow equation appears in \urlref{https://arxiv.org/abs/hep-th/9307154}{HIS93};
A more compact flow equation that realizes this idea is given in a following paper \urlref{https://arxiv.org/abs/hep-th/9409157}{HINS94-a}.
A similar discussion that involving a two-matrix model is presented in \urlref{https://arxiv.org/abs/hep-th/9409009}{HINS94-b}.
