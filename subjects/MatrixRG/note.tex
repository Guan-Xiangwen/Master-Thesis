\section{continum limit}
\marginpar{\today}
A critical point can be defined for matrix integral in the $N\to\infty$ imit. 
Consider the partition function $Z(g;N)$ 
\begin{equation}
	\mathrm{e}^{Z} = \int [\mathrm{d}M] \mathrm{e}^{-\frac{N}{g}V[M]}
.\end{equation}
and denote the leading order in $N\to\infty$ as $Z_0(g)$. 
This corresponds to the planar diagrams.
The critical point $g_*$ then is conventionally defined as the point where
\[
Z_0(g)\sim (g-g_*)^{2-\gamma}
.\]
The critical exponent $\gamma$ is introduced in such a way to make contact with that of string susceptibility (in the contex of 2d gravity).
This definition follows
\urlref{https://arxiv.org/abs/hep-th/9306153}{Francesco92}.

The motivation for introducing the critical point is that
at this point the matrix model is argued to generate the partition function of a continuous 2d surface.
This can be proved order by order:
the order in $N$ labels the genus and the order in $g$ labels the area.
It is around the critical point that the ``macroscopic'' area contributes significantly to the partition function.
However, this ``order by order argument'' leads to a non-perturbative statement which is skeptical...
One should also note that $Z$ is the $\ln$ of the matrix integral, or the matrix partition function.
It's the free energy in the matrix model sense.

What's the value of $\gamma$? The typical value that is calculated in conformal field theory of 2d gravity is $\gamma = -1 / m, m=2,3,\cdots$.
The value $m$ is related to the central charge.
In these cases, $g_*$ is the end point of branch cuts of $Z_0(g)$.
Let's remark on the position of $g_*$.
The matrix integral is usually well defined on the positive real axis $g>0$.
However, to locate the critical point,
one needs to continue $g$ to the complex plane.
The $g_*$ generally locate at the extended regime.
A typical example for quartic action $V(M)=\frac{1}{2}M^2 + \frac{1}{4}M^4$ will give a critical point $g_*=-\frac{1}{12}$.
The calculation is sketched in
\urlref{https://arxiv.org/abs/hep-th/9306153}{Francesco92}.
I will review it in the following.

\subsection{methods of calculating the critical point (the resolvent method)}
It's possible to determine $g_*$ by the resolvent method.

This method bases on the assumption that
the $N\to\infty$ limit of matrix model can be studied according to a contitnuous function $\rho(\lambda)$.
It's defined as
\begin{equation*}
    \rho(\lambda) := \lim_{N\to\infty} \frac{1}{N}\left\langle \sum_{i=1}^N \delta(\lambda - \lambda_i) \right\rangle.
\end{equation*}
$\langle\cdots\rangle$ denotes the average over the probability distribution;
$\lambda_i$ are the eigenvalues of the matrices in the ensemble.
By definition the moment $\langle \frac{1}{N} \mathrm{Tr} M^k \rangle$ can be calculated by $\int \lambda^k \rho(\lambda)\mathrm{d}\lambda$.

I'd like to add here a remark on $\rho(\lambda)$.
It's by no mean a complete characterization of the matrix model.
In particular, it's impossible to rewrite the $\mathrm{exp}(-S[M])$ in terms of $\rho(\lambda)$.
Interesting quantities, like ``average level spacing'' or trace product $\langle\mathrm{Tr} M^k \mathrm{Tr} M^j\rangle$, can not be calculated from $\rho(\lambda)$.
A proper understanding of $\rho(\lambda)$ is
rather than telling us how the probability distribution looks like,
it tells us what's the ``typical matrix'' of the matrix model in $N\to\infty$.

$\rho(\lambda)$ is analized by an auxillary function $\omega(z)$, the average trace of the resolvent,
\begin{equation}
	\omega(z) := \left\langle \mathrm{tr}\frac{1}{z - M} \right\rangle
.\end{equation}
It turns out that $\omega(z)$ is the Stieltjes transformation of $\rho(\lambda)$.
Reversely, $\rho(\lambda)$ is determined by the discontinuity of $\omega(z)$ acrossing the branch cut.
The analyticity of $\omega(z)$ will be determined by the saddle point equation of $N\to\infty$,
It turns out that the singular part of $\omega(z)$ has the form of branch cuts of square root.
The end points of the branch cuts are essentially determined by the action $S[M]$.
After solving for the $\omega(z)$,
one can use the saddle point approximation to evaluate $Z_0(g)$,
for which the ``on shell action'' is evaluated at $\rho(\lambda)$. 

Let's illustrate the result for the quartic potential $V(\lambda)=\frac{1}{2}\lambda^2+\frac{1}{4}\lambda^4$.
There is only one branch cut $[-a,a]$, and the end point turns out to be \[
	a^2(g) = \frac{2}{3}(-1+\sqrt{1+12 g})
.\] 
$a$ decreases to $0$ as $g$ decreasing to $0$.
The matrix model is ill defined for $g\leq 0$,
however the partition function can be defined by continuation.
Then for $g<0$, the branch cut is on the imaginary axis.
Until $g=g_*=-\frac{1}{12}$, one expects a branch cut singularity,
which turns out to be the critical point of $Z_0(g)$.
I will simply cite the equations that can be used to study the singularity of $Z_0(g)$
\begin{align}
	\frac{\partial}{\partial g} \left(g^3 \frac{\partial Z_0(g)}{\partial g}\right) \equiv g u(g),\\
	\frac{\partial gu(g)}{\partial g} = 2 N^2 g \frac{\ln a(g)}{\partial g}.
\end{align}
By integrating one gets the leading singularity of $Z_0(g)$ is
\begin{equation}
	Z_0(g) \sim - \frac{4}{15} N^2 x^{\frac{5}{2}} + \cdots
\end{equation}
with $x=1 - \frac{g}{g_c}$.

It's also possible to determine $g_*$ with the orthogonal polynomial method.
Assume that one finds the family of orthogonal polynomials with the normalization
\begin{equation}
	\int (\mathrm{d}\lambda) \mathrm{e}^{-\frac{N}{g}\mathrm{tr}V(\lambda)}p_n(\lambda) p_m(\lambda) = s_n \delta_{nm}
.\end{equation}
To obtain the continuum limit, one must first take $N\to\infty$,
then replacing $s_n/s_{n-1}=f(\xi),\xi=n/N$ with a continuous function.
The matrix partition function $Z(g)$ then can be calculated from $f(\xi)$ through
\begin{equation}
	\lim_{N\to\infty} \frac{1}{N^2} \ln Z(g) \sim \int \mathrm{d}\xi (1-\xi) \ln f(\xi)
.\end{equation}
We use $\sim$ to ignore additive constant which is irrelavent to the asymptotic $g\to g_*$.
It turns out that around the critical point $g\approx g_*$,
$f(\xi)$ should have the following scaling
\begin{equation}
	f(\xi) - f_* \sim (g_* - g\xi)^{-\gamma}
.\end{equation}
where $f_*$ is some constant determined by the potential.
This will give the expected scaling of $\ln Z(g)$.

The definition of $g_*$ above based on $N\to\infty$.
Is it possible to define it for finite $N$?

\section{RG equation for continuum limit}

In this section, I review the RG method proposed in 
\urlref{https://arxiv.org/abs/hep-th/9206035}{Brezin92}.
This is proposed as another way to find the critical point of continuum limit,
so $N\to\infty$ is imposed in the calculation.
They use an even potential as example $V(\lambda)=\frac{1}{2}\lambda^2+\frac{g}{4}\lambda^4$,
whose critical point is $g_*=-\frac{1}{12}<0$.
It's hard to make sense of the matrix integral with $g<0$.

Another point I want to remark.
The direction of the flow is $N\to N-1$ which is opposite to the continuum limit $N\to\infty$. 
If the RG flow is interpreted as a result of coarse-graining,
it's impossible to reverse the flow.
Then how to argue that this method will give us information about the continuum limit?

I try to understand this point.
A matrix model defines by itself a sequence of models in terms of $N$.
By studying how the model behaves as $N\to\infty$, one finds scaling of $Z$ around certain coupling $g_*$.
However, there are other way to get a sequence of models:
given a $N\times N$ matrix, there are various way to embed it into a $(N+1)\times(N+1)$ matrix
or map a $(N+1)\times(N+1)$ matrix to a $N\times N$ matrix.
Then the action can be induced on the smaller matrices.
Although such a sequence must be ad hoc, especially when considering the matrix model action work for all $N$. 
However, it's natural to believe that the continuum limit should be independent of how the sequence defined.
In other word, starting from $N$-model, the induced $N-1$-models are different,
but the difference should tend to zero when $N\to\infty$.
I'm not sure to what extent we should believe this.

Another point is that
an RG method could allow us to study more general matrix models,
especially those without an eigenvalue representation.
This application of the RG method appears in
\urlref{https://arxiv.org/abs/1306.3019}{Kawamoto13}
in which the ``Yang-Mills two-matrices model'' is studied.
This model is characterized by a term $[A_1,A_2]^2$ in the action.
Such a term can arise from the Yang-Mills Lagrangian $F^2=(\mathrm{d}A)^2$ by ignoring the derivative terms.
The gauge group of this model is $U(N)$.
Until now, this is the only paper I find that using RG method to study multi-matrices model.

Intuition of RG flow starts from statistical model, 
but is also applied in studying various QFT. 
Lattice spacing is a natural scale in statistical model, 
which allows us to define a family of models with different scales but related with the same observables. 
Classical field theory, on the other hand, as a continuum theory, is not defined with a particular scale: 
a continuum model is defined naturally in all scales. 
However, one can still define a family of models by rescaling various ingredients: $x,\phi,g$ etc. 
There is a natural way to do that such that the functional form of the action doesn't change along the rescaling. 
This way of rescaling gives the notion of ``engineer dimension'' of those ingredients. 
Without any additional input, there maybe ambiguous in how to rescale: 
for example, if one ingredient (or a certain combination of two ingredients) is homogeneous, 
then one can give it an arbitrary engineer dimension. 
The most common input is like: the action should be dimensionless (in natural unit), 
but more may necessary to eliminate the ambiguity. 
But such rescaling does not lead to any interesting RG flow: 
no coarse-graining is involved because a continuum model is defined naturally in all scales. 
Instead, the rescaling invariance plays a more important role. 
When it comes to QFT, the theory is usually defined in the form of path integral of classical field theory. 
This is the formalism where the analogs with a statistical model is intriguing, 
which makes a nontrivial notion of RG flow possible while the rescaling invariance still present in the action. 
The invariant property in QFT is usually formulated in terms of the Ward identity, 
while the coarse-graining procedure naturally appears when one realizes that a regulator is necessary to make sense the path integral measure. 
Usually, regulator will introduce a particular scale in the theory, just like what happens in statistical model. 
A difference maybe, one doesn't think the cut-off scale is physical in QFT, but it may be taken as essential in a statistical model (a part of definition of the model). 
Fortunately, RG flow method doesn't care this philosophical aspect of that scale, but just ask what happens if a coarse-graining procedure taking us from one scale to a nearby another scale. 
Moreover, if there exist a fixed point, then one may relieve: a certain continuum theory is stable along the coarse-graining.


